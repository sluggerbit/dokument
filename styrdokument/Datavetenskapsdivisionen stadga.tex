\documentclass{dvd}

\KOMAoptions{
	headwidth = 18cm,
	footwidth = 18cm,
}

\begin{document}

	\title{Stadga}
	\author{Divisionsstämman}
	\date{Senast uppdaterad: 2021-11-19}

        \section{Ordlista för styr- och principdokument}

	\begin{description}
		\item[institutionen] \emph{Institutionen för Data- och Informationsteknik} vid \emph{Göteborgs universitet} och \emph{Chalmers Tekniska Högskola}.

		\item[IT-sektionen] IT-sektionen vid Göta studentkår.

		\item[kandidatprogrammet] Kandidatprogrammet \emph{Datavetenskapligt program} vid \emph{Göteborgs universitet}.

		\item[kåren] \emph{Göta studentkår} (organisationsnummer )

		\item[kårmedlem] Person som har \emph{medlemskap} eller \emph{stödmedlemskap} i Göta studentkår.

		\item[läsdag] Veckodag från och med måndag till och med fredag under läsåret.
					  De dagar som är arbetsfria enligt Chalmers kollektivavtal räknas inte som läsdgar.

		\item[läsår] Varar från första dagen av höstterminen till sista dagen av vårterminen.

		\item[termin] \emph{Chalmers Tekniska Högskola} fastställer datumet för varje termin.

		\item[universitetet] \emph{Göteborgs universitet}
	\end{description}

	\section{Allmänt}

	\begin{enumerate}[label=\arabic* §, ref=\arabic*]

		\item Föreningens namn är \emph{Datavetenskapsdivisionen}, hädanefter kallad ``divisionen''.

		Vid översättning till engelska ska divisionens namn översättas till \emph{Division for Computer Science}.

		\item Divisionen är en ideell, religiöst och partipolitiskt obunden sektionsförening vid Göta studentkår.

		\item Divisionens säte är i Göteborgs kommun.

		\item Divisionens syfte är öka sammanhållningen mellan medlemmarna och ta tillvara deras gemensamma intressen i utbildnings- och sociala frågor.

		Man ska även arbeta för att hjälpa medlemmarna ut på arbetsmarknaden.

		\item Divisionen ska aktivt verka för att samtliga studenter som uppfyller divisionens medlemskrav även blir medlemmar i divisionen.
	\end{enumerate}

	\section{Verksamhet}

	\begin{enumerate}[label=\arabic* §, ref=\arabic*]
		\item Denna stadga är underställd svensk lag och svenska myndigheters regler.
		Divisionens samtliga organ är underställda denna stadga.
		Absolut inget av divisionens organ får aktivt fatta ett beslut som strider mot denna stadga.
		Den ända gången det är tillåtet för divisionens organ att fatta ett beslut som bryter mot denna stadga är när man anser att stadgan bryter mot svensk lag eller svenska myndigheters regler.

		Samtliga organ behöver även följa de regler som definieras i styr- och principdokumenten publicerade i dokumentsamlingen.
		Undantag från divisionens styr- och principdokumenten publicerade i dokumentsamlingen kan endast fattas av det organ som antagit dokumentet, eller av ett organ överställt det organ som har antagit dokumentet.
		För att ett undantag ska kunna göras behöver man uppfylla samma krav som om man skulle ändra dokumentet.

		Verksamheten utövas genom organen

		\begin{itemize}
			\item divisionstämman;
			\item styrelsen;
			\item talmanspresidiet;
			\item valberedningen;
			\item kommittéer;
			\item intressegrupper.
		\end{itemize}

		\item Divisionsstämman är divisionens högsta beslutande organ.
		Styrelsen är divisionens högsta verkställande organ.
	\end{enumerate}

	\section{Medlemskap}

	\begin{enumerate}[label=\arabic* §, ref=\arabic*]

		\item Personer som är antagna på

		\begin{itemize}
			\item kandidatprogrammet;

			\item masterprogrammen \emph{Computer Science} och \emph{Applied Data Science} vid institutionen;

			\item något annat masterprogram vid institutionen men som innehar kandidatexamen från kandidatprogrammet; eller

			\item fristående kurs vid institutionen
		\end{itemize}

		och innehar kårmedlemskap har rätt att bli medlemmar i divisionen.
		Kårmedlemskapet behöver vara aktivt så länge man är medlem i divisionen.

		\item Man blir medlem i divisionen genom att kontakta divisionens styrelse.

		\item Medlemskap i divisionen sträcker sig från och med den första oktober till och med den sista september, hädanefter kallat ``medlemsperioden''.

		Person kan närhelst under medlemsperioden bli registrerad som medlem.

		Ett medlemskap kan avslutas aktivt genom att sktiftligen kontakta styrelsen eller avslutas passivt genom att inte förnya medlemskapet efter sista dagen i medlemsperioden.

		\item Medlemmar är skyldiga att följa divisionens beslut samt divisionens styr- och principdokument.
	\end{enumerate}

	\section{Ekonomi och ansvar}

	\begin{enumerate}[label=\arabic* §, ref=\arabic*]

		\item Divisionens räkenskapsår löper från och med den första januari till och med den sista december.

		\item För granskning av divisionens räkenskaper och förvaltning väljs av divisionsstämman en eller två revisorer.

		Mandatperioden är densamma som räkenskapsåret.
		Revisorerna behöver inte vara medlemmar i divisionen.

		\item Divisionens revisorer kan inte inneha något annat förtroendeuppdrag i divisionen under dennes mandatperiod som revisor.

		Revisorerna måste även vara myndiga och inte vara försatta i konkurs.
		De behöver även ha tillräckliga kunskaper i redovisning och ekonomiska förhållanden för att kunna utföra sina uppdrag.

		\item Att bevilja styrelsen ansvarsfrihet innebär att medlemmarna utifrån den information de har på ordinarie divisionsstämman godkänner att styrelsen har arbetat korrekt utifrån styrdokumenten.
		Om ansvarsfrihet inte beviljas markerar divisionsstämman att man inte är nöjd med styrelsens arbete, att man misstänker att något inte skötts korrekt/lagligt, att man vill underlätta framtida skadeståndsansvar eller att man har resterande uppgifter man vill att styrelsen skall ha löst innan ansvarsfrihet beviljas.
		Om ansvarsfrihet inte beviljats skall frågan behandlas på följande divisionsstämma.

		\item Divisionens firma tecknas av kassören och ordföranden var för sig.

		\item Om styrelsen anser att särskilda skäl föreligger, kan styrelsen även utse ytterligare person att teckna divisionens firma.

		\item Divisionens verksamhet och räkenskaper ska granskas av revisor utsedd av årsmötet.

	\end{enumerate}

	\section{Divisionsstämman}

	\begin{enumerate}[label=\arabic* §, ref=\arabic*]
		\item Divisionsstämman ska ha endast ett ordinarie sammanträde under höstterminen och endast ett ordinarie sammanträde under vårterminen.

		Stämman får endast sammanträda på en läsdag utanför tentaveckor.
		Man får heller inte ha ett sammanträde under veckan innan tentaveckorna.
		Sammanträdet får endast börja från och med klockan 17:15 lokal tid.
		Stämmolokalen måste ligga på campus Johanneberg, campus Lindholmen, eller lokal som kåren förfogar över.

		\begin{enumerate}[label=\theenumi~\alph* §, ref=\theenumi~\alph*]
			\item Divisionsstämmans ordinarie sammanträden annonseras av talmansspresidiet.

			Divisionsstämman måste sammanträda under höstterminen innan nästkommande räkenskapsår börjar.

			Divisionsstämman måste sammanträda under vårterminen innan den första maj.

			\item Extra sammanträden kan kallas av styrelsen, talmanspresidiet, revisorerna eller minst 25 av divisionens medlemmar.
		\end{enumerate}

		\item Senast dagen före ett nytt räkenskapsår börjar ska divisionsstämman beslutat om
		\begin{itemize}
			\item tillräckligt många styrelsemedlemmar för att styrelsen ska vara beslutsfattig; samt
			\item revisor
		\end{itemize}
		för det kommande räkenskapsåret.

		Om detta inte har skett förlängs mandatperioden för den nuvarande styrelsen och de nuvarande revisorerna till och med den dag då divisionsstämman väljer efterträdarna till förtroendeuppdragen.

		\item Divisionsstämmans sammanträde måste annonseras senast tio läsdagar innan sammanträdesdatumet.

		Motioner som ska behandlas under divisionsstämman måste skickas till styrelsen senast fem läsdagar innan mötet.

		Stämmohandlingarna ska offentliggöras senast två läsdagar innan divisionsstämman sammanträder.

		\item Divisionsstämman får endast sammanträda om kraven i detta kapitel och reglerna för divisionsstämmans sammanträden uppfylls.

		Om färre än 10 medlemmar är närvarande då beslut ska fattas, kan detta endast ske om ingen yrkar på bordläggning.

		\item Endast medlemmar eller förtroendevalda som kan styrka sin identitet med giltiga identitetshandlingar och kårmedlemskap med giltig kårlegitimation har närvarande-, yttrande-, förslags- och rösträtt på divisionsstämman.

		Röstning med fullmakt får inte ske.

		Röstning skall ske öppet, om inte sluten votering begärs.
		Röstsiffror ska alltid redovisas och godkännas av rösträknarna och mötesordförande.

		Personer som inte är medlemmar i divisionen får endast närvara och yttra sig under sammanträdet efter beslut av divisionsstämman.

		Vid lika röstutfall äger divisionssordförande utslagsröst, utom vid personval då lotten avgör.

		Alla frågor som behandlas av divisionsstämman avgörs med enkel majoritet om inget annat anges i denna stadga.
		Nedlagda röster räknas inte.

		Vid personval har mötesordförande rätt att välja att genomföra valet med valmetoden STV likt den som används vid Skottlands lokalval där kvoten beräknas enligt Droops metod och överskottsröster förs över enligt viktad inklusiv Gregorymetoden.

		\item Efter divisionsstämmans sammanträde är avslutat har talmanspresidiet tjugo läsdagar på sig att publicera det justerade stämmoprotokollet.
		Protokollet ska vara underskrivet av mötesordförande, mötessekreteraren samt vara justerat av två justerare.

		Styrelsen har tjugofem läsdagar på sig att publicera nya eller uppdaterade styrdokument.

		\item Senast sex månader efter räkenskapsårets början ska man på divisionsstämman redovisa divisionens årsredovisning samt revisorernas berättelse.
		Styrelsens ansvarsfrihet ska även beslutas om.
	\end{enumerate}

	\section{Talmanspresidiet}

	\begin{enumerate}[label=\arabic* §, ref=\arabic*]
		\item Talmanspresidiet består av

		\begin{itemize}
			\item talman;
			\item vice talman; och slutligen
			\item sekreterare.
		\end{itemize}

		\item Talmanspresidiets medlemmar väljs av divisionsstämman.
		Mandatperioden är densamma som räkenskapsåret.
		Medlemmarna i talmanspresidiet behöver inte vara medlemmar i divisionen.
	\end{enumerate}

	\section{Styrelsen}

	\begin{enumerate}[label=\arabic* §, ref=\arabic*]
		\item Styrelsen består av

		\begin{itemize}
			\item ordförande (även kallad divisionsordförande);

			\item vice ordförande;

			\item sekreterare;

			\item kassör (även kallad divisionsskassör);

			\item SAMO; och

			\item övriga styrelsemedlemmar.
		\end{itemize}

		\item Styrelsens medlemmar väljs av divisionsstämman.
		Mandatperioden är densamma som räkenskapsåret.
		Valbar till styrelsen är medlem i divisionen.

		Både ordförande och kassör måste vara myndiga.

		\item Styrelsen ska högst bestå av sju medlemmar.

		\item Styrelsen konstituerar sig själv.

		\item Styrelsen är endast beslutsmässig då samtliga styrelsemedlemmar har fått kallelsen till styrelsemötet och minst hälften av styrelsemedlemmarna är närvarande.
		Ordförande eller vice ordförande måste vara närvarande när beslut tas.
		
		Endast styrelsemedlemmar har rösträtt på styrelsemöten.
		Vid lika röstetal har divisionsordförande utslagsröst under styrelsemöten.
		Om inte divisionsordförande är närvarande har vice divisionsordförande utslagsröst.

		\item Medlemmar har närvaranderätt vid styrelsemöten, men måste avlägsna sig om mötesordförande ber om det.

		\item Medlem utanför styrelsen får endast yttra sig under styrelsemötet om denne har fått tillåtelse av mötesordförande.

		\item Protokollen ska vara signerade av mötesordförande, mötessekreterare samt justeras av en justerare.

		Protokoll från styrelsemöten ska vara tillgängliga för medlemmarna senast 10 läsdagar efter mötets avslut.

		\item Styrelsen har full insyn i divisionens alla organ och har närvaro- och yttranderätt på alla deras möten.
	\end{enumerate}

	\section{Valberedning}

	\begin{enumerate}[label=\arabic* §, ref=\arabic*]
		\item Valberedningens syfte är att ta fram förslag på personer till
		\begin{itemize}
			\item styrelsen;
			\item revisorer;
			\item talmanspresidiet; samt
			\item kommittéer.
		\end{itemize}

		Valberedningens förslag ska offentliggöras tillsammans med handlingarna som skickas ut inför divisionsstämmans sammanträde.

		\item Styrelsen utser sin representant tillika sammankallande i valberedningen.
	\end{enumerate}

	\section{Kommittéer}

	\begin{enumerate}[label=\arabic* §, ref=\arabic*]
		\item Divisionens kommittéer är de vars regler är publicerade i dokumentsamlingen.

		\item Kommittéer är skyldiga att följa divisionens stadga och regler, samt alla andra relevanta styrdokument.

		\item Alla kommittéer ska ha följande information i sina regler

		\begin{itemize}
			\item ett namn;

			\item ett specifikt och väldefinierat syfte;

			\item antal förtroendevalda;

			\item längd på mandatperiod.
		\end{itemize}

		Fler detaljer om kommittén kan skrivas ned i reglerna.

		\item Varje kommitté består av en ordförande och övriga förtroendevalda.

		Endast medlemmar kan väljas till förtroendeposter i en kommitté.
		Man kan endast vara ordförande för en kommitté i divisionen under den mandatperiod som man innehar posten.

		Divisionsstämman väljer ordförande och de övriga förtroendevalda till kommittéer.
		Valet av ordförande sker separat från valet av de övriga förtroendevalda.

		Vid skapande av en ny kommitté ska val av kommittéens ordförande och förtroendevalda införas i schemat för innevarande sammanträde.

		\item Kommittéer har rätt att använda sitt egna namn och symboler, samt divisionens namn och symboler.
	\end{enumerate}

	\section{Intressegrupper}

	\begin{enumerate}[label=\arabic* §, ref=\arabic*]
		\item Divisionens intressegrupper är de vars regler är publicerade i dokumentsamlingen.

		\item Intressegrupper är skyldiga att följa divisionens stadga och regler, samt alla andra relevanta styrdokument.

		\item Alla intressegrupper ska ha följande information i sina regler

		\begin{itemize}
			\item ett namn
			\item ett specifikt och väldefinierat syfte.
		\end{itemize}

		Fler detaljer om intressegruppen kan skrivas ned i reglerna.

		\item Endast divisionsmedlemar kan vara medlemmar i intressegrupper.

		\item Intressegrupper har rätt att använda sitt egna namn och symboler, samt divisionens namn och symboler.
	\end{enumerate}

	\section{Avgång, avstägning eller avsättning}

	\begin{enumerate}[label=\arabic* §, ref=\arabic*]
		\item Förtroendevald kan närhelst under verksamhetsåret självmant avgå från sina förtroendeuppdrag.
		Uppdraget anses avslutat när detta har meddelats skriftligen till styrelsen.

		\item Det enda sättet för divisionsstämman att avsätta en förtroendevald att lämna sina uppdrag under innevarande mandatperiod är genom misstroendevotum.

		Misstroendevotum kan endast väckas genom att ärendet inkommit som motion till divisionsstämman.
		Undantaget är om den förtroendevalde har blivit avstängd från sina förtroendeuppdrag av styrelsen.
		För att misstroendeförklaringen ska gå igenom krävs det att två tredjedelars majoritet av divisionsstämman är ense om beslutet.

		Misstroendevotum kan endast riktas mot
		\begin{itemize}
			\item divisionsordförande;
			\item divisionskassör;
			\item styrelsen som helhet;
			\item kommitté som helhet;
			\item talmanpresidiumet som helhet; och
			\item enskilda individer som styrelsen har beslutat om att stänga av från sina nuvarande förtroendeuppdrag.
		\end{itemize}

		Om medlem innehar flera förtroendeuppdrag i divisionen och blir misstroendeförklarad för någon av dem kan stämman välja att avsätta personen från en eller flera av uppdragen.

		\begin{enumerate}[label=\theenumi~\alph* §, ref=\theenumi~\alph*]
			\item Vid ordförandes avgång eller avstättning ska vice ordförande ta upp ordförandes arbetsuppgifter och rättigheter, tills dess att divisionsstämman utser en ny ordförande.

			\item Vid styrelsens avsättning ska en temporär styrelse väljas på innevarande sammanträde.
			Den temporära styrelsen tar över ordinarie styrelses befogenheter och skyldigheter tills dess en ny ordinarie styrelse har valts.
			Den temporära styrelsen eller talmanspresidiet ska annonsera ett extra sammanträde av divisionsstämman.
			Sammanträdet ska ske inom 15 läsdagar eller sista läsdag för terminen, vad som kommer först.
		\end{enumerate}

		\item Styrelsen har rätt att

		\begin{itemize}
			\item stänga av medlem från deras förtroendeuppdrag.
		\end{itemize}

		Detta kan endast ske om två tredjedelars majoritet av styrelsens samtliga medlemmar röstar för avstängningen eller uteslutningen.
		Personen som är föremål för ärendet har rätt att yttra sig till styrelsen innan beslut tas i ärendet.

		Om medlem blivit avstängd från sina uppdrag väcks automatiskt en misstroendeförklaring av personen vid nästa sammanträde av divisionsstämman.
		Om inte misstroendeförklaringen går igenom hävs avstägningen.

		\item Varken styrelsen eller divisionsstämman kan ta beslut om att utesluta en medlem ur divisionen då styrelsen skall arbeta för att främja kamratskapen medlemmarna emellan.
	\end{enumerate}

	\section{Dokument}

	\begin{enumerate}[label=\arabic* §, ref=\arabic*]
		\item Dessa stadgar kan endast ändras genom beslut av divisionsstämman med två tredjedelars majoritet vid två på varandra följande sammanträden.
		Ett av divisionsstämmans sammanträden behöver vara ordinarie.
		Det måste vara minst 10 läsdagar mellan sammanträderna.

		Antagen ändring av dessa dokument träder i kraft omedelbart under innevarande sammansträde av divisionsstämman.

		\item Styrdokument eller principdokument får införas i dokumentsamlingen eller ändras endast efter divisionsstämman fattat beslut med två tredjedelars majoritet.

		För att ett dokument ska gälla krävs det att dokumentet är publicerat i divisionens dokumentsamling.

		\item Styrelsen har rätt att endast göra redaktionella ändringar i dokument antagna av divisionsstämman.
		Ändringarna behöver beslutas om på styrelsemöte och dokumenteras tydligt i styrelsemötesprotokollet.

		\item Vid tvist om hur man ska tolka dokument ska dokumentet tolkas av en instans minst ett steg högre upp än den instans där tvisten uppkom för avgörande.

		Om det uppstår en tvist om hur man ska tolka dokument där endast divisionsstämman kan besluta om ändringar tolkas dokumentet av divisionssordförande för avgörande.
	\end{enumerate}

	\section{Upplösning}

	\begin{enumerate}[label=\arabic* §, ref=\arabic*]

		\item Förslag om divisionens upplösning får behandlas endast på årsmöte.

		\item Divisionen ska inte upplösas så länge minst fem medlemmar vägrar godkänna upplösningen.

		\item Vid upplösning ska divisionens skulder betalas.
		Därefter ska kvarvarande och icke-ekonomiska tillgångar förvaras av Göta studentkårs IT-sektion för användning endast till framtida studentorganisationer för datavetenskap vid \emph{Göteborgs universitet}.

	\end{enumerate}

\end{document}
