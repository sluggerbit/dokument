\documentclass{dvd}

\KOMAoptions{
    headwidth = 18cm,
    footwidth = 18cm,
}

\begin{document}

    \title{Regler för personuppgifter}
	\author{Styrelsen}
	\date{Senast uppdaterad: 2021-08-26}

    \section{Allmänt}

    \begin{enumerate}[label=\arabic* \S, ref=\arabic*]
        \item Att den personliga integriteten respekteras och att personuppgifter behandlas på rätt sätt anser Datavetenskapsdivisionen är viktigt.
		Den här informationen har syftet att i enlighet med EU:s dataskyddsförordning ge övergripande information om hur Datavetenskapsdivisionen behandlar de personuppgiftshandlingar som föreningen är personuppgiftsansvarig för.
		Informationen har även till syfte att ge information till de som Datavetenskapsdivisionen behandlar personuppgifter om, så att de vet vilka rättigheter de har.
    \end{enumerate}

    \section{Personuppgiftsansvar}

    \begin{enumerate}[label=\arabic* \S, ref=\arabic*]
        \item Datavetenskapsdivisionens styrelse är personuppgiftsansvarig för de behandlingar av personuppgifter som föreningen bestämmer ändamål och medel för.
		Du hittar kontaktuppgifterna till styrelsen i sidfoten av det här dokumentet och på föreningens hemsida.
    \end{enumerate}

    \section{Hur Datavetenskapsdivisionen behandlar personuppgifter}

    \begin{enumerate}[label=\arabic* \S, ref=\arabic*]
        \item Vid medlemsregistrering behandlar Datavetenskapsdivisionen personuppgifter och kontaktuppgifter.
        Detta görs för att föreningen ska kunna kontrollera identiteter vid sammanträden; för att kunna kontrollera att personen har kårmedlemskap; samt skicka ut medlemsinformation.

        \item Vid medlemsregistrering behandlar Datavetenskapsdivisionen personuppgifter och kontaktuppgifter.
        Detta görs för att föreningen ska kunna kontrollera identiteter vid sammanträden; för att kunna kontrollera att personen har kårmedlemskap; samt skicka ut medlemsinformation.

        \item Datavetenskapsdivisionen behandlar personuppgifter i samband med anmälan om prenumeration av föreningens mejlutskick.
		Behandlingen sker för att föreningen ska kunna administrera prenumerationen och skicka ut informationen.
		Den rättsliga grunden är att fullgöra det avtal som ingåtts i samband med anmälan som prenumerant.

		\item Divisionen behandlar personuppgifter och kontaktuppgifter i samband med ifyllnad av formulär.
        Behandlingen sker för att kunna genomföra arrangemang och undersökningar i enlighet med divisionens syfte, som är definierat i stadgan.
        Den rättsliga grunden är att fullgöra divisionens syfte.

        \item De kategorier av personuppgifter som behandlas är

		\begin{itemize}
		  \item namn;
		  \item personnummer; samt
		  \item kontaktuppgifter till enskilda som vänt sig till föreningen.
		\end{itemize}

		Om ärendet i grunden avser en organisation av något slag och en kontaktperson har utsetts för organisationen behandlas namn och kontaktuppgifter till kontaktpersonen.

		I handlingar och meddelanden som skickas in till föreningen förekommer ibland andra typer av personuppgifter.
		Dessa uppgifter hanteras endast genom att handlingen läggs in i det aktuella ärendet.
		Uppgifterna registreras inte särskilt och uppgifterna i den inkomna handlingen görs inte sökbara.

        \item Endast föreningens förtroendevalda medlemmar kommer att ta del av uppgifterna för att kunna utföra sina arbetsuppgifter.

		Datavetenskapsdivisionen använder sig i vissa fall av personuppgiftsbiträden.
		De personuppgiftsbiträden som anlitas får endast behandla personuppgifter i enlighet med de ändamål och instruktioner som föreningen har lämnat för behandlingen.
		Biträdet och de som agerar under biträdets ledning får vidare aldrig ta del av mer uppgifter än vad som krävs för utförande av den tjänst som avtalet med föreningen omfattar.
		När personuppgifter ska behandlas av ett personuppgiftsbiträde upprättas ett så kallat personuppgiftsbiträdesavtal.
		Datavetenskapsdivisionen använder personuppgiftsbiträden för olika typer av IT-tjänster.

		Som sektionsförening vid Göta studentkår (organisationsnummer 857200-4144) kommer medlemsregistret att delas med kåren för att verifiera att samtliga medlemar i Datavetenskapsdivisionen är medlemmar i Göta studenkår.

        \item Personuppgifter som ingår i medlemsregistret sparas endast tills medlemsperiodens slut.

		Personuppgifter som ingår i avtal sparas endast så länge de är nödvändiga för de ändamål som de behandlas för.

		Handlingar av ringa eller tillfällig betydelse gallras i regel direkt eller senast efter tolv månader.

		Personuppgifter om prenumeranter raderas vid avslutad prenumeration.

		Personuppgifter som ingår i övriga handling sparas endast så länge de är nödvändiga för de ändamål som de behandlas för.
		Handlingar som innehåller personuppgifter och som inte ska bevaras raderas eller rensas på personuppgifter.

    \end{enumerate}

    \section{Dina rättigheter}

    \begin{enumerate}[label=\arabic* §, ref=\arabic*]
        \item Som registrerad har du flera rättigheter.
		Om du som registrerad hos Datavetenskapsdivisionen vill utöva dina rättigheter eller har frågor som rör föreningens behandling av dina personuppgifter kan du vända dig till föreningens styrelse.
		Kontaktuppgifterna till styrelsen hittar du i sidfoten av föreningens dokument och på föreningens hemsida.

        \item Du kan begära att få ett besked om huruvida Datavetenskapsdivisionen behandlar personuppgifter som rör dig och i så fall få en kopia av dessa - ett så kallat registerutdrag - tillsammans med viss närmare information.

        \item Om du anser att personuppgifterna som rör dig är felaktiga eller ofullständiga kan du begära att få uppgifterna rättade eller kompletterade.

        \item När Datavetenskapsdivisionen behandlar personuppgifter inom ramen för att kunna bedriva föreningens verksamhet har du rätt att när som helst invända mot behandlingen.
		Om Datavetenskapsdivisionen inte kan visa att det finns tvingande, berättigade skäl att fortsätta att behandla uppgifterna måste föreningen upphöra med behandlingen.

        \item Du har i vissa fall, till exempel om du har invänt mot behandlingen, möjlighet att kräva begränsning av behandlingen av dina personuppgifter.
		Genom att begära en begränsning har du, i vart fall under en viss tid, möjlighet att stoppa Datavetenskapsdivisionen från att använda uppgifterna annat än för att exempelvis försvara rättsliga anspråk.
		Du kan även hindra föreningen från att radera uppgifterna, till exempel om du behöver uppgifterna för att kräva skadestånd.

        \item Du kan i vissa fall få dina personuppgifter raderade (``rätten att bli bortglömd'').
		När dina personuppgifter behövs för att föreningen ska kunna uppfylla juridiska krav har Datavetenskapsdivisionen ingen möjlighet att radera uppgifterna.
    \end{enumerate}
\end{document}
