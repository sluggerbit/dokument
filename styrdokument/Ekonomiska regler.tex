\documentclass{dvd}

\KOMAoptions{
	headwidth = 18cm,
	footwidth = 18cm,
}

\begin{document}

	\title{Ekonomiska regler}
	\author{Styrelsen}
	\date{Senast uppdaterad: 2020-02-24}

	\begin{enumerate}[label=\arabic* §, ref=\arabic*]
		\item Dessa regler beskriver hur föreningens pengar får användas.
		Det är tänkt som ett hjälpmedel för styrelsen, revisorerna, kommittéerna och intressegrupperna i deras arbete.

		\item Grundprincipen är att föreningens pengar är till föreningens samtliga medlemmar.

		\item Samtliga ledamöter i styrelsen är ansvariga för att

		\begin{itemize}
			\item Känna till föreningens ekonomiska regler.
		\end{itemize}
		\label{par:all-board-members-responsibility}

		\item Utöver kraven i \ref{par:all-board-members-responsibility} är föreningsordförande ansvarig för att

		\begin{itemize}
			\item hjälpa kassören med ekonomiska frågor vid behov.
		\end{itemize}

		\item Utöver kraven i \ref{par:all-board-members-responsibility} är kassören ansvarig för att

		\begin{itemize}

			\item fortlöpande kontrollera föreningens räkenskaper och bokföring;

			\item till varje årsmöte kunna redogöra för sektionens ekonomiska ställning; och

			\item utbilda nya föreningsfunktionärer i föreningens ekonomiska rutiner och blanketter.

		\end{itemize}

		\item Vid arrangemang anordnat av föreningen får arrangemangets budget belastas med utgifter för inköp av mat om arrangemanget slutresultat inte blir negativt.
		Storleken på matbidraget baseras på tidsåtgång för arrangemanget:

		\begin{itemize}
		\item 1-4 timmar: 30 SEK per person;

		\item 4-8 timmar: 60 SEK per person.
		\end{itemize}

		Matens införskaffning eller tillagning får endast medräknas i arrangemangets tidsåtgång om matservering är en del av arrangemanget.

		\item Alla samarbeten med organisationer utanför universitetet eller andra studentföreningar går via presidiet.

		\item Köp av produkter eller tjänster som inte krävs för den dagliga förvaltningen av föreningens verksamhet eller Monaden ska godkännas genom styrelsebeslut.
		Detta ska göras innan köpet genomförs.
		Vid omröstning av köpet har föreningsordförande och kassören enskilt vetorätt mot besluten.

		Medlem som önskar att föreningen ska köpa något kan skicka in en äskning till styrelsen.
		Äskningen ska behandlas på nästkommande styrelsemöte.

		\item När någon förtroendevald inhandlar något åt föreningens vägnar för föreningens dagliga verksamhet ska man först få ett skriftligt godkännande av kassören.
		Godkännandet kommuniceras via fysiska eller digitala medier.

		Alla utlägg för föreningen ska redovisas genom ifyllnad av anvisad blankett.

		Om man gör inköp till föreningen innan man har hunnit få ett godkännande från kassören gäller reglerna

		\begin{itemize}
		\item om inköpen understiger 200 SEK behöver endast kassören godkänna köpet i efterhand; och

		\item om inköpen överstiger 200 SEK behöver både kassören och ordföranden godkänna köpet i efterhand.
		\end{itemize}
	\end{enumerate}
\end{document}
