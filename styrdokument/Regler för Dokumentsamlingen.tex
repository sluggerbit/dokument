\documentclass{dvd}

\KOMAoptions{
	headwidth = 18cm,
	footwidth = 18cm,
}

\begin{document}

	\title{Regler för dokumentsamlingen}
	\author{Divisionsstämman}
	\date{Senast uppdaterad: 2021-11-19}

	\begin{enumerate}[label=\arabic* §, ref=\arabic*]
		\item Dokumentsamlingen är är ett begrepp på det repository som förvarar alla publika officiella dokument som Datavetenskapsdivisionen antar.
		Syftet med dokumentsamlingen är att förvara alla dokument i ett läsbart format på ett och samma ställe.

		Föreningens styrelse ansvarar för att underhålla samlingen och att den är åtkomlig för alla intresserade läsare.
		
		\item Samtliga officiella dokument ska använda divisionens dokumentmall
		Dokumentklassen utvecklas och förvaltas av styrelsen.

		Samtliga protokoll ska läggas upp med fysiska signaturer.
		
		\item Alla dokument ska skrivas med stadgan i åtanke.
		Ett ord definierat i Stadgans ordlista får inte användas med en annan definition än den som står i ordlistan.

		\item Principdokument är dokument där föreningens grundprinciper finns nedskrivna.
		Dessa grundprinciper beskriver den andemening som ligger till grund till föreningens verksamhet.

		Med styrdokment menas dokument som konkret reglerar förenignens verksamhet.
		Styrdokument är konkretiseringar av principdokumenten.

		Protokoll är de officiella mötesanteckningar som förs under föreningsstämmans sammanträden och de som förs under styrelsemöten.
	\end{enumerate}
\end{document}
