\documentclass{dvd}

\KOMAoptions{
  headwidth = 18cm,
  footwidth = 18cm,
}

\begin{document}
	\title{Regler för styrelsen}
	\author{Divisionsstämman}
	\date{Senast uppdaterad: 2020-05-13}

	\begin{enumerate}[label=\arabic* §, ref=\arabic*]
		\item Styrelsen som helhet ansvarar för att
		\begin{itemize}
			\item föreningens syfte -- som definierat enligt stadgarna -- efterlevs och genomsyrar allt arbete inom föreningen;

			\item att föreningens stadgar, regler och principer efterföljs;

			\item de vardagliga uppgifterna inom föreningen sköts;

			\item beslut fattade av föreningsstämman verkställs;

			\item välja studentrepresentanter till programrådet;

			\item nominera studentrepresentanter till institutionsrådet; och
		\end{itemize}

		\item Ordförande har det yttersta ansvaret för föreningen.
		Vice ordförandes uppgift är att hjälpa ordförande i dennes arbete, samt överta ordförandes uppgifter i dennes frånvaro.

		Utöver följande uppgifter har även ordförande uppgifter som beskrivs i de ekonomiska reglerna.

		Ordförandes uppdrag är att
		\begin{itemize}
		\item vara föreningens ansikte utåt och representera föreningen när möjlighet ges;

		\item kalla till styrelsemöten och agera ordförande under dessa;

		\item ta fram utkast till nya styrdokument eller föreslå ändringar i befintliga styrdokument vid behov;

		\item skicka in ansökan till sektionsstyrelsen för att förnya föreningens sektionsföreningsstatus;

		\item bevaka de ärenden som behandlas av kårfullmäktige och kårstyrelsen; och slutligen

		\item inför varje sammanträde av föreningsstämman producera en verksamhetsrapport för att sedan presenteras inför föreningsstämman under sammanträdet.
		\end{itemize}

		\item Sekreterarens uppdrag är att
		\begin{itemize}
		\item agera mötessekreterare under styrelsemöten, och därefter ansvara för att mötesprotokollen publiseras;

		\item sammla in alla styrdokument och mötesprotokoll som föreningens organ producerar för att göras lättåtkomliga för dess medlemmar, verksamhetsrevisorer samt kommande styrelser;

		\item förvalta föreningens digitala plattformar och se till att regler och rutiner följs.
		\end{itemize}

		\item Kassörens uppgifter är dokumenterade i föreningens ekonomiska regler.

		\item SAMOs uppgifter är dokumenterade i dokumentet \emph{Arbetsbeskrivning för SAMO} som \emph{GUS} (Göteborgs universitets studentkårer) har fastställt.

		\item Övriga ledamöter har till uppgift att representera kommittéerna och intressegrupperna i styrelsen.
		De övriga ledamöterna kan även få andra uppgifter under sin mandatperiod gång.
	\end{enumerate}
\end{document}
