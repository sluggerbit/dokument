\documentclass{dvd}

\KOMAoptions{
	headwidth = 18cm,
	footwidth = 18cm,
}

\begin{document}

	\title{Regler för divisionssstämman}
	\author{Divisionsstämman}
	\date{Senast uppdaterad: 2020-05-13}

	Om divisionsstämmans möte genomförs med hjälp av ett eller flera digitala verktyg får mötespresidiet välja att göra avsteg från de paragrafer som inte är applicerbara med de digitala verktygen.
	Man måste dock upprätthålla dessa reglers andemening.

	\section{Inför sammanträdet}

	\begin{enumerate}[label=\arabic* §, ref=\arabic*]
		\item Annonseringen av sammanträdet ska inneålla

		\begin{itemize}
			\item datum;
			\item tid; och
			\item plats
		\end{itemize}

		för sammanträdet.

		\item Sammanträdets annonsering ska ske

		\begin{itemize}
			\item på divisionens hemsida;
			\item i Monaden; och
			\item via divisionens mejllista för utskick (utskick@dvet.se).
		\end{itemize}

		\item Möteshandlingarna ska skickas ut via divisionens mejllista för utskick (utskick@dvet.se) innan divisionsstämmans sammanträde.
		Hur lång tid innan regleras av stadgan.

		Möteshandlingarna består av

		\begin{itemize}
			\item kallelsen;
			\item förslag på mötesschema;
			\item verksamhetsrapport;
			\item ekonomisk redovisning av divisionens nuvarande räkenskaper;
			\item eventuella förslag på personer till förtroendeuppdrag;
			\item propositioner;
			\item motioner;
			\item interpellationer och svar på dessa; och slutligen
			\item skriftliga frågor och svar på dessa.
		\end{itemize}

		\item Förslaget på mötesschema ska innehålla

		\begin{itemize}
			\item sammanträdets öppnande;
			\item fastställande av röstlängden;
			\item godkännande av kallelsen;
			\item godkännande av mötesschemat;
			\item val av mötesordförande;
			\item val av mötessekreterare;
			\item val av två justerare tillika rösträknare;
			\item presentation av verksamhetsrapporter;
			\item presentation av divisionens räkenskaper;
			\item behandling av beslutsärenden; samt
			\item debatt kring eventuellt inkomna interpellationer.
		\end{itemize}

		\item Beslut får inte fattas i fråga som inte varit utsänd som beslutspunkt.

		Ingen ska på sittande möte kunna lyfta en fråga till beslut som inte funnits i handlingarna, divisionsstämman ska ha haft möjlighet att läsa på, fundera och skaffa sig en åsikt före mötet.

		\item För att en fråga ska kunna behandlas på mötet måste den finnas med på den slutgiltiga mötesschemat.
		Om beslut ska fattas i frågan behöver divisionsstämman ett underlag med förslag till beslut från styrelsen.
		Exempel på olika sätt att lyfta en fråga till divisionsstämman lyder nedan:

		Som en motion är förslag till handling, diskussion, uppmaning, uttalande, information, åtgärd eller dylikt.
		Styrelsen lämnar yttrande över motionen.
		Motionen behandlas och beslut fattas av divisionsstämman.
		För att motionen ska kunna behandlas på mötet måste den finnas med på den slutgiltiga mötesschemat.

		En interpellation är en frågeställning ställd till förtroendevald i Datavetenskapsdivisionen kring ett ämne som ska debatteras på nästkommande sammankomst av divisionsstämman.
		Interpellationen ska besvaras skriftligt och ska skickas med i handlingarna.
		Interpellationsdebatten äger rum på divisionsstämmans sammanträde där samtliga medlemmar i divisionsstämman har möjlighet att yttra sig.
		Divisionsstämman beslutar om frågan ska anses tillräckligt besvarad.
		Divisionsstämman kan besluta om återremittering vid icke tillfredsställande svar.
		Ofta leder en interpellation till att det skickas en motion till divisionsstämman för framtida beslut för att åstadkomma förändring.
		Det är ett grundläggande verktyg för ansvarsutkrävning.

		En skriftlig fråga är en frågeställning som är i behov av ett längre svar och är av allmänt intresse för divisionsstämman.
		Frågan ska besvaras skriftligen utan debatt eller kontrareplik från den som ställt frågan.
		Det innebär att ingen allmän debatt kommer att hållas på divisionsstämman men att samtliga får ta del av svaret skriftligen i handlingarna som skickas ut till medlemmarna och frågan blir muntligen besvarad på divisionsstämmanes sammanträde.

		Medlem har möjlighet att ta upp en enklare fråga.
		En enklare fråga ska påvisas till divisionsordförande som bedömer om frågan kan besvaras utan vidare behandling av divisionsstämman eller om frågan bör justeras in i mötesschemat för kortare diskussion.
		Divisionsordförande ska vid om frågan bedöms enklare ställa frågan till divisionsstämman om det är divisionsstämmans mening att behandla frågan under mötet.
		Detta gäller inte sakupplysning eller ordningsfråga.

		\item Styrelsen ska lämna ett yttrande över inkomna motioner till divisionsstämman.
		Divisionsstämman ska före mötet känna till styrelsens hållning i en viss fråga.
	\end{enumerate}

	\section{Talarprinciper}

	\begin{enumerate}[label=\arabic* §, ref=\arabic*]
		\item Principen om första- och andratalarlista följs.

		De som begär ordet för första gången i ett ärende antecknas på en talarlista och de som redan har yttrat sig i ärendet, på en annan talarlista.
		Första talarlistan ska alltid vara tom innan någon på den andra talarlistan får ordet.
		Tanken med förstatalarlista är att fördela ordet jämnare mellan flera personer och inte bara två, tre stycken ska prata hela tiden.

		Mötesordförande äger rätt att besluta om talartidsbegränsning i samråd med tidsunderlättaren.

		\item Den som har blivit apostroferad i visst anförande har rätt till replik om högst en minut.
		Replik ska begäras i omedelbar anslutning till det aktuella anförandet.
		Kontrareplik, det vill säga replik på replik, beviljas inte.

		När person A omnämns så blir A apostroferad. A har då rätt att svara; att begära replik.
	\end{enumerate}

	\section{Principer för diskussion}

	\begin{enumerate}[label=\arabic* §, ref=\arabic*]
		\item Sakupplysning bryter talarordning.
		Vid sakupplysning får talare endast upplysa om fakta och inte uttrycka åsikter eller på annat sätt föra debatt.

		En sakupplysning är att någon upplyser övriga ledamöter om något som har relevans för den diskussion som pågår.
		Detta används för att diskussionen ska föras på korrekt grund.

		\item Ordningsfråga bryter debatt i sakfråga och ska avgöras innan annan fråga tas upp till behandling.

		En ordningsfråga berör omständigheterna eller förutsättningarna för mötet.
		Det kan exempelvis vara en begäran om paus i mötet, att öppna ett fönster eller att man tycker att ett beslut ska fattas i annan ordning.

		\item  Vid beslut om streck i debatten ska ordföranden lämna tillfälle åt dem som önskar att framställa yrkanden, läsa upp dessa och bereda dem som så önskar tillfälle att begära ordet.
		När streck i debatten ska sättas får alla möjlighet att sätta upp sig på talarlistan, och endast en gång.
		Det är inte tillåtet att stryka sig från listan förrän streck har satts.
		Sedan detta skett kan nya yrkanden inte läggas fram och nya talare kan inte få ordet i den aktuella frågan.
		Apostrofering efter streck i debatten är inte tillåten, inte heller replik.
	\end{enumerate}

	\section{Förslagsbehandling}

	\begin{enumerate}[label=\arabic* §, ref=\arabic*]
		\item Yrkanden ska göras skriftligt om talmännen eller mötespresidiets sekreterare begär det.

		Ett yrkande är ett förslag rörande någon av dagordningens punkter, från någon på mötet med yrkanderätt.

		\item När motion behandlas har motionären rätt att först föredra sin motion.

		\item Styrelsens föredragande har rätt att, efter motionären, hålla ett inledande anförande och få sista anförandet.

		\item Vid proposition har styrelsens föredragande alltid första och sista ordet om så begärs.

		\item Antagna ändringsförslag får inte förändra en motions andemening.
	\end{enumerate}

	\section{Omröstningar}

	\begin{enumerate}[label=\arabic* §, ref=\arabic*]
		\item Generellt sker omröstningar öppet med acklamation.
		Votering begärs öppet och sker genom handuppräckning.

		Acklamation innebär att man röstar genom att ropa ``ja'' på mötesordförandens fråga om man vill bifalla ett förslag eller avslå detsamma.
		Votering innebär att omröstningen görs så att mötesordföranden kan räkna antalet röster mer exakt än vid acklamation.
		Votering begärs av en medlem om denne inte håller med om mötesordföranes bedömning av röstfördelningen.

		Medlem har även rätt att yrka att vid omröstningen genomförs via sluten votering.

		Om någon har begärt att votering ska göras så måste voteringen genomföras.
		Man kan inte dra tillbaka begäran om att genomföra votering.

		\item Vid omröstning som utfaller med lika röstetal i en fråga som inte tål uppskov, äger divisionsordförande utslagsröst.
		Tål frågan uppskov, bordläggs ärendet till nästkommande sammankomst av divisionsstämman.
	\end{enumerate}

	\section{Personval}

	\begin{enumerate}[label=\arabic* §, ref=\arabic*]
		\item Personval avgörs alltid med sluten omröstning om inte divisionsstämman enhälligt beslutar annat.

		Vid val av person använder man sig i regel av sluten omröstning eftersom det kan vara känsligt att välja en person framför en annan.
		Om det inte är konkurrens om en plats så kan man hålla valet öppet, med acklamation

		För att mötet ska kunna diskutera kandidaterna fritt kan kandidaterna lämna rummet medan en sådan diskussion förs.
		Vid diskussion om kandidater ska det i huvudsak talas FÖR dessa.
		Dock kan relevant information som kan påverka ämbetet i fråga även lyftas fram.
		Detta ska inte övergå till personliga påhopp eller icke-relevant kritik.

		När divisionsstämman ska gå till beslut ska kandidaterna lämna rummet om de inte redan har gjort det.
		Efter divisionsstämman har fattat ett beslut är kandidaterna välkomna tillbaka in i mötessalen.

		När val av person har gått till omröstning ska alltid vakantsättning av posten vara ett alternativ.

		\item Personval med sluten omröstning genomförs med valsedlar.

		\item Vid val till likvärdiga poster ska listval tillämpas.
		Vid listval ska giltig avlagd röst uppta lika antal namn som det antal poster som ska tillsättas.
		En eventuell vakantsättning av platser ska avgöras innan listvalet förrättas.

		\item Som giltiga röstsedlar vid sluten omröstning gäller inte

		\begin{itemize}
			\item märkta röster;
			\item röster som upptar fel antal namn;
			\item blanka röster; och
			\item röster som inte är utformade enligt fullmäktigeordförandens föreskrifter.
		\end{itemize}
	\end{enumerate}

	\section{Jäv}

	\begin{enumerate}[label=\arabic* §, ref=\arabic*]
		\item Vid jävsituation ska medlem justera ut sig från röstlängden under den punkten.
		Medlem kan väcka ordningsfråga där jäv riktas mot en annan medlem.
		Om ledamot som jäv riktas mot inte vill justera ut sig avgör divisionsstämman om jävsituation har uppstått och om medlemen ska justeras ut under beslutspunkten.

		När divisionsstämman ska gå till beslut ska medlemmarna som har bedömt sig själva eller av divisionsstämman bedömts jäviga lämna rummet.
		Efter divisionsstämman har fattat ett beslut är de jäviga välkomna tillbaka in i mötessalen.
	\end{enumerate}

	\section{Övrigt}

	\begin{enumerate}[label=\arabic* §, ref=\arabic*]
		\item Eventuella reservationer och protokollsanteckningar ska anmälas i direkt samband med beslutet som reservationen eller protokollsanteckningen berör.
		Denna ska sedan inkomma skriftligt innan mötets avslutande.

		Medlemmar reserverar sig mot ett beslut som den inte vill ta ansvar för.
		En reservation förs in i protokollet och visar eftervärlden att personen är så starkt emot beslutet att denne reserverade sig och därmed inte är ansvarig för eventuella positiva eller negativa konsekvenser av beslutet.

		En protokollsanteckning kan förtydliga något som annars inte tas med i protokollet, exempelvis att man hade en annan åsikt än den som vann bifall, men ändå kan tänka sig att ta det kollektiva ansvar som medlem normalt innebär.
	\end{enumerate}
\end{document}
