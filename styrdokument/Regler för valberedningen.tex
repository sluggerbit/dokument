\documentclass{dvd}

\KOMAoptions{
  headwidth = 18cm,
  footwidth = 18cm,
}

\begin{document}

	\title{Regler för valberedningen}
	\author{Divisionsstämman}
	\date{Senast uppdaterad: 2020-05-13}

    \begin{enumerate}[label=\arabic* §, ref=\arabic*]
        \item Valberedningens syfte är att vara det organ som tar fram och presenterar förslag på

		\begin{itemize}
			\item styrelsens;
			\item talhenspresidiet;
			\item revisorernas; och
			\item kommittéers
		\end{itemize}

		sammansättning under nästkommande mandatperiod.

		De olika organen bedriver själva aspirantperioder där de bjuder in medlemmar för att passivt observera aspiranterna när de deltar i olika aktiviteter.
		Organen ska sedan förmedla dessa intryck till valberedningen som går vidare och interjuvar aspiranterna som fortfarande är intresserade.
		Därefter presenterar man sitt förslag på nästkommande föreningsstämma.

        \item Enligt stadgarna utser styrelsen sin representant i valberedningen.
		Denna representera ska även agera sammankallade, vilket är den titel som används för ordförande i valberedningar.

        \item Varje kommitté ska utse en person som ska representera kommittén i valberedningen.
		Representanternas mandatperioder i valberedningen är desamma som deras mandatperioder i kommittéerna de representerar.
    \end{enumerate}
\end{document}
