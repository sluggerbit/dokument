\documentclass[protokoll]{dvd}

\KOMAoptions{
	headwidth = 18cm,
	footwidth = 18cm,
}

\begin{document}

\title{Divisionsstämmans möte 1}
\subtitle{2023}
\author{Divisionsstämman}
\date{2023-02-22}

\textbf{Datum:} \csname @date\endcsname\\
\textbf{Tid:} 17:17\\
\textbf{Plats:} Monaden (ev. EA)\\
\textbf{Styrelsemedlemmar:}
\begin{närvarande_förtroendevalda}
\förtroendevald{Ordförande}{Samuel Hammersberg}{}
\förtroendevald{Vice ordförande}{\emph{Vakant}}{}
\förtroendevald{SAMO}{Tekla Siesjö}{}
\förtroendevald{Kassör}{Lukas Gartman}{}
\förtroendevald{Sekreterare}{\emph{Vakant}}{}
\end{närvarande_förtroendevalda}

% \textbf{Närvarande övriga medlemmar:}
% \begin{närvarande_medlemmar}
% \end{närvarande_medlemmar}

\section{Öppnande av möte}
Mötet beräknas öppnas av Samuel Hammersberg 17.17

\section{Formalia}

\subsection{Divisionsstämmans beslutbarhet}

6 kap. i stadgan definierar regler Divisionstämman.

Den 1 februari kallade styrelsen till divisionsstämma genom att skriva i discordservern \emph{MonadenDV};

Tyvärr kunde inte ett utskick ske över mail då divisionen har förlorat
access till studenternas maillistor då när Göteborgs Universitet
bytte över till Outlook, ändrades de även reglerena angående mailutskick
till maillistor. Styrelsen jobbar med att fixa access igen.

Dessa möteshandlingar ska ha skickats ut under måndagen den 20 februari.

\subsubsection*{Förslag till beslut}

\begin{attsatser}
    \item divisionsstämman har uppnått kraven i stadgan för att få hålla möte, och är därmed beslutbar.
\end{attsatser}

\subsection{Fastställande av mötesschema}

För att divisionsstämman ska kunna fatta ett beslut eller protokollföra en diskussion behöver punkten i mötesschemat där stämman ska fatta beslut vara inlagd eller föras in i mötesschemat senast vid den här punkten.

\subsubsection*{Förslag till beslut}

\begin{attsatser}
    \item mötesschemat fastställs utan några ändringar
\end{attsatser}


\subsection{Val av mötesordförande}

Mötesordförande har till uppgift att leda Divisionsstämmans sammankomst.
Hen ansvarar för att mötesformalia sköts.

\subsubsection*{Förslag till beslut}

\begin{attsatser}
    \item Samuel Hammersberg väljs till mötesordförande.
\end{attsatser}


\subsection{Val av vice mötesordförande}

Vice mötesordförande hjälper mötesordförande med att hålla talarlistan, och att alla får komma till tals.

\subsubsection*{Förslag till beslut}

\begin{attsatser}
    \item inag förslag från styrelsen innan mötet.
\end{attsatser}

\subsection{Val av mötessekreterare}

Mötessekreteraren har till uppgift att anteckna diskussioner, beslut, och eventuella reservationer under mötet.

\subsubsection*{Förslag till beslut}

\begin{attsatser}
    \item Tekla Siesjö väljs till vice mötesordförande.
\end{attsatser}

\subsection{Val av protokolljusterare}

Protokolljusterare har till uppgift att kontrollera att protokollet i slutändan reflekterar de faktiska besluten och diskussionerna som fördes under sammanträdet; samt agera rösträknare vid slutna omröstningar.
Utöver protokolljusterarna så ska mötesordförande och mötessekreteraren signera protokollet.
Vid Divisionsstämmans sammanträden ska det vara två justerare.
Mötesordförande och mötessekreteraren kan inte vara justerare.

\subsubsection*{Förslag till beslut}

\begin{attsatser}
    \item Inga förslag från styrelsen innan mötet.
\end{attsatser}

\newpage

% \section{Rapporter}
% \subsection{Arbetsgrupp för logga arbete}
% Under förra stämman så startades det en arbetsgrupp styrd av Tim Persson
% med syfte att ta fram en ny logotyp för divisionen. Under kommande stämman
% kommer en presentation framföras av gruppen för att visa hur arbetet har gått.

% \section{Rapporter}
% 
% \subsection{Styrelsen}
% 
% \subsubsection*{Verksamhetsrapport}
% 
% Styrelsen har haft fyra möten sedan Divisionens stämma den 17 maj.
% Protokollen från mötena finns (snart) tillgängliga för allmänheten på styrelsens drive.
% Länk till styrdokument och protokoll: \verb|https://drive.google.com/drive/folders/1o_lLM7g7ph-xdxgut3E_BU0ywichWYTX?usp=sharing|
% 
% Efter en lång kamp, så har vi äntligen fått en rullstolsramp till monaden, men kampen är inte över än, då vi fortfarande behöver ett större kök.
% 
% Samuel har varit på institutionsråd och programråd. På programrådet så togs saker som introkursen, obligatoriska kurser och möjligheten om ett tillägg av en kurs inom open source licenser. Vi åt väldigt mycket fika.
% 
% Mottagningen har även gått bra, och alla kommittéer har gjort ett bra jobb! Verkar även som att de fortsätter att rulla på bra!
% 
% \newpage
% 
% \subsubsection*{Ekonomi}
% 
% Den gångna tiden har varit en ny epok för Datavetenskapsdivisionen i avseende finanser och ekonomi. Mottagningsbudgeten 2022 var i relation till tidigare mottagningsbudgetar, enorm(!). Av vår tilldelade Göta-budget, för mottagningen, om 30tkr har 14tkr spenderats. Av vår tilldelade institutions-budget om 78tkr har 71tkr spenderats. Det totala beloppet för mottagningen i skrivande stund är alltså 85tkr. Totalt för året har vi haft kostnader om 95tkr och inkomster om 92tkr. Med det sagt, så har vi utstående inbetalningar från Göta som gör att summan förväntas landa på plus för kalenderåret.
% 
% Det nya bokföringssystemet har varit en lärandeprocess men då vi inte har något bokföringskrav så har vi tämligen lösa regler för vad som "krävs" och vi kan därmed utefter behov skapa konton och resultatgrupper för att göra resultrapporter "berättande" för medlemmar. En struktur för bokföringen är nu, och kommer förbli, en levande mall där tydlighet gentemot medlemmarna är en stor del av anledningen till bokföringssystemet. Jag tror och hoppas att nästa kassör med enkelhet kommer kunna fortsätta utveckla strukturen.
% 
% Då detta är min sista divisionsstämma som styrelsemedlem och kassör, så önskar jag att tacka för mig.
% 
% mvh, Morgan
% 
% (Notera att siffrorna är levande stämmer för 2022-11-20)
% 
% Tillägg: utöver mottagningen är ekonomin oförändrad, speciellt då det inte har gått så många skolveckor sedan förra stämman.
% 
% \newpage
% 
% \subsection{Verksamhetsrapporter av kommittéer}
% 
% \subsubsection*{DVRK}
% Muntlig verksamhetsrapport av DVRKs ordförande Leo Mirzajanzadeh.
% 
% \subsubsection*{ConCats}
% Muntlig verksamhetsrapport av Tekla Siesjö.
% 
% \subsubsection*{Femme++}
% Muntlig verksamhetsrapport av Femme++s ordförande Emmie Berger.
% 
% \subsubsection*{DVArm}
% Muntlig verksamhetsrapport av DVArms ordförande Albin Otterhäll.
% 
% \subsubsection*{Mega6}
% Muntlig verksamhetsrapport av Petter Blomkvist.
% 
% \newpage
% 
\section{Beslutsärenden}

Enligt Stadgan måste ändringar av Stadgan röstas igenom på två av Divisionsstämmans varandra följande möten.
Om en beslutpunkt innehåller ``första läsningen' innebär det att det är första gången beslutet tas upp för omröstning.
Om en beslutspunkt innehåller ``andra läsningen'' innebär det att beslutspunkten har röstats igenom förra stämmomötet, och beslutet behöver bekräftas för att gå igenom.

\subsection*{Proposition: Förlänging av tiden på väggundersökning}
Då denna stämman var på begäran har inte styreslen hunnit komma till någon
konkret slutsats om frågan om att riva väggen mellan stora och lilla rummet i
Monaden och skulle på grund av det vilja föreslå att arbetet kan fortsätta
till nästkommande stämma.

\subsubsection*{Förslag till beslut}
\begin{attsatser}
    \item att uppdraget för att undersöka frågan om att riva väggen
    fortsätter till nästkommande stämma
\end{attsatser}

\subsection*{Motion: Uppdatering av logotyp}
Under förra stämman så startades det en arbetsgrupp styrd av Tim Persson
med syfte att ta fram en ny logotyp för divisionen. Under kommande stämman
kommer en presentation framföras av gruppen för att visa hur arbetet har gått.

Gruppen vill även ställa beslutsfrågan om en av dessa presenterade loggor
kan användas som den slutgiltiga loggan, eller om en av loggorna ska användas som
bas för de resterande arbete.

\subsubsection*{Förslag till beslut}
\begin{attsatser}
    \item bestämma vilken logga som ska användas eller användas som bas
    \item förlänga arbetsgruppens arbete till nästa stämma
\end{attsatser}

\subsection{Motion: Inval av DVRK'23}

DVRK'22 har hållit och slutfört aspning (aspirantperioden) för att kunna hitta efterträdare till nästkommande verksamhetsår, och för att kunna lämna en rekommendation till stämman för val av kommittéordförande.

DVRK'22 förordar Tim "Tim" Persson som ordförande med motiveringen att med ett starkt driv; många ideér; och med förtroende hos de andra asparna är perfekt för att leda arbetet med att organisera höstens Mottagning.

DVRK'22 förordar Oscar "Oscar" Rei som kassör och resurschef med motiveringen att det viktigaste för han är att Mottagningen blir storslagen, och vår bedömning är att DVRK kommer att lyckas bäst om Oscar räknar pengarna.

DVRK'22 förordar Sebastian "Sebbe" Pålsson som propagandaminister med motiveringen att han är riktigt taggad på att tagga reccar att hänga med på Mottagningen.
Med storslagna idéer om nymodigheter som Instagram är DVRK'22 säkra på att med hjälp av Sebbe kommer reccarna att slåss om att få vara med på Mottagningen!

DVRK'22 förordar Nils "Nils" Lyrevik som eventchef med motiveringen att det är få personer i Monaden som ger ett så festtaggat intryck som Nils, och vi är säkra på att han kommer med stor glädje se till att DVRK arrangerar storslagna arrangemang.

Det är helt upp till nästkommande ordförande vilka andra personer som blir invalda under mandatperioden 2023, och posterna bestäms sedan av kommittéens medlemmar.
DVRK'22s rekommendationer är vårt råd till nästkommande kommittéordförande.

Vi önskar nästkommande år lycka till!

\emph{Albin "Slaget" Otterhäll, DV'18} \\
\emph{Kristoffer "KG" Gustafsson DV'20} \\
\emph{Leo "Leo" Mirzajanzadeh DV'21}\\
\emph{DVRK'22}

\subsubsection*{Förslag till beslut}
\begin{attsatser}
    \item Att välja Tim "Tim" Persson till ordförande för DVRK under verksamhetsåret 2023.
\end{attsatser}

% \subsubsection*{Beslut}
% \begin{attsatser}
%     \item styrelsen får i uppdrag att driva frågan om att väggen mellan det stora och det lilla rummet rivs och att en permanent skärmvägg installeras i dess ställe.
% \end{attsatser}

\subsection{Val of revisorer för verksamhetsåret 2023}
Vi behöver revisorer för 2023 så att stämman har koll på att styrelsen sköter sitt jobb!

\subsubsection*{Förslag från styrelsen}
Under tidigare stämmor har vi valt att vakantsätta denna posten, så vi i styrelsen tänker
att vi gör det även denna gången.

\begin{attsatser}
    \item posten vakantsätts även denna divisionsstämman
\end{attsatser}

\subsubsection*{Förslag till beslut}

\begin{attsatser}
    \item punkten bordläggs till nästa divisionsstämma
\end{attsatser}

%\section{Diskussioner}\label{sec:discussioner}

%\section{Avslutande av möte}

%Mötet avslutades klockan 20.10

%\stämmosignaturer

\end{document}
