\documentclass[protokoll]{dvd}

\KOMAoptions{
	headwidth = 18cm,
	footwidth = 18cm,
}

\begin{document}

\title{Divisionsstämmans möte 2}
\subtitle{2022}
\author{Divisionsstämman}
\date{2022-11-24}

\textbf{Datum:} \csname @date\endcsname\\
\textbf{Tid:} 17:17\\
\textbf{Plats:} Monaden (ev. EA)\\
\textbf{Styrelsemedlemmar:}
\begin{närvarande_förtroendevalda}
	\förtroendevald{Ordförande}{Samuel Hammersberg}{Ja}
	\förtroendevald{Vice ordförande}{\emph{Vakant}}{}
	\förtroendevald{SAMO}{Tekla Siesjö}{Ja}
	\förtroendevald{Kassör}{Morgan Thowsen}{Ja}
	\förtroendevald{Sekreterare}{Sebastian Selander}{Ja}
\end{närvarande_förtroendevalda}

\textbf{Närvarande övriga medlemmar:}
\begin{närvarande_medlemmar}
    \medlem{Kristoffer Gustafsson}
    \medlem{Leo Mirzajanzadeh}
    \medlem{Alva Johansson, lämnade vid punkt 4.4}
    \medlem{Josefin Kokkinakis, lämnade vid punkt 4.5}
    \medlem{Albin Otterhäll}
    \medlem{Tim Persson}
    \medlem{Gustav Dalemo}
    \medlem{Petter Blomkvist}
    \medlem{Emmie Berger}
    \medlem{Alexander Lisborg}
    \medlem{Lukas Gartman, ankom vid punkt 2.2}
\end{närvarande_medlemmar}

\section{Öppnande av möte}

Mötet öppnades av Samuel Hammersberg 17.18

\section{Formalia}

\subsection{Divisionsstämmans beslutbarhet}

6 kap. i stadgan definierar regler Divisionstämman.

Den 8 november kallade styrelsen till divisionsstämma genom att skicka ut mejl, samt skriva i discordservern \emph{MonadenDV};

Dessa möteshandlingar ska ha skickats ut under måndagen den 21 november.

\subsubsection*{Beslut}
Divisionsstämman bekräftar röstlängden.

\begin{attsatser}
	\item divisionsstämman har uppnått kraven i stadgan för att få hålla möte, och är därmed beslutbar.
\end{attsatser}

\subsection{Fastställande av mötesschema}

För att divisionsstämman ska kunna fatta ett beslut eller protokollföra en diskussion behöver punkten i mötesschemat där stämman ska fatta beslut vara inlagd eller föras in i mötesschemat senast vid den här punkten.

\subsubsection*{Beslut}

\begin{attsatser}
	\item inför i mötesschemat att respektive kommitté kör en muntlig verksamhetsrapport
    \item mötesschemat fastställs med första attsatsen
\end{attsatser}


\subsection{Val av mötesordförande}

Mötesordförande har till uppgift att leda Divisionsstämmans sammankomst.
Hen ansvarar för att mötesformalia sköts.

\subsubsection*{Beslut}

\begin{attsatser}
	\item Samuel Hammersberg väljs till mötesordförande.
\end{attsatser}


\subsection{Val av vice mötesordförande}

Vice mötesordförande hjälper mötesordförande med att hålla talarlistan, och att alla får komma till tals.

\subsubsection*{Beslut}

\begin{attsatser}
	\item Morgan Thowsen väljs till vice mötesordförande.
\end{attsatser}

\subsection{Val av mötessekreterare}

Mötessekreteraren har till uppgift att anteckna diskussioner, beslut, och eventuella reservationer under mötet.

\subsubsection*{Förslag till beslut}

\begin{attsatser}
	\item Sebastian Selander väljs till mötessekreterare.
\end{attsatser}

\subsection{Val av protokolljusterare}

Protokolljusterare har till uppgift att kontrollera att protokollet i slutändan reflekterar de faktiska besluten och diskussionerna som fördes under sammanträdet; samt agera rösträknare vid slutna omröstningar.
Utöver protokolljusterarna så ska mötesordförande och mötessekreteraren signera protokollet.
Vid Divisionsstämmans sammanträden ska det vara två justerare.
Mötesordförande och mötessekreteraren kan inte vara justerare.

\subsubsection*{Beslut}

    \begin{attsatser}
        \item Kristoffer Gustafsson och Tim Persson väljs till protokolljusterare.
    \end{attsatser}

\newpage

\section{Rapporter}

\subsection{Styrelsen}

\subsubsection*{Verksamhetsrapport}

Styrelsen har haft fyra möten sedan Divisionens stämma den 17 maj.
Protokollen från mötena finns (snart) tillgängliga för allmänheten på styrelsens drive.
Länk till styrdokument och protokoll: \verb|https://drive.google.com/drive/folders/1o_lLM7g7ph-xdxgut3E_BU0ywichWYTX?usp=sharing|

Efter en lång kamp, så har vi äntligen fått en rullstolsramp till monaden, men kampen är inte över än, då vi fortfarande behöver ett större kök.

Samuel har varit på institutionsråd och programråd. På programrådet så togs saker som introkursen, obligatoriska kurser och möjligheten om ett tillägg av en kurs inom open source licenser. Vi åt väldigt mycket fika.

Mottagningen har även gått bra, och alla kommittéer har gjort ett bra jobb! Verkar även som att de fortsätter att rulla på bra!

\newpage

\subsubsection*{Ekonomi}

Den gångna tiden har varit en ny epok för Datavetenskapsdivisionen i avseende finanser och ekonomi. Mottagningsbudgeten 2022 var i relation till tidigare mottagningsbudgetar, enorm(!). Av vår tilldelade Göta-budget, för mottagningen, om 30tkr har 14tkr spenderats. Av vår tilldelade institutions-budget om 78tkr har 71tkr spenderats. Det totala beloppet för mottagningen i skrivande stund är alltså 85tkr. Totalt för året har vi haft kostnader om 95tkr och inkomster om 92tkr. Med det sagt, så har vi utstående inbetalningar från Göta som gör att summan förväntas landa på plus för kalenderåret.

Det nya bokföringssystemet har varit en lärandeprocess men då vi inte har något bokföringskrav så har vi tämligen lösa regler för vad som "krävs" och vi kan därmed utefter behov skapa konton och resultatgrupper för att göra resultrapporter "berättande" för medlemmar. En struktur för bokföringen är nu, och kommer förbli, en levande mall där tydlighet gentemot medlemmarna är en stor del av anledningen till bokföringssystemet. Jag tror och hoppas att nästa kassör med enkelhet kommer kunna fortsätta utveckla strukturen.

Då detta är min sista divisionsstämma som styrelsemedlem och kassör, så önskar jag att tacka för mig.

mvh, Morgan

(Notera att siffrorna är levande stämmer för 2022-11-20)

Tillägg: utöver mottagningen är ekonomin oförändrad, speciellt då det inte har gått så många skolveckor sedan förra stämman.

\newpage

\subsection{Verksamhetsrapporter av kommittéer}

\subsubsection*{DVRK}
Muntlig verksamhetsrapport av DVRKs ordförande Leo Mirzajanzadeh.

\subsubsection*{ConCats}
Muntlig verksamhetsrapport av Tekla Siesjö.

\subsubsection*{Femme++}
Muntlig verksamhetsrapport av Femme++s ordförande Emmie Berger.

\subsubsection*{DVArm}
Muntlig verksamhetsrapport av DVArms ordförande Albin Otterhäll.

\subsubsection*{Mega6}
Muntlig verksamhetsrapport av Petter Blomkvist.

\newpage

\section{Beslutsärenden}

Enligt Stadgan måste ändringar av Stadgan röstas igenom på två av Divisionsstämmans varandra följande möten.
Om en beslutpunkt innehåller ``första läsningen' innebär det att det är första gången beslutet tas upp för omröstning.
Om en beslutspunkt innehåller ``andra läsningen'' innebär det att beslutspunkten har röstats igenom förra stämmomötet, och beslutet behöver bekräftas för att gå igenom.

\subsection{Proposition: ny logotyp}

Under senaste divisionsstämman beslutade vi om vilken logotyp vi i divisionen vill ha.
Det enda som saknas i dagsläget är att vi beslutar om en färgsättning på loggan.

Syftet med att faktiskt införa en ny logga diskuteras och förtydligas.
Användandet av vår nuvarande inofficiella Mega Man-logga diskuteras, varför kan vi inte fortsätta använda den?

    \subsubsection*{Yrkande till beslut}

        \begin{attsatser}

            \item arbetet med att ta fram en officiell logga avslutas, eller;
            \item yrka för att upprätta en arbetsgrupp för att ta fram en officiell logotyp som beslutas om på nästa stämma

        \end{attsatser}

    \subsubsection*{Beslut}

    \begin{attsatser}
        \item andra attsatsen bifalles
    \end{attsatser}

\newpage

    \subsubsection*{Ytterligare yrkanden till beslut}
        \begin{attsatser}
                \item divisionen inte representeras officiellt av någon logga.
                \item den arbetsgruppen har i uppdrag att ta fram minst 5 (fem)
                      alternativ av en officiell logga som presenteras till nästa stämma,
                      där minst ett av alternativen är en silouettelogga; och minst ett av alternativen är en textlogga.
        \end{attsatser}

    \subsubsection*{Beslut}
    \begin{attsatser}
    \item andra attsatsen bifalles
    \end{attsatser}

\subsubsection*{Val av ansvarig för arbetsgruppen}

    \subsubsection*{Nomineringar}

        \begin{itemize}
            \item Tim Persson
        \end{itemize}

        \subsubsection*{Utjustering}
        Den nominerade justeras ut

        \subsubsection*{Beslut}
            \begin{attsatser}
                \item Tim Persson väljs till ansvarig för arbetsgruppen
            \end{attsatser}

        \subsubsection*{Injustering}
        Den nominerade justeras in

\newpage

\subsection{Proposition: Inval av ordförande till UPPDRAG verksamhetsåret 2023}
Under divisionsstämman i maj startade vi upp kommittén UPPDRAG, vilket är en studienämndskommitté.
Det hade varit grymt roligt om någon var intresserad av rollen som ordförande i kommittén.

    \subsubsection*{Nomineringar}

        \begin{itemize}
            \item Josefin Kokkinakis nominerar sig själv
        \end{itemize}

        Josefin presenterar syftet med studienämnden, och lite hur det kommer gå till för att förbättra kurser.

        \subsubsection*{Utjustering}
        Den nominerade justeras ut

        \subsubsection*{Beslut}
            \begin{attsatser}
                \item Josefin Kokkinakis väljs till ordförande för UPPDRAG.
            \end{attsatser}

        \subsubsection*{Injustering}
        Den nominerade justeras in

\subsection{Motion: Riva vägg}

Idag har Monaden två rum där studenter kan umgås och studera; samt ett kök.
Fördelarna med att ha två rum istället för ett stort rum är att om det är mycket ljud i det ena rummet kan man gå in i det andra för att få det lugnare.
Nackdelarna med att ha två rum istället för ett stort rum är att när man har studiesociala aktiviteter kan det bli väldigt trångt när man är begränsat till ett rum.

Om man istället lyckades riva väggen mellan de två rummen och istället installerade en permanent skärmvägg i dess ställe får man det bästa av två världar.
Dels har man vanligtvis två olika rum där man kan välja att byta rum om det är mycket ljud i det ena rummet, men man kan få ett stort rum när man väl behöver det.
Den här lösningen har man idag i fysikprogrammens sektionslokaler; GUs fysikprogram och Chalmers fysik- och matteprogram delar på en tredelad lokal där programmen har de två yttre delarna, och sedan delar de på den mellersta delen.
Varje del är avskild med varsin skärmvägg för att kunna avskilja de yttre delarna av rummet när det är stök i mittendelen.

\emph{Albin "Slaget" Otterhäll, DV'18} \\
\emph{Kristoffer "KG" Gustafsson DV'20} \\
\emph{Leo "Leo" Mirzajanzadeh DV'21}

Anledning till att riva vägg diskuteras. Eventuella problem med en rivning diskuteras också.


    \subsubsection*{Förslag till beslut}
        \begin{attsatser}
            \item styrelsen får i uppdrag att driva frågan om att väggen mellan det stora och det lilla rummet rivs och att en permanent skärmvägg installeras i dess ställe.
        \end{attsatser}

    \subsubsection*{Beslut}
        \begin{attsatser}
            \item styrelsen får i uppdrag att driva frågan om att väggen mellan det stora och det lilla rummet rivs och att en permanent skärmvägg installeras i dess ställe.
        \end{attsatser}

\subsection{Motion: En historielös utbildning}

Idag finns det ingen kurs vid institutionen som behandlar datavetenskapens och datorernas historia nämnvärt, och det tycker författaren av den här motionen är mycket tråkigt.
Hur många vet
    \begin{itemize}
        \item att den första dokumenterade datorn byggdes av antikens greker för runt 2 000 år sedan;
        \item vad Charles Babbage och Ada Lovlace tänkte och byggde;
        \item vad Alan Turing \& Co. byggde i Bletchley Park; eller
        \item alla de olika datorerna som byggdes under efterkrigstiden fram till och med 90-talet som definierar nästan alla moderna datorer idag?
    \end{itemize}

"Allt gammalt blir nytt igen" stämmer väldigt bra in på vårt fält, men med nackdelen att när utvecklare återuppfinner hjulet igen så görs det utan kunskaper om vad som har försökts tidigare.

Därför tycker jag att divisionen ska arbeta för universitetet ska ta fram en kurs ägnad för att studera vårt fälts historia för att studenter ska bli medvetna om vad som har kommit tidigare.
Tidigare har matematikprogrammet haft en liknande kurs, men det verkar som att det inte längre finns.

\emph{Albin "Slaget" Otterhäll, DV'18} \\

    \subsubsection*{Förslag till beslut}
        \begin{attsatser}
            \item divisionens representanter i programrådet driver frågan att en eller flera kurser explicit behandlar datavetenskapens och datorernas historia.
        \end{attsatser}

    \subsubsection*{Beslut}
        \begin{attsatser}
            \item attsatsen bifalles
        \end{attsatser}

\newpage

\subsection{Motion: Visa vårt stöda till Ukraina}

Jag tror att vi alla sympatiserar med Ukraina i deras kamp mot Ryssland, och känner för att på bästa möjliga sätt säga "fuck you" till Putin.

Och det finns inget som säger "fuck you" lika mycket som att köra runt med ett gigantiskt hangarfartyg, så jag tycker vi skaffar ett till Ukraina.
Liksom, hur svårt kan det vara?

Jag förstår att ni är lite skeptiska till att köpa ett hangarfartyg till Ukraina.
En hangarfartyg eskader (ett hangarfartyg, ubåtar, flygplan, helikoptrar, och massa annat bös) är ju det minsta varje land med självrespekt behöva ha.
Jag förstår det, instämmer till och med!
Men man måste börja smått.

Man kanske är skeptisk till hur man ska kunna leverera hangarfartyget till Ukraina, då Turkiet har stängt Bosporen och Dardanellerna sundet.
Jag håller med om att det är en tuff nöt att knäcka, kanske till och med den svåraste.
Men jag tror att vi kan lösa detta ganska smidigt genom att beställa flygfrakt i samband med att vi lägger beställningen på hangarfartyget hos Kockum.
De kanske till och med bjuder på frakten!?

Då ConCats är vårt rustmästeri bör de hålla koll på hur man bygger och beställer saker och ting, så jag tycker att det är rimligast att de får lägga beställningen.

\emph{Albin "Slaget" Otterhäll, DV'18}

    \subsubsection*{Förslag till beslut}
        \begin{attsatser}
            \item ConCats får i uppdrag till nästa stämma begära en offert från SAAB Kockums på ett "Bautastort hangarfartyg med 69 stycken kanoner, 420 stycken kulsprutor, 1337 stycken sjömålsrobotar, och 9001 stycken enheter bärs (för man kan ju inte åka båt allt för länge utan att dricka lite bärs).
            \item vid nästa stämma presentera offerten för att möjliggöra beslut om införskaffning.
        \end{attsatser}

    \subsubsection*{Beslut}
        \begin{attsatser}
            \item båda attsatser nekas
        \end{attsatser}

\newpage

\subsection{Motion: Inval av ordförande till ConCats verksamhetsåret 2023}

    \subsubsection*{Förslag till beslut}

        \begin{attsatser}
            \item Alexander Lisborg väljs till ordförande för ConCats
        \end{attsatser}

        \subsubsection*{Utjustering}
        Den nominerade justeras ut

        \subsubsection*{Beslut}
            \begin{attsatser}
                \item Alexander Lisborg väljs till ordförande för ConCats
            \end{attsatser}

        \subsubsection*{Injustering}
        Den nominerade justeras in

\subsection{Val of revisorer för verksamhetsåret 2023}
Vi behöver revisorer för 2023 så att stämman har koll på att styrelsen sköter sitt jobb!

        \subsubsection*{Förslag från styrelsen}
        Under tidigare stämmor har vi valt att vakantsätta denna posten, så vi i styrelsen tänker
        att vi gör det även denna gången.

            \begin{attsatser}
                \item posten vakantsätts även denna divisionsstämman
            \end{attsatser}

        \subsubsection*{Beslut}

        \begin{attsatser}
        \item punkten bordläggs till nästa divisionsstämma
        \end{attsatser}

\subsection{Val av divisionsordförande}
Den nuvarande styrelsens mandatperiod tar slut 31 december, vi behöver därmed ännu en gång rösta in
en divisionsordförande.

Då Samuel Hammersberg idag är ordförande för divisionen och kan tänka sig fortsätta med denna rollen
för 2023 tycker vi i styrelsen han är passande!

\subsubsection*{Förslag till beslut}
    \begin{attsatser}
    \item Divisionsstämman väljer Samuel Hammersberg (DV'20) till divisionsordförande för verksamhetsåret 2023
    \end{attsatser}

        \subsubsection*{Utjustering}
        Den nominerade justeras ut

        \subsubsection*{Beslut}
            \begin{attsatser}
                \item Samuel Hammersberg väljs till divisionsordförande för verksamhetsåret 2023.
            \end{attsatser}

        \subsubsection*{Injustering}
        Den nominerade justeras in

\newpage

\subsection{Inval till styrelsen verksamhetsår 2023}
Nytt år innebär att vi återigen måste göra nya val av styrelsemedlemmar.
Morgan Thowsen och Sebastian Selander kommer välja att inte ställa upp för året 2023 tyvärr.

    \subsubsection*{Förslag till beslut}
        \begin{attsatser}
            \item Tekla Siesjö (DV'20); Lukas Gartman (DV'20) väljs in till styrelsen för verksamhetsåret 2023
        \end{attsatser}

        \subsubsection*{Utjustering}
        De nominerade justeras ut

        \subsubsection*{Beslut}
            \begin{attsatser}
                \item Tekla Siesjö (DV'20); Lukas Gartman (DV'20) väljs in till styrelsen för verksamhetsåret 2023
            \end{attsatser}

        \subsubsection*{Injustering}
        De nominerade justeras in

Vi har sparat denna beslutspunkt till sist i beslutssektionen då vi tror att den kan leda till
en del diskussion. Är ni intresserade av att gå med i styrelsen och har frågor är det ett toppläge
att ställa dessa.

\newpage

\section{Diskussioner}\label{sec:discussioner}

    \subsection*{Samlad plats för divisionens information}
    Aktiva medlemmar i discord har diskuterat om att starta upp en form av wikisida för divisionen.
    Detta är för att samla all kunskap, information, och styrdokument på ett och samma ställe.
    För- och nackdelar diskuteras kring de olika alternativen.

\section{Avslutande av möte}

Mötet avslutades klockan 20.10

\stämmosignaturer

\end{document}
