\documentclass[protokoll]{dvd}

\KOMAoptions{
	headwidth = 18cm,
	footwidth = 18cm,
}

\begin{document}

\title{Divisionsstämmans möte 1}
\subtitle{2022}
\author{Divisionsstämman}
\date{2022-05-17}

\textbf{Datum:} \csname @date\endcsname\\
\textbf{Tid:} 17:17\\
\textbf{Plats:} Monaden\\
\textbf{Styrelsemedlemmar:}
\begin{närvarande_förtroendevalda}
	\förtroendevald{Ordförande}{Samuel Hammersberg}{Ja}
	\förtroendevald{Vice ordförande}{\emph{Vakant}}{}
	\förtroendevald{SAMO}{Tekla Siesjö}{Ja}
	\förtroendevald{Kassör}{Morgan Thowsen}{Ja}
	\förtroendevald{Sekreterare}{Sebastian Selander}{Ja}
\end{närvarande_förtroendevalda}

\textbf{Närvarande övriga medlemmar:}
\begin{närvarande_medlemmar}
    \medlem{Albin Otterhäll}
    \medlem{Petter Blomqvist}
    \medlem{Leo Mirzajanzadeh}
    \medlem{Emmie Berger}
    \medlem{Miranda Jernberg}
    \medlem{William Bodin}
    \medlem{Kristoffer Gustafsson}
    \medlem{Anthon Wirback}
    \medlem{Jeffrey Wolff}
    \medlem{Alexander Lisborg (anlände efter punkt 4.1)}
\end{närvarande_medlemmar}

\newpage

\section{Öppnande av möte}

Mötet öppnades av Samuel Hammersberg 17.32

\section{Formalia}


\subsection{Divisionsstämmans beslutbarhet}

6 kap. i stadgan definierar regler Divisionstämman.

Den 3 maj kallade styrelsen till divisionsstämma genom att skicka ut mejl via \verb|utskick@dvet.se|; skriva i discordservern \emph{MonadenDV}; samt satte upp en lapp i Monaden.

Dessa möteshandlingar ska ha skickats ut under fredagen den 13 maj.

\subsubsection*{Förslag till beslut}

\begin{attsatser}
	\item divisionsstämman har uppnått kraven i stadgan för att få hålla möte, och är därmed beslutbar.
\end{attsatser}

\subsubsection*{Beslut}

\begin{attsatser}
    \item attsatsen bifalles
\end{attsatser}

\subsection{Fastställande av mötesschema}

För att divisionsstämman ska kunna fatta ett beslut eller protokollföra en diskussion behöver punkten i mötesschemat där stämman ska fatta beslut vara inlagd eller föras in i mötesschemat senast vid den här punkten.

Inga interpellationer, eller enklare frågor inkom till styrelsen.


\subsubsection*{Förslag till beslut}

\begin{attsatser}
	\item Mötesschemat fastställs utan några förändringar.
\end{attsatser}

\subsubsection*{Ändringsyrkanden}

\begin{attsatser}
    \item fastställande av röstlängden införs direkt efter nuvarande punkt \emph{Fastställande av mötesschema} %bifalles
    \item under punkt 3 införa att Mega6 och ConCats ordförande avlägger muntliga verksamhetsrapporter %bifalles
    \item under punkt 4 införa punkten \emph{Val av revisorer} %bifalles
\end{attsatser}

\subsubsection*{Beslut}

\begin{attsatser}
    \item samtliga ändringsyrkanden bifalles
    \item attsatsen bifalles efter ändringsyrkanden bifallits
\end{attsatser}

\subsection{Val av mötesordförande}

Mötesordförande har till uppgift att leda Divisionsstämmans sammankomst.
Hen ansvarar för att mötesformalia sköts.

\subsubsection*{Förslag till beslut}

\begin{attsatser}
	\item Samuel Hammersberg väljs till mötesordförande.
\end{attsatser}

\subsubsection*{Beslut}

\begin{attsatser}
    \item attsatsen bifalles
\end{attsatser}

\subsection{Val av vice mötesordförande}

Vice mötesordförande hjälper mötesordförande med att hålla talarlistan, och att alla får komma till tals.

\subsubsection*{Förslag till beslut}

\begin{attsatser}
	\item \emph{Inga förslag från styrelsen innan mötet}
\end{attsatser}

\subsubsection*{Beslut}

\begin{attsatser}
    \item Tekla Siesjö väljs till vice mötesordförande
\end{attsatser}

\subsection{Val av mötessekreterare}

Mötessekreteraren har till uppgift att anteckna diskussioner, beslut, och eventuella reservationer under mötet.

\subsubsection*{Förslag till beslut}

\begin{attsatser}
	\item Sebastian Selander väljs till mötessekreterare.
\end{attsatser}

\subsubsection*{Beslut}

\begin{attsatser}
    \item attsatsen bifalles
\end{attsatser}


\subsection{Val av protokolljusterare}

Protokolljusterare har till uppgift att kontrollera att protokollet i slutändan reflekterar de faktiska besluten och diskussionerna som fördes under sammanträdet; samt agera rösträknare vid slutna omröstningar.
Utöver protokolljusterarna så ska mötesordförande och mötessekreteraren signera protokollet.
Vid Divisionsstämmans sammanträden ska det vara två justerare.
Mötesordförande och mötessekreteraren kan inte vara justerare.

\subsubsection*{Förslag till beslut}

\emph{Inga förslag från styrelsen innan mötet}

\subsubsection*{Diskussion}
Emmie Berger och Kristoffer Gustaffson erbjuder sig vara protokolljusterare

\subsubsection*{Beslut}

\begin{attsatser}
    \item Emmie Berger väljs till protokolljusterare
    \item Kristoffer Gustafsson väljs till protokolljusterare
\end{attsatser}

\newpage

\section{Rapporter}

\subsection{Styrelsen}

\subsubsection*{Verksamhetsrapport}

Styrelsen har haft åtta möten sedan Divisionens stämma den 17 november.
Protokollen från mötena finns (snart) tillgängliga för allmänheten på styrelsens drive.
Länk till styrdokument och protokoll: \verb|https://drive.google.com/drive/folders/1o_lLM7g7ph-xdxgut3E_BU0ywichWYTX?usp=sharing|

Samuel har varit på instituionsråd.
Morgan som kassör har fixat bankkonto och ett bokföringssystem.

Styrelsen och DVRK har haft ett möte med Alex (PA), Jonathan (SYV) och Lisa (UA) den 22 april
Under detta mötet diskuterades mottagningen, studienämnd och budget.
Alla valbara kurser under första året kommer tas bort, och upplägget kommer vara: 

LP1: (DIT980) Diskret matematik för Datavetare, (DIT440) Introduktion till Funktionell programmering 

LP2: (DIT791) Grundläggande Datorteknik, (DIT012) Imperativ programmering med grundläggande objektorientering

LP3: (TMV216 / MMGD20) Linjär Algebra, (DIT953) Objektorienterad Programmering och Design

LP4: (TMV170 / MMGD30) Matematisk Analys, (DIT961) Datastrukterer \\

Kommittéerna har vaknat till liv ordentligt och hållit i många roliga aktiviteter!

\subsubsection*{Ekonomi}

Datavetenskapsdivsionens omsättning har ökat markant. Det är ett nollsummespel i dagens läge vilket vi givetvis hoppas ändra på då vi står dåligt till ifall vi får oförutsedda utgifter. Med det sagt så är inte bara jag, utan hela styrelsen(!), glada över den ökade aktiviteten som omsättningen fört med sig. Trots anledning till fortsatt försiktighet vid sjukdom så är Covid19 inte längre en våt filt över föreningens liv.

Några saker som har ändrats sedan föregående divisionsstämma är att vi nu börjat med bokföring samt införskaffat fakturasystem. Vi har varit flitiga under året och har från Göta äskat ca 6tkr i skrivande stund. Med det sagt så är Götas kassa, gentemot oss, stängd för terminen då resterande skall gå till examinationsfesten.


Vi har under årets första månader, fram till vårens divisionsstämma utgifter om ca 7.2tkr. Det skall anmärkas att bokförda beloppet är något lägre i skrivande stund men vi har stående äskningar som blivit beviljade. Pengarna har bland annat gått till dekoration av Monaden, kläder till Mega6, mat och snacks mm.

Vi har också haft vårt första supportevent men pengarna är i överrenskommelse med Dvarm öronmärkta Dvarm tills det att kommittéen har en tillfredställande finansiell grund.

Kassan står i skrivande stund i 4.4tkr men den verkliga summan är högre då vi har blivit beviljade att retroaktivt äskan från institutionen för de utgifter vi haft under året som inte täckts av Göta Studentkår. Kort och gott, vi har haft fler evenemang och utgifter än på många år (kanske någonsin) samt bättre ekonomi är någonsin. Bra jobbat till alla!


\subsubsection*{DVArm}
Sedan skapandet av DVarm vid förra stämmomötet kommer DVarm ha genomfört ett samarbete med en sammarbetsparner.
Lite arbete görs förnärvarande för att inleda kontakter med arbetsgivare. Arbetet förväntas dra igång mera när den grafiska profilen är klar, och man kan börja marknadsföra kommittén med grafiskt material.

Muntlig verksamhetsrapport gås igenom av ordförande för DVArm

\emph{Albin "Slaget" Otterhäll \\ Ordförande, \\ DVarm DV'18}

\subsubsection*{ConCats}
Muntlig verksamhetsrapport presenteradese av ordförande för ConCats

\emph{Miranda Jernberg \\ Ordförande, \\ DVarm DV'18}

\subsubsection*{Mega6}
Muntlig verksamhetsrapport presenteradese av ordförande för Mega6

\emph{Anthon Wirback \\ Ordförande, \\ DVarm DV'18}

\subsection{Fastställande av röstlängden}

Alla medlemmar som deltar på stämman bekräftas vara medlemmar i divisionen, detta är för att säkerställa att de som faktiskt är här har rösträtt.
Inga konstigheter dök upp, alla är medlemmar i divisionen.


\newpage

\section{Beslutsärenden}

Enligt Stadgan måste ändringar av Stadgan röstas igenom på två av Divisionsstämmans varandra följande möten.
Om en beslutpunkt innehåller ``första läsningen' innebär det att det är första gången beslutet tas upp för omröstning.
Om en beslutspunkt innehåller ``andra läsningen'' innebär det att beslutspunkten har röstats igenom förra stämmomötet, och beslutet behöver bekräftas för att gå igenom.


\subsection{Proposition: ny logotyp}

Under senaste divisionstämman beslutade vi att en arbetsgrupp skulle skapas, vars syfte var att plocka fram minst ett förslag till denna divisionsstämma.
Inför förra mötet hade vi även fått in förslag, dessa sitter i hallen på whiteboarden. 
Ni hittar även dessa här: \verb|https://drive.google.com/drive/folders/1OLFXEvzHtClHOYIqKFPtRCFaB3OPxPBO?usp=sharing|

Det vi i styrelsen nu tänker är att vi faktiskt röstar på antingen ett förslag från arbetsgruppen eller ett förslag vi har fått in tidigare.
Det finns dock möjligheten att bordlägga denna punkt till nästa möte ännu en gång.

\subsubsection*{Förslag till beslut}

\emph{Inga konkreta förslag från styrelsen innan möte}

\subsubsection*{Diskussion}

En hederlig utslagsturnering med alla inkomna förslag utförs, en logga ryker åt gången tills vi har en vinnare. 
Alla medlemmar får rösta anonymt.

Röstningen avslutades med en riktigt nailbiter mellan DV i Haskell-liknande format och lambda i Haskell-liknande format.
DV i Haskell-liknande format vann med 10 röster mot 3 röster.

Önskemål om att skicka den till grafiker och få input lyfts, samt att det är viktigt att lambdan är en del av loggan
Färgerna av loggan nämns även som lite otydlig, andra förslag av färger ges som exempel. Stämman påpekar att vi kan utnyttja en eventuell designer för färgschema.
Möjligheten att använda ett stort d istället för ett litet diskuteras även. Megamans färgschema önskas behållas.
Att fjompen färgläggs med något annorlunda än samma som dv kan också vara en möjlighet för att förstärka att det faktiskt står dv.
Angående cirklarna som finns på den yttre delen var tanken att de skulle se ut som funktionskompisitions-\'operatorn\'.
Möjligheten att plocka bort dem helt, fylla ut dem eller kanske byta ut mot \'::\'.

\subsubsection*{Förslag till beslut från stämman}

\begin{attsatser}
    \item vi väljer stora bokstäver och triangel; eller
    \item vi väljer stora bokstäver och fjomp; eller
    \item vi väljer små bokstäver och triangel; eller
    \item vi väljer små bokstäver och fjomp
    \item vi lägger till en rektangulär ram runt texten DV; eller
    \item vi lägger inte till en rektangulär ram runt texten DV
    \item byta ut DV-texten mot Megaman i loggan som målas upp på backen på Chalmers; eller
    \item inte byta ut DV-texten Megaman i loggan som målas upp på backen och istället använder den logga vi idag röstat igenom
    \item Jeffrey Wolf tar loggan till en grafiker och sköter detta arbete till nästa stämma
\end{attsatser}

\subsubsection*{Beslut}

\begin{attsatser}
    \item första attsatsen bifalles
    \item sjätte attsatsen bifalles
    \item sjunde attsatsen bifalles
    \item nionde attsatsen bifalles
\end{attsatser}

\subsection{Proposition: divisionsmedlemsskap för kommittémedlemmar}
Idag behöver man inte vara medlem i divisionen för att få vara med i en kommitté. Detta är på grund att kommittéer inte har någon formell lista över vem som är medlem.
Vårt förslag är att kommittéerna inför medlemslistor och därmed kan sätta divisionsmedlemsskap som krav.
Detta skulle dock innebära att kommitténs ordförande behöver föra en medlemslista och verifiera via styrelsen att en kommittémedlem är en divisionsmedlem.

\subsubsection*{Diskussion}
Det diskuteras för- och nackdelar om vi bör införa detta.
Vad det innebär att representera eller endast associeras med en kommitté förtydligas och diskuteras.

    \subsubsection*{Förslag till beslut}
    \begin{attsatser}
        \item varje kommitté för en officiell medlemslista över alla medlemmar; och
        \item varje kommitté kräver att man som kommittémedlem även är medlem i datavetenskapsdivisionen
    \end{attsatser}

    \subsubsection*{Ändringsyrkanden}
    \begin{attsatser}
        \item första attsatsen ändras till
        \begin{displayquote}
            Att det införs ett nytt kapitel i reglementet mellan nuvarande kapitel 3 och 4 som lyder

            \subsection*{Regler för alla kommittéer}
            \begin{enumerate}[label=\arabic* §]
                \item Varje kommitté ska föra ett medlemsregister över kommitténs medlemmar
            \end{enumerate}
        \end{displayquote}

    \item propositionens andra attsats stryks
    \end{attsatser}

    \subsubsection*{Jämkan}
    Styrelsen beslutar att ändringsyrkanden är bättre formulerade och ersätter de föreslagna attsatserna.

    \subsubsection*{Beslut}
    \begin{attsatser}
        \item samtliga jämkade attsatser bifalles
    \end{attsatser}

\subsection{Proposition: skapa studienämndskommitté}
Vi har fått förfrågan från både Alex och Jonathan om att skapa en studienämndskommitté.
Under senaste läsåret har några studenter från 2021 fört en form av studienämnd inofficiellt och det verkar ha varit givande både för lärare och studenter.
En studienämnd skulle ha i uppdrag att sköta kontakt mellan studenter och lärare, ungefär som kursrepresentanter fast stående för hela året.
Detta skulle endast gälla för de kurser som går första året.

Vi vill påpeka att nedanstående punkter endast är förslag från styrelsen och får gärna diskuteras under stämman, exempelvis namnet på kommittén

Styrelsen förtydligar vad syftet med en studienämndskommitté är, samt hur DNS (datatekniks studienämndskommitté) sköter sitt arbete.

    \subsubsection*{Förslag från styrelsen}

        \begin{attsatser}
        \item Att i avsnittet med kommittéer i reglementet, eller som självständigt dokument i dokumentsamlingen, införa följande
        \begin{displayquote}
            \subsection*{Studienämndskommittén}
            \begin{enumerate}[label=\arabic* §]
                \item Kommitténs namn är Studienämndskommittén

                \item Studienämndskommittén har till uppgift att arbeta för en mer rättvis och tryggare relation mellan kurser och studenter

                \item Mandatperioden för kommittémedlemmarna bestäms internt inom kommittén
            \end{enumerate}
        \end{displayquote}
    \end{attsatser}

    \subsubsection*{Ändringsyrkanden}
    \begin{attsatser}
        \item den första paragrafen på de föreslagna reglerna för \emph{Studienämndskommittén} ändras till
            \begin{displayquote}
                Kommitténs namn är UPPDRAG, som står för \emph{UPPDRAG Påminner och Påpekar Datavetare Runt Alla daGar}
            \end{displayquote}
        \item övriga paragrafer i de föreslagna reglerna för \emph{Studienämndskommittén} korrigeras till att använda det nya namnförslaget.
    \end{attsatser}

    \subsubsection*{Beslut}
    \begin{attsatser}
        \item samtliga ändringsyrkanden bifalles
        \item de föreslagna attsatserna som inte ändras i \emph{ändringsyrkanden} bifalles
    \end{attsatser}


\subsection{Inval av ordförande till UPPDRAG verksamhetsåret 2022}

    \subsubsection*{Nomineringar}

            \emph{inga nomineringar under stämman}

        \subsubsection*{Beslut}
            \begin{attsatser}
                \item posten vakantsätts
            \end{attsatser}

\newpage

\subsection{Motion: Skapa kommitté för kvinnor, icke-binära och transpersoner}
    Det är viktigt att ha en plats där de som i nuläget tillhör dessa minoritetsgrupper på DV kan uttrycka sin åsikt och prata om saker som de annars känner är svårt att prata om.
    Det är flera personer som uttryckt att anledningen till att de inte tar del av studie-livet på skolan är för de har svårt att hitta sin plats och uttrycka sig när de befinner sig i en minoritet.
    Genom lite mindre sociala evenemang samt aktiviteter som workshops och självförsvar så hoppas vi stärka dessa personers kraft att uttrycka sig även i andra sociala sammanhang eller bara känna att de har någon plats att vara med på, i divisionen.

\emph{Emmie Berger \\ DV'20}

\emph{Tekla Siesjö \\ DV'20}

\emph{Miranda Jernberg \\ DV'19}


\subsubsection*{Förslag från motionären}

    \begin{attsatser}
        \item Att i avsnittet med kommittéer i reglementet, eller som självständigt dokument i dokumentsamlingen, införa följande

        \begin{displayquote}
            \subsection*{Femme++}
            \begin{enumerate}[label=\arabic* §]
                \item Kommitténs namn är Femme++

                \item Femme++ har till uppgift att stödja dessa minoriteter och skapa en trygg miljö att uttrycka åsikter

                \item Mandatperioden för kommittémedlemmarna bestäms internt inom kommittén.

                \item Kommittémedlemmar bestäms internt i kommittén 
            \end{enumerate}
        \end{displayquote}
    \end{attsatser}

    \subsubsection*{Styrelsens svar på motion}
    Självklart vill även vi i styrelsen skapa en trygg miljö för minoriteter så vi tycker detta är en jättebra idé.

    \subsubsection*{Förslag från styrelsen}
        \begin{attsatser}
            \item motionens samtliga attsatser bifalles
        \end{attsatser}

        \subsubsection*{Ändringsyrkanden}
        \begin{attsatser}
            \item den fjärde paragrafen på de föreslagna reglerna för \emph{Femme++} tas bort
        \end{attsatser}

        \subsubsection*{Jämkan}
        Motionärerna väljer att jämka sina attsatser vilket innebär att ändringsyrkanden väljs istället.

        \subsubsection*{Beslut}
        \begin{attsatser}
            \item samtliga attsatser bifalles
        \end{attsatser}

\subsection{Inval av ordförande till Femme++ verksamhetsåret 2022}
Då vi nu valt att starta Femme++ behövs även en ordförande för denna kommitté

Motionärerna förtydligar syftet med denna kommitté, vad deras plan är, samt svarar på ett par frågor som dyker upp från resterande divisionsmedlemmar.

    \subsubsection*{Nomineringar}
    Emmie Berger nominerar sig själv till ordförande för kommittén

    \subsubsection*{Utjustering}
    Den nominerade justeras ut

    \subsubsection*{Beslut}
        \begin{attsatser}
            \item Emmie Berger väljs till ordförande för Femme++
        \end{attsatser}

    \subsubsection*{Injustering}
    Den nominerade justeras in

\subsection{Inval av ordförande till DVRK verksamhetsåret 2022}
Då DVRKs förra ordförande valt att avgå behöver vi välja in en ny ordförande till DVRK.

    \subsubsection*{Förslag från styrelsen}
    Leo Mirzajanzadeh har visat intresse när både styrelsen och aktivt DVRK-folk pratat med honom och vi tycker därmed att han är en passande nominering.

    \subsubsection*{Utjustering}
    Den nominerade justeras ut

    \subsubsection*{Diskussion}
    Albin Otterhäll är skeptisk till den nominerades favoritfärg

        \begin{attsatser}
            \item Leo Mirzajanzadeh väljs till ordförande för DVRK
        \end{attsatser}

    \subsubsection*{Injustering}
    Den nominerade justeras in

\subsection{Val av revisorer för verksamhetsåret 2022}
Då denna punkt bordlades under senaste divisionstämma behöver vi lyfta den även detta möte.

\subsubsection*{Förslag till beslut från divisionsstämman}

\begin{attsatser}
    \item vi vakantsätter posterna
\end{attsatser}

\subsubsection*{Beslut}
\begin{attsatser}
    \item attsatsen bifalles
\end{attsatser}

% \section{Diskussioner}\label{sec:discussioner}

\section{Avslutande av möte}

Mötet avslutades kl. 20.30

\stämmosignaturer

\end{document}
