\documentclass[protokoll]{dvd}

\KOMAoptions{
	headwidth = 18cm,
	footwidth = 18cm,
}

\begin{document}

\title{Divisionsstämmans möte 2}
\subtitle{2021}
\author{Divisionsstämman}
\date{2021-09-15}

\textbf{Datum:} \csname @date\endcsname\\
\textbf{Tid:} 18:00\\
\textbf{Plats:} Monaden (eventuellt EA-salen om det är många personer som deltar i mötet)
% \textbf{Styrelsemedlemmar:}
% \begin{närvarande}
% 	\medlem[Ordförande]{Albin Otterhäll}{?}
% 	\medlem[Kassör]{Morgan Thowsen}{?}
% \end{närvarande}
% \textbf{Övriga medlemmar:}
% \begin{närvarande}
%     \medlem
% \end{närvarande}

\section{Öppnande av möte}

Mötet kommer öppnas av Albin Otterhäll vid 18-tiden.

\newpage
\section{Formalia}

\subsection{Läsning av dokument}

Stadgan och styrdokumentet \emph{Regler för Divisionsstämman} reglerar hur Divisionsstämmans möten får kallas och hur de genomförs.
För att mötet ska kunna genomföras korrekt bör samtliga mötesdeltagare ha läst igenom dessa styrdokument.
Därför inleder vi mötet med att samtliga mötesdelgare läser dokumenten.
Detta dokument innehåller alla punkter som kommer diskuteras under detta möte.

\subsubsection*{Förslag till beslut}

\begin{attsatser}
	\item Divisionsstämmans samtliga mötesdeltagare har läst och förstått Stadgan.
	\item Divisionsstämmans samtliga mötesdeltagare har läst och förstått dokumentet \emph{Regler för Divisionsstämman}.
	\item Divisionsstämmans samtliga mötesdeltagare har läst och förstått dessa möteshandlingar.
\end{attsatser}

\subsection{Divisionsstämmans beslutbarhet}

5 kap. i Stadgan och 1 kap. 1--5 §§ i styrdokumentet \emph{Regler för Divisionsstämman} definierar regler Divisionstämman.

Den 26 augusti kallade styrelsen till extrastämma genom att skicka ut mejl via \verb|utskick@dvet.se|; lägga upp en notis på \url{https://dvet.se}; samt satte upp en lapp i Monaden.
Man skapade även ett event på Facebook och länkade eventet i Facebookgruppen \emph{Monaden (GU Datavetenskap)}.
Det lades upp även notiser på Discordservrarna \emph{MonadenDV} och \emph{DV2021}.

Dessa möteshandlingar ska ha skickats ut under måndagen den 13 september.

\subsubsection*{Förslag till beslut}

\begin{attsatser}
	\item Divisionsstämman har uppnått kraven i Stadgan och i \emph{Regler för Divisionsstämman} för att få hålla möte, och är därmed beslutbar.
\end{attsatser}

\subsection{Fastställande av mötesschema}

För att Divisionsstämman ska kunna fatta ett beslut eller protokollföra en diskussion behöver punkten i mötesschemat där stämman ska fatta beslut vara inlagd eller föras in i mötesschemat senast vid den här punkten.

Inga motioner, interpellationer, eller enklare frågor inkom till styrelsen.

Då Morgan Thowsen, divisionens kassör, inte ännu har fått tillgång till divisionens bankkonto har vi tyvärr inte haft möjlighet att framställa en ekonomisk rapport.

\subsubsection*{Förslag till beslut}

\begin{attsatser}
	\item Mötesschemat fastställs utan några förändringar.
\end{attsatser}

\subsection{Val av mötesordförande}

Mötesordförande har till uppgift att leda Divisionsstämmans sammankomst.
Hen ansvarar för att mötesformalia sköts.

\subsubsection*{Förslag till beslut}

\begin{attsatser}
	\item Albin Otterhäll väljs till mötesordförande.
\end{attsatser}

\subsection{Val av vice mötesordförande}

Vice mötesordförande hjälper mötesordförande med att hålla talarlistan, och att alla får komma till tals.

\subsubsection*{Förslag till beslut}

\begin{attsatser}
	\item Morgan Thowsen väljs till vice mötesordförande.
\end{attsatser}

\subsection{Val av mötessekreterare}

Mötessekreteraren har till uppgift att anteckna diskussioner, beslut, och eventuella reservationer under mötet.

\subsubsection*{Förslag till beslut}

\begin{attsatser}
	\item Sebiastian Selander väljs till mötessekreterare.
\end{attsatser}

\subsection{Val av protokolljusterare}

Protokolljusterare har till uppgift att kontrollera att protokollet i slutändan reflekterar de faktiska besluten och diskussionerna som fördes under sammanträdet; samt agera rösträknare vid slutna omröstningar.
Utöver protokolljusterarna så ska mötesordförande och mötessekreteraren signera protokollet.
Vid Divisionsstämmans sammanträden ska det vara två justerare.
Mötesordförande och mötessekreteraren kan inte vara justerare.

\subsubsection*{Förslag till beslut}

\emph{Inga förslag från styrelsen innan mötet}

\newpage
\section{Rapporter}

\subsection{Verksamhetsrapport}

\subsubsection{Styrelsen}

Mellan föregående möte och detta möte har Styrelsen genomfört

\begin{itemize}
	\item Mottagningen tillsammans med våra eminenta phaddrar; och
	\item tre styrelsemöten, inklusive det konstituerande mötet.
\end{itemize}

Flera saker är påbörjade men har inte avslutats under denna period.
Vi väntar svar från tredje parter.

\newpage
\section{Beslutsärenden}

Enligt Stadgan måste ändringar av Stadgan röstas igenom på två av Divisionsstämmans varandra följande möten.
Om en beslutpunkt innehåller ``första läsningen' innebär det att det är första gången beslutet tas upp för omröstning.
Om en beslutspunkt innehåller ``andra läsningen'' innebär det att beslutspunkten har röstats igenom förra stämmomötet, och beslutet behöver bekräftas för att gå igenom. 

\subsection{Proposition: Inför ordlistan i Stadgan (första läsningen)}

Det finns egentligen ingen poäng av att ha ordlistan som ett separat styrdokument skilt från Stadgan.
Stadgan är divisionens grundstomme, och det är alltid den man ska utgå ifrån.
För att göra saker och ting lättare för läsare av dokumentet önskar vi göra ordlistan till en del av Stadgan.

Syftet med ordlistan är att ha en central plats för styrdokument där man kan hitta exakta definitioner av visa termer som används i styrdokument.

\subsubsection*{Förslag till beslut}

\begin{attsatser}
	\item Styrdokumentet \emph{Ordlista för styr- och principdokument} införs som ett nytt första kapitel i stadgan, och försvinner som individuellt dokument.

	\item ändra 12 kap. 1 § första stycket i stadgan från

	\begin{displayquote}
		Dessa stadgar eller ordlistan för styr- och principdokument kan endast ändras genom beslut av divisionsstämman med två tredjedelars majoritet vid två på varandra följande sammanträden.
		Ett av divisionsstämmans sammanträden behöver vara ordinarie.
		Det måste vara minst 10 läsdagar mellan sammanträderna.
	\end{displayquote}

	till

	\begin{displayquote}
		Denna stadga kan endast ändras genom beslut av divisionsstämman med två tredjedelars majoritet vid två på varandra följande sammanträden.
		Ett av divisionsstämmans sammanträden behöver vara ordinarie.
		Det måste vara minst 10 läsdagar mellan mötena.
	\end{displayquote}

	\item ändra 4 § i \emph{Regler för dokumentsamlingen} från

	\begin{displayquote}
		Alla dokument ska skrivas med föreningens ordlista i åtanke.
		Ett ord definierat i ordlistan får inte användas med en annan definition än den som står i ordlistan.
	\end{displayquote}

	till

	\begin{displayquote}
        Alla dokument ska skrivas med stadgan i åtanke.
        Ett ord definierat i Stadgans ordlistan får inte användas med en annan definition än den som står i ordlistan.
    \end{displayquote}
\end{attsatser}

\newpage
\subsection{Proposition: Ändra \emph{årsredovisning} till \emph{årsrapport} (första läsningen)}

Enligt Stadgan så ska vi skapa en \emph{årsredovisning} av vår ekonomi.
Termen \emph{årsredovisning} har en juridisk innebörd i Sverige, och innebär att man måste följa visa precisa regler.
Vi har aldrig gjort en \emph{årsredovisning}, och kommer inte göra heller under överskådlig framtid.

Det vi däremot ska göra, som vi har tyvärr inte haft möjlighet att göra till detta möte, är att skapa en \emph{ekonomisk årsrapport} med all den information om ekonomin som ni behöver ha.
Självklart får ni grotta ner er i bokföringen om ni känner för det.

\subsubsection*{Förslag till beslut}

\begin{attsatser}
	\item 5 kap. 7 § första meningen i Stadgan ändras från

	\begin{displayquote}
        Senast sex månader efter räkenskapsårets början ska man på divisionsstämman redovisa divisionens årsredovisning samt revisorernas berättelse.
    \end{displayquote}

    till

    \begin{displayquote}
        Senast sex månader efter räkenskapsårets början ska man på Divisionsstämman redovisa divisionens ekonomiska årsrapport samt revisorernas berättelse.
    \end{displayquote}
\end{attsatser}

\newpage
\subsection{Proposition: Ändra \emph{årsmöte} till \emph{Divisionsstämman} (första läsningen)}

Divisionens högsta organ heter \emph{Divisionsstämman}, och inte \emph{årsmöte}.

\subsubsection*{Förslag till beslut}

\begin{attsatser}
	\item ändra 13 kap. 1 § i Stadgan från

	\begin{displayquote}
        Förslag om divisionens upplösning får behandlas endast på årsmöte.
    \end{displayquote}

	till

	\begin{displayquote}
		Förslag om divisionens upplösning får behandlas endast av Divisionsstämman.
	\end{displayquote}

	\item ändra 4 kap. 7 § i Stadgan från

	\begin{displayquote}
        Divisionens verksamhet och räkenskaper ska granskas av revisor utsedd av årsmötet.
    \end{displayquote}

	till

	\begin{displayquote}
		Divisionens verksamhet och räkenskaper ska granskas av revisor utsedd av Divisionsstämman.
	\end{displayquote}
\end{attsatser}

\newpage
\subsection{Proposition: Häv icke genomförda beslut (första läsningen)}

Styrelsebeslut och stämmobeslut gäller fram tills dess att de hävs.
Det kan medföra att beslut som fattades för flera år sedan, men som ännu inte verkställts, är tekniskt sett fortfarande giltiga.
För närvarande behöver varje ny styrelse gå igenom divisionens alla tidigare protokoll för att hitta eventuella beslut som de inte håller med om och häva dessa.
Vi föreslår att vi i stadgan inför att alla beslut behöver verkställas under det verksamhetsår där man fattade beslutet.
Om den nya styrelsen håller med ett tidigare icke verkställt beslut behöver de bara protokollföra att det tidigare beslutet fortfarande gäller.

\subsubsection*{Förslag till beslut}

\begin{attsatser}
	\item Under 5 kap. i Stadgan införa en ny paragraf med lydelsen

	\begin{displayquote}
		Samtliga icke verkställda beslut fattade av Divisionsstämman hävs efter det verksamhetsår under vilket beslutet fattades.
	\end{displayquote}

	\item Under 7 kap. i Stadgan införa en ny paragraf med lydelsen

	\begin{displayquote}
		Samtliga icke verkställda beslut fattade av Styrelsen hävs efter det verksamhetsår under vilket beslutet fattades.
	\end{displayquote}
\end{attsatser}

\newpage
\subsection{Proposition: Ändra medlemsperiod (första läsningen)}

Medlemsperioden är den period under vilket medlemskap i Datavetenskapsdivisionen gäller.
Idag är den från och med 1 oktober till och med 30 september.
Anledningen till att vi valde denna period från början är att Mecenats kårlegitimation uppdateras 1 oktober baserat på kårmedlemskap, även om kårmedlemskapet går ut tidigare.

Problemet är att nya studenter som går med i divisionen när de börjar studera behöver förnya sitt medlemskap kort efter de har gått med från första början.

Vi har pratat med IT-sektionsstyrelsen, och kommit fram till att vi bara behöver skicka över personnummer för att kontrollera kårmedlemskap.
Vi behöver då alltså inte själva kontrollera kårmedlemskapet, utan låta Göta göra det åt oss.
Vi kan därmed ha ha en medlemsperiod mer anpassad efter hur läsåret ser ut.

\subsubsection*{Förslag till beslut}

\begin{attsatser}
	\item för in följande i \emph{ordlistan}

	\begin{displayquote}
		\begin{description}
			\item[medlemsperiod] Perioden under vilket medlemskap i divisionen sträcker sig över.
		\end{description}
	\end{displayquote}

	\item ändra 3 kap. 3 § första stycket från

	\begin{displayquote}
		Medlemskap i divisionen sträcker sig från och med den första oktober till och med den sista september, hädanefter kallat ``medlemsperioden''.
	\end{displayquote}

	till

	\begin{displayquote}
		Medlemsperioden sträcker sig från från och med den första augusti till och med den sista juli.
	\end{displayquote}
\end{attsatser}

\newpage
\subsection{Proposition: Ta bort valberedningen (första läsningen)}

Valberedningen är det organ som har till uppgift att lägga fram förslag på personer man tror hade passat in bra på de förtroendeposter där man ska hålla inval.
Tanken är inte riktigt att valberedningen själv ska ta fram personerna som ska väljas, utan mer granska de personer som nuvarande förtroendevalda rekommenderar.

Styrelsen önskar ta bort Valberedningen då vi anser att nackdelarna överväger fördelarna med ett sådant organ.

En av de viktigaste fördelarna med Valberedningen som har lyfts fram är att den introducerar en mer opartiskhet i processen att rekommendera personer till stämman.

Det första problemet är att det behöves folk för att engagera sig i Valberedningen.
Det innebär att personer som annars hade haft tid och engagemang att organisera arrangemang och andra roliga saker behöver lägga den tiden för att ha möten och andra aktiviteter av byråkratisk karaktär.

Det andra problemet är att Valberedningen har stort sett varit icke-existerande det senaste decenniet.
Så Valberedningen är, och har varit det senaste tio åren, en pappersprodukt.

Därför tycker Styrelsen att det är rimligt att ta bort Valberedningen.

\subsubsection*{Förslag till beslut}

\begin{attsatser}
	\item ta bort ``valberedningen'' från listan i 2 kap. 1 § tredje stycket i stadgan.

	\item ta bort 8 kap. från stadgan.

	\item ta bort styrdokumentet \emph{Regler för valberedningen}.

	\item ta bort sista punkten från 1 § i styrdokumentet \emph{Regler för styrelsen}.

	\item ändra punkt 5 i 1 kap. 3 § andra stycket i styrdokumentet \emph{Regler för divisionssstämman} från

	\begin{displayquote}
		valberedningens eventuella förslag på personer till förtroendeuppdrag
	\end{displayquote}

	till

	\begin{displayquote}
		eventuella förslag på personer till förtroendeuppdrag
	\end{displayquote}

	\item ta bort punkt 1 och 2 i 1 § i styrdokumentet \emph{Regler för alla kommittéer}.
\end{attsatser}

\newpage
\subsection{Proposition: Ta bort Talmanspresidiet (första läsningen)}

Talmanspresidiet är det organ som har till uppgift att arrangera och leda Divisionsstämmans möten.
Tanken är att några få personer kan ta på sig en av Styrelsens regelbundna uppgifter som inte kräver att Styrelsen agerar som Styrelsen.
Man vill helt enkelt underlätta för Styrelsen så att några personer som annars inte hade varit aktiva i divisionen har möjlighet att göra några mindre arbetsuppgifter.

Styrelsen önskar ta bort Talmanspresidiet av ungefär samma skäl som vi önskar ta bort Valberedningen.

\subsubsection*{Förslag till beslut}

\begin{attsatser}
	\item ta bort ``talmanspresidiet'' från listan i 2 kap. 1 § tredje stycket i stadgan.

	\item ta bort 6 kap. från stadgan.

	\item ta bort 5 kap. 1 a § och 5 kap. 1 b § i Stadgan

	\begin{displayquote}
		Divisionsstämmans ordinarie sammanträden annonseras av talmansspresidiet.

		Divisionsstämman måste sammanträda under höstterminen innan nästkommande räkenskapsår börjar.

		Divisionsstämman måste sammanträda under vårterminen innan den första maj.

		Extra sammanträden kan kallas av styrelsen, talmanspresidiet, revisorerna eller minst 25 av divisionens medlemmar.
	\end{displayquote}

	och lägg till följande som nya stycken direkt under 5 kap. 1 §

	\begin{displayquote}
		Divisionsstämmans möten kan kallas av

		\begin{itemize}
			\item styrelsen;
			\item revisorerna; eller
			\item minst 25 av divisionens medlemmar.
		\end{itemize}

		Divisionsstämman måste hålla möte under höstterminen innan nästkommande räkenskapsår börjar.

		Divisionsstämman måste hålla möte under vårterminen innan 1 maj.
	\end{displayquote}

	\item ändra 5 kap. 6 § första stycket första meningen i Stadgan från

	\begin{displayquote}
		Efter divisionsstämmans sammanträde är avslutat har talmanspresidiet tjugo läsdagar på sig att publicera det justerade stämmoprotokollet.
	\end{displayquote}

	till

	\begin{displayquote}
		Efter divisionsstämmans sammanträde är avslutat har mötespresidiet tjugo läsdagar på sig att publicera det justerade stämmoprotokollet.
	\end{displayquote}

	\item ta bort ``talmanpresidiumet som helhet'' från listan i 11 kap. 3 § tredje stycket i Stadgan.

	\item ändra 11 kap. 2 b § i Stadgan från

	\begin{displayquote}
		Vid styrelsens avsättning ska en temporär styrelse väljas på innevarande sammanträde.
		Den temporära styrelsen tar över ordinarie styrelses befogenheter och skyldigheter tills dess en ny ordinarie styrelse har valts.
		Den temporära styrelsen eller talmanspresidiet ska annonsera ett extra sammanträde av divisionsstämman.
		Sammanträdet ska ske inom 15 läsdagar eller sista läsdag för terminen, vad som kommer först.
	\end{displayquote}

	till

	\begin{displayquote}
		Vid styrelsens avsättning ska en temporär styrelse väljas på innevarande sammanträde.
		Den temporära styrelsen tar över ordinarie styrelses befogenheter och skyldigheter tills dess en ny ordinarie styrelse har valts.
		Den temporära styrelsen ska annonsera ett extra sammanträde av divisionsstämman.
		Sammanträdet ska ske inom 15 läsdagar eller sista läsdag för terminen, vad som kommer först.
	\end{displayquote}

	\item ta bort 2 kap. från styrdokumentet \emph{Regler för divisionssstämman}.

	\item ändra samtliga referenser till talmanen i styrdokumentet \emph{Regler för divisionssstämman} till ``möteordförande''.

	\item ändra samtliga referenser till talmanspresidiet i styrdokumentet \emph{Regler för divisionssstämman} till ``mötespresidiet''.	
\end{attsatser}

\newpage
\subsection{Proposition: Mer flexibilitet i Dokumentsamlingen (första läsningen)}

Det finns några paragrafer i divisionens regler som på ett väldigt detaljerat sätt beskriver hur man ska arbeta med officiella dokument.
De flesta av dessa regler är onödigt detaljerade, och förhindrar mer än vad de möjliggör.

Anledningen till att de kom till är därför att man ville få bukt på problemet att massor av styrdokument fanns på flera olika platser (både digitalt och fysiskt), och man inte visste vilka av dessa som fortfarande gällde.

Styrelsen anser att man bör ta bort dessa, för att de förhindrar att man hittar nya smarta lösnignar på problemet.
Det finns möjligheter i framtiden att hårdare beskriva hur dokumenten ska hanteras, men det bör inte göras nu.

Resterande regler för dokumentsamlingen passar bra in i Stadgan, och bör därmed införas där för se till att all information om något finns på ett ställe.

\subsubsection*{Förslag till beslut}

\begin{attsatser}
	\item ta bort styrdokumentet \emph{Regler för alla kommittéer}.

	\item ändra 2 § första stycket från styrdokumentet \emph{Regler för dokumentsamlingen}.

	\begin{displayquote}
		Samtliga dokument ska vara skrivna med \LaTeX~och använda föreningens dokumentklass och relevant dokumentmall.
		Dokumentklassen utvecklas och förvaltas av styrelsen.
	\end{displayquote}

	till

	\begin{displayquote}
		Samtliga officiella dokument ska använda divisionens dokumentmall.
	\end{displayquote}

	\item ta bort 2 § andra stycket från styrdokumentet \emph{Regler för dokumentsamlingen}.

	\item ta bort 3 § från styrdokumentet \emph{Regler för dokumentsamlingen}.

	\item flytta samtliga resterande paragrafer i \emph{Regler för dokumentsamlingen} till 12 kap. i Stadgan.

	\item ta bort dokumentet \emph{Regler för dokumentsamlingen}.
\end{attsatser}

\newpage
\subsection{Proposition: Ersätt Dokumentsamlingen med ett Reglemente (första läsningen)}

Vi tycker att det är enklare att hålla koll på två större dokument, än flera små dokument.
Därför föreslår vi att vi konsoliderar alla styrdokument som Divisionsstämman har beslutat om, förutom Stadgan, till ett dokument och kallar det \emph{Reglementet}.

\subsubsection*{Förslag till beslut}

\begin{attsatser}
	\item ändra 2 kap. 1 § andra stycket i Stadgan från

	\begin{displayquote}
		Samtliga organ behöver även följa de regler som definieras i styr- och principdokumenten publicerade i dokumentsamlingen.
		Undantag från divisionens styr- och principdokumenten publicerade i dokumentsamlingen kan endast fattas av det organ som antagit dokumentet, eller av ett organ överställt det organ som har antagit dokumentet.
		För att ett undantag ska kunna göras behöver man uppfylla samma krav som om man skulle ändra dokumentet.
	\end{displayquote}

	till

	\begin{displayquote}
		Samtliga organ behöver även följa divisionens styr- och principdokument som de berörs av.
		Undantag från divisionens styr- och principdokumenten kan endast fattas av det organ som antagit dokumentet, eller av ett organ överställt det organ som har antagit dokumentet.
		För att ett undantag ska kunna göras behöver man uppfylla samma krav som om man skulle ändra dokumentet.
	\end{displayquote}

	\item ändra 12 kap. 2 § i Stadgan från

	\begin{displayquote}
		Styrdokument eller principdokument får införas i dokumentsamlingen eller ändras endast efter divisionsstämman fattat beslut med två tredjedelars majoritet.

		För att ett dokument ska gälla krävs det att dokumentet är publicerat i divisionens dokumentsamling.
	\end{displayquote}

	till

	\begin{displayquote}
		Reglementet får endast ändras efter divisionsstämman fattat beslut med två tredjedelars majoritet.

		För att en regel ska gälla krävs det att dokumentet är publicerat i Reglementet.
	\end{displayquote}

	\item ändra 2 § första stycket i \emph{Regler för dokumentsamlingen} från

	\begin{displayquote}
		Dokumentsamlingen är är ett begrepp på det repository som förvarar alla publika officiella dokument som Datavetenskapsdivisionen antar.
		Syftet med dokumentsamlingen är att förvara alla dokument i ett läsbart format på ett och samma ställe.
	\end{displayquote}

	till

	\begin{displayquote}
		Reglementet ärr ett dokument som innehåller samtliga regler som Divisionsstämman har beslutat om.
		Samtliga regler ska vara i ett läsbart format på ett och samma ställe.
	\end{displayquote}

	\item flytta samtliga styrdokument som Divisionsstämman beslutat om till Reglementet.
\end{attsatser}

\newpage
\subsection{Proposition: Minska antalet styrelseledamöter (första läsningen)}

Tidigare var det ända sättet att bli aktiv i divisionen var att gå med i Styrelsen, och därmed gjorde Styrelsen allt jobb.
Det inkluderar att anordna Mottagningen, sittningar, sköta om Monaden, och viss representation mot universitet.
Följderna var att arbetsuppgifter som åligger Styrelsen att göra inte genomfördes, då man fokuserade på att till exempel anordna sittningar eller Mottagningen.
Detta ledde till en process att strukturera upp divisionen som började för några år sedan och fortsätter.
Visionen är att Styrelsen ska fokusera på sina styrelseuppdrag och se till att dessa sköts och inte försummas till förmån för andra aktiviteter.

Styrelsen vill tydliggöra detta genom att minska ner Styrelsen till endast de poster som krävs för att genomföra Styrelsens huvudsakliga uppgift.
Studiesociala aktiviteter kommer arrangeras av kommittéer, se propositionen \textit{Omstrukturera kommitéer och ta bort intressegrupper}.

\subsubsection*{Förslag till beslut}

\begin{attsatser}
	\item ta bort ``övriga styrelsemedlemmar'' från listan i 7 kap. 1 § i stadgan.
	\item ändra 7 kap. 3 § i stadgan från

	\begin{displayquote}
		Styrelsen ska högst bestå av sju medlemmar.
	\end{displayquote}

	till

	\begin{displayquote}
		Styrelsen kan högst bestå av fem medlemmar.
	\end{displayquote}

	\item ta bort 6 § från styrdokumentet \emph{Regler för styrelsen}.
\end{attsatser}

\newpage
\subsection{Proposition: Omstrukturera kommittéer och ta bort intressegrupper (första läsningen)}

Då Styrelsen fokusera på att genomföra styrelsearbete och de vanliga administrativa uppgifterna ligger det på kommittéerna att arrangera studiesociala aktiviteter och ansvarar för att det praktiska arbetet genomförs.

Styrelsen föreslår att divisionen ska ha en liknande modell på kommittéer som sektionsföreningar vid Stockholms universitets Studentkår.
Det innebär att Datavetenskapsdivisionen kommer ha flera kommittéer som har varsina arbetsområden.
Till varje kommitté ska Divisionsstämman välja en ordförande, som ska inneha posten under ett verksamhetsår.
Det är endast kommittéordföranden som väljs av Divisionsstämman.
Styrelsen kommer alltså inte ha makt att välja kommittéordförande.

Ordförande har är den som ansvarar för att kommitténs uppgifter genomförs; och välja in nya medlemmar.
Att vara ansvarig för att något sker betyder \textbf{inte} att man ska genomföra arbetet.
Ordförande är den som har till uppdrag att stämma av med styrelsen.
Utöver själva arbetsuppgifterna så ansvarar ordförande för att välja in nya medlemmar i kommittén.

Då vi gör kommittéer mycket mera flexibla försvinner syftet med intresseföreningar.
Därför finns det ingen anledning att ha kvar dem.
Det finns idag inga intresseföreningar.

\subsubsection*{Förslag till beslut}

\begin{attsatser}
	\item ta bort punkten ``antal förtroendevalda'' från listan i 9 kap. 3 § i stadgan.
	\item ändra 9 kap. 4 § från

	\begin{displayquote}
		Varje kommitté består av en ordförande och övriga förtroendevalda.

		Endast medlemmar kan väljas till förtroendeposter i en kommitté.
		Man kan endast vara ordförande för en kommitté i divisionen under den mandatperiod som man innehar posten.

		Divisionsstämman väljer ordförande och de övriga förtroendevalda till kommittéer.
		Valet av ordförande sker separat från valet av de övriga förtroendevalda.

		Vid skapande av en ny kommitté ska val av kommitténs ordförande och förtroendevalda införas i schemat för innevarande sammanträde.
	\end{displayquote}

	till

	\begin{displayquote}
		Varje kommitté består av en ordförande och övriga kommittémedlemmar.
		Endast divisionsmedlemmar kan bli medlemmar i kommittén.

		Kommittéordförande väljs av Divisionsstämman.
		Mandatperioden för kommittéordförande är densamma som divisionens verksamhetsår.

		Övriga medlemmar väljs in av kommittéordförande.
		Det finns ingen bestämd mandatperiod för övriga kommittémedlemmar.

		Vid skapande av en ny kommitté ska val av kommittéordförande införas i schemat för innevarande möte.
		Detta ska göras även om mötesschemat är fastställt tidigare under mötet.
	\end{displayquote}

	\item ta bort ``intressegrupper'' från listan i 2 kap. 1 § tredje stycket i stadgan.
	\item ta bort 10 kap. i stadgan.
\end{attsatser}

\newpage
\subsection{Fyllnadsval till Styrelsen}

Vid Divisionsstämmans möte den 14 juni beslutade Divisionstämman att välja in Albin Otterhäll som divisionsordförande, och Morgan Thowsen som övrig styrelseledamot för verksamhetsåret 2021.
Morgan innehar idag posten som kassör.
Enligt Stadgan behöver vi även en vice ordförande; en sekreterare; och en SAMO.
Styrelsen har pratat med flera potentiella kandidater, och kommit fram till att följande personer hade passat in i den nuvarande styrelsen.

Personerna som väljs in den 15 september väljs in som styrelseledamöter för verksamhetsåret 2021.
Senare under hösten kommer Divisionsstämman hålla ett ytterligare där man kommer välja in Styrelsen 2022, som ska gå på den januari.
Det är den nuvarande Styrelsens förhoppning att personerna som väljs in på detta möte även kommer ställa upp för omval till nästa verksamhetsår.

\subsubsection*{Förslag till beslut}

\begin{attsatser}
	\item Samuel Hammmersberg (DV'20); Sebastian Selander (DV'20); och Tekla Siesjö (DV'20) väljs in till den nuvarande Styrelsen under verksamhetsåret 2021.
\end{attsatser}

\newpage
\section{Diskussioner}\label{sec:discussioner}

\subsection{Ny logotyp}

Den 29 oktober 2020 under styrelsemöte 4 beslutade styrelsen 2020 att den dåvarande logotypen inte längre skulle användas, på grund av att den bryter mot upphovsrättslagen.
Först och främst handlade det om att styrelsen inte vill ha en officiell logotyp som bryter mot upphovsrättslagen, och eventuellt universitetets regler för hur deras namn används.
Om våran logga bryter mot lagar, så kan det skapa problem näringslivspartners.
Sedan skulle nog inte Capcom, upphovsrättsinnehavaren av Mega Man, bli jätteglada om de fick reda på att vi använde Mega Man utan tillåtelse.

Därför har vi börjat samla in förslag på nya logotyper.
De nuvarande inkomna förslagen hittar ni här: \url{https://cloud.dvet.se/nextcloud/index.php/s/WLGfd23XKLmMKJw}.
Självklart får ni jättegärna komma med era åsikter och era egna förslag.

Vi planerar att hålla omröstning om vilket förslag som ska bli vår officiella logga under den ordinarie stämman innan årsskiftet (mer information senare under året).

\section{Avslutande av möte}

Mötet förväntas avslutas mellan klockan 20 och klockan 21.

% \stämmosignaturer

\end{document}
