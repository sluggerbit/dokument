\documentclass[protokoll]{dvd}

\KOMAoptions{
    headwidth = 18cm,
    footwidth = 18cm,
}

\begin{document}

\title{Styrelsemöte 5}
\subtitle{2021}
\author{Styrelsen}
\date{2021-11-01}

\textbf{Datum:} \csname @date\endcsname\\
\textbf{Tid:} 10:00\\
\textbf{Plats:} Styrelserummet\\
\textbf{Styrelsemedlemmar:}
\begin{närvarande_förtroendevalda}
    \förtroendevald{Ordförande}{Albin Otterhäll}{Ja}
    \förtroendevald{Kassör}{Morgan Thowsen}{Ja}
    \förtroendevald{Vice ordförande}{Samuel Hammersberg}{Ja}
    \förtroendevald{Sekreterare}{Sebastian Selander}{Ja}
    \förtroendevald{SAMO}{Tekla Siesjö}{Ja}
\end{närvarande_förtroendevalda}

% \textbf{Övriga medlemmar:}






\section{Öppnande av möte}

Mötet öppnades av Albin Otterhäll 10.10






\section{Runda bordet}

Runda bordet innebär att varje person berättar hur de känner sig.
Man kan till exempel berätta att man är stressad på grund av en inlämning, irriterad på sin granne, eller bara väldigt glad därför att man ligger i fas med plugget.








\section{Formalia}

\subsection{Styrelsens beslutbarhet}

\blockquote[7 kap. 5 \S~första stycket i stadgan][]{%
    Styrelsen är endast beslutsmässig då samtliga styrelsemedlemmar har fått kallelsen till styrelsemötet och minst hälften av styrelsemedlemmarna är närvarande.
    Ordförande eller vice ordförande måste vara närvarande när beslut tas.
}

Den 27 oktober föreslog Albin tid och datum för nästa styrelsemöte. Samma dag fastställdes datumet i styrelsens discordserver.

\subsubsection*{Beslut}

\begin{attsatser}
    \item Styrelsen har uppnått kraven i 7 kap. 5 § första stycket i stadgan och är därmed beslutbar.
\end{attsatser}






\subsection{Fastställande av mötesschema}

För att styrelsen ska kunna fatta ett styrelsebeslut eller protokollföra en diskussion behöver punkten i mötesschemat där styrelsen ska fatta beslut vara inlagd eller föras in i mötesschemat senast vid den här punkten.

\subsubsection*{Beslut}

\begin{attsatser}
    \item mötesschemat fastställs utan några förändringar.
\end{attsatser}






\subsection{Val av protokolljusterare}

Protokolljusterare har till uppgift att kontrollera att protokollet i slutändan reflekterar de faktiska besluten och diskussionerna som fördes under mötet.
Utöver protokolljusteraren så ska mötesordförande och mötessekreteraren signera protokollet.
Vid styrelsemöten ska det endast vara en justerare.
Mötesordförande och mötessekreteraren kan inte vara justerare.


\subsubsection*{Beslut}
\begin{attsatser}
    \item Samuel Hammersberg väljs till protokolljusterare
\end{attsatser}



\newpage


\section{Rapporter}


\subsection{Styrelseövergripande}

\subsubsection*{Låsa in brädspel}

Under senaste tre styrelsemöten har denna punkt varit aktuell men ingen har åtagit sig att göra arbetet. 

\subsubsection*{Beslut}
\begin{attsatser}
    \item bordlägga punkten 
\end{attsatser}


\newpage



\subsection{Ordförande}

\subsubsection*{Möte med Thor}
    Thor tänker att han kommer äska pengar till DVD som förening, så får vi fördela pengarna till kommittéerna själva. Han tror att de kommer godkännas utan några större krångel.

    För mottagning av masterprogrammen (CS och ADS) tyckte han det lät rimligt att få stöd, och såg inte att det skulle vara något större problem att få det.

    Han kommer äska för att vi ska få en större mottagningsbudget. Han sa att här var han osäker när det gällde både summan och vart han ska äska, så vi får se hur vi lägger upp det.

\subsection*{Tricksat med GSuite / bråkat med Google}
    Har varit i kontakt med Googles support, de är inte värst hjälpsamma.
    Ärendet går frammåt, men långsamt går det.




\subsection{Vice ordförande}
        Mummanyckeln är borta. Var under sittningen ute med skräp och det kan vara därför mummanyckeln är borta. Vi letar fram den, det bör inte vara några problem 



\subsection{SAMO}

    \subsubsection*{Medlemsregistrering}
        Har tillsammans med Albin kommit fram till ett förslag där ett formulär ska skickas till Studentoffice, efter bekräftelse från Studentoffice och Göta kan vi bocka av medlemmar i vårt register.
        Vi behöver dubbelkolla med alla nuvarande (tidigare) medlemmar om de vill förnya sitt medlemskap.

    Formuläret behöver skapas, google forms är tanken just nu.

    \subsubsection*{Köket}
        Ett mail är sammanställt som ska skickas till Roger om att utöka köket så det är handikappsanpassat.
        Mailet beräknas skickas under veckan.

\newpage



\subsection{Kassör}
    \subsubsection*{Deklaration}
    Har fyllt i årets skattedeklaration och skickat in den. 



\subsection{Sekreterare}
    \subsubsection*{Protokoll}
    Fortsätter som vanligt. Stämmans protokoll ska påbörjas snart.


\newpage

\section{Beslutsärenden}

\subsection{Fullmakt Swedbank}

    Vi fattade det här beslutet för första gången vid Styrelsemöte 1.
    Swedbank tolkar våran nuvarande Stadga att för att styrelsebeslut ska vara beslutskraftiga så behöver samtliga poster i Styrelsen vara tillsatta.
    Sedan det andra Divisionsstämmomötet 2021 så är Styrelsen 2021 fulltaliga, och kan därmed fatta beslut i frågan.

\subsubsection*{Förslag till beslut:}
    \begin{attsatser}
        \item Albin Otterhäll (960131-3753) och Morgan Thowsen (19910720-0058), eller den de/den sätter i sitt ställe, att oinskränkt företräda föreningen var för sig. \newline Denna rätt innebär bland annat, men inte uteslutande, att företräda föreningen gentemot Swedbank AB samt att utse behörighetsadministratörer och företagsanvändare i föreningens internetbank.
    \end{attsatser}

\subsubsection*{Beslut}
    \begin{attsatser}
        \item förslaget till beslut bifalles
    \end{attsatser}


\newpage


\section{Diskussioner}\label{sec:discussioner}

\subsection{Pengar till kommittéer}
    Pengar från Göteborgs Universitet skulle ta minimum 2-3 månader att komma in.
    För att äska pengar från Göta behöver man ha något konkret pengarna ska gå till, exempelvis kläder.

\subsection{Kommitéers sociala kontrakt}
    Vi diskuterar ifall styrelsen borde godkänna dessa, eller ha en åsikt överhuvudtaget.


\subsection{Saker som bör diskuteras under stämman}
    De punkter som bara röstats på en gång bör tas upp under nästa stämma.

    Stämman bestämmer antingen en nuvarande föreslagen logga, bestämmer en arbetsgrupp som tar fram en logga, eller bordlägger ärendet.

    Albin har skapat en motion om att skapa en arbetsmarksnadskommitté. Den skickas ut till de andra kommittéerna.

    Önskemål om att korta ner den byråkratiska delen av mötet. Gör det tydligt att protokollet bör läsas innan.

\newpage




\section{Avslutande av möte}

\subsection{Mötesutvärdering}
    Vi är fortfarande långsamma. Vissa saker diskuteras utan att det behövs. Större andel diskussionspunkter bör ske över Discord

\subsection{Nästa möte}
   Fredag 12 November 13.15 preliminärt datum och tid för nästa möte 

\subsection{Mötets avslutande}

    Mötet avslutades 11.24

\styrelsesignaturer

\end{document}
