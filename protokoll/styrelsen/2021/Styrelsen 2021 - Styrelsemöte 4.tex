\documentclass[protokoll]{dvd}

\KOMAoptions{
    headwidth = 18cm,
    footwidth = 18cm,
}

\begin{document}

\title{Styrelsemöte 4}
\subtitle{2021}
\author{Styrelsen}
\date{2021-10-11}

\textbf{Datum:} \csname @date\endcsname\\
\textbf{Tid:} 12:00\\
\textbf{Plats:} Styrelserummet\\
\textbf{Styrelsemedlemmar:}
\begin{närvarande_förtroendevalda}
    \förtroendevald{Ordförande}{Albin Otterhäll}{}
    \förtroendevald{Kassör}{Morgan Thowsen}{}
    \förtroendevald{Vice ordförande}{Samuel Hammersberg}{}
    \förtroendevald{Sekreterare}{Sebastian Selander}{}
    \förtroendevald{SAMO}{Tekla Siesjö}{}
\end{närvarande_förtroendevalda}
% \textbf{Övriga medlemmar:}






\section{Öppnande av möte}

Mötet beräknas öppnas av Albin Otterhäll klockan 12:00.








\section{Runda bordet}

Runda bordet innebär att varje person berättar hur de känner sig.
Man kan till exempel berätta att man är stressad på grund av en inlämning, irriterad på sin granne, eller bara väldigt glad därför att man ligger i fas med plugget.









\section{Formalia}

\subsection{Styrelsens beslutbarhet}

\blockquote[7 kap. 5 \S~första stycket i stadgan][]{%
    Styrelsen är endast beslutsmässig då samtliga styrelsemedlemmar har fått kallelsen till styrelsemötet och minst hälften av styrelsemedlemmarna är närvarande.
    Ordförande eller vice ordförande måste vara närvarande när beslut tas.
}

Tisdag den 5 oktober föreslog Albin tid och datum för nästa styrelsemöte.
Den 7 oktober fastställdes datumet i styrelsens discordserver.

\subsubsection*{Förslag till beslut}

\begin{attsatser}
    \item Styrelsen har uppnått kraven i 7 kap. 5 § första stycket i stadgan och är därmed beslutbar.
\end{attsatser}




\subsection{Fastställande av mötesschema}

För att styrelsen ska kunna fatta ett styrelsebeslut eller protokollföra en diskussion behöver punkten i mötesschemat där styrelsen ska fatta beslut vara inlagd eller föras in i mötesschemat senast vid den här punkten.

\subsubsection*{Förslag till beslut}

\begin{attsatser}
    \item Mötesschemat fastställs utan några förändringar.
\end{attsatser}





\subsection{Val av protokolljusterare}

Protokolljusterare har till uppgift att kontrollera att protokollet i slutändan reflekterar de faktiska besluten och diskussionerna som fördes under mötet.
Utöver protokolljusteraren så ska mötesordförande och mötessekreteraren signera protokollet.
Vid styrelsemöten ska det endast vara en justerare.
Mötesordförande och mötessekreteraren kan inte vara justerare.

Med tanke på att Tekla var protokolljusterare senast kan vi låta någon annan ta den rollen idag.

\subsubsection*{Förslag till beslut}

\begin{attsatser}
\item Samuel Hammersberg väljs till protokolljusterare
\end{attsatser}






\section{Rapporter}

\subsubsection{Låsa in brädspel}

Under senaste två styrelsemöten har denna punkt varit aktuell men ingen har åtagit sig att göra arbetet. 









\newpage

\subsection{Ordförande}


\subsubsection{Nästa ordförande}


\subsubsection{GSuite}

Albin har tittat på GSuite och försökt lösa problemen med det.



\newpage

\subsection{Vice ordförande}


\subsubsection{Inkommet förslag - vem är i Monaden}

Vi har fått förslag utifrån om att implementera något sätt att se vilka personer är i Monaden. Exempelvis som D eller IT har gjort det genom att kolla vilka datorer som är uppkopplade på routern.



\subsubsection{Styrelsen undviker ansvarsroller inom kommittéer}

Det har märkts att kommittéerna låter styrelsemedlemmar ta huvudroller under möten. Detta kan medföra att medlemmarna tror det är styrelsens åsikter som höjs istället för personens egna åsikter. 




\newpage

\subsection{Kassör}

\subsubsection{Bankkonto hos Swedbank}

\begin{description}[style=multiline, widest=00.00, align=left, leftmargin=2.5cm]
    \item[2021-10-11] Ingen uppdatering i ärendet i nuläget.
\end{description}


\subsubsection{Signaturer på protokoll}

I dagsläget signerar vi varje sida på protokollen. Skulle det inte räcka att endast signera sista sidan?




\newpage

\subsection{Sekreterare}

\subsubsection{Inkommen punkt från kommitté}

Festkommittén har har påpekat att två sittningar per månad kan komma och ställas som ett krav. De tycker detta är orimligt och önskar att det tas upp.


\subsubsection{Styrelsemailen}

Styrelsemailen vidarebefordras till våra personliga mailadress just nu och detta gör att den personliga mailen blir rörig. Vi behöver hitta ett alternativ där styrelsemailen kan separeras från våra personliga mailadresser.




% \newpage

% \subsection{SAMO}





\newpage

\section{Beslutsärenden}


\subsubsection{Möte med Divisionsstämman}

\subsubsection{Datum för paxxning av Monaden}

När det blivande sexmästeriet och det blivande rustmästeriet/PR bad om att paxxa Monaden röstade vi om det direkt. Den här gången fungerade det bra, men Albin tänker att det kan skapa problem i framtiden. Till exempel kan vi ha fallet att två eller flera kommittéer planerar att hålla event i Monaden samma dag, och vi godkänner paxxningen för kommittén som ber om det först även om vi senare inser att vi hellre hade gett den till en annan.

Vi kan inte heller endast behandla Monadenpaxxningar på våra möten, då vi endast håller dem var tredje vecka.

Därför föreslår Albin att vi skapar en process där vi behandlar alla Monadenpaxxningar via per capsulam beslut, och att vi sedan protokollför samtliga paxxningar på våra styrelsemöten.



\subsubsection*{Förslag till beslut}

\begin{attsatser}
    \item vi endast röstar på Monadenpaxxningar på fredagar varje vecka
    \item vi på något sätt listar alla paxxningsförfrågningar som vi fått in 
    \item vi informerar tydligt om att vi behandlar paxxningsförfrågningar en specifik idag, och att alla som vill få sina paxxningar införda den veckan behöver ha dem inskickade senast på torsdag
\end{attsatser}










\subsubsection{Ansvar för medlemsskap}

Vem av oss är det som ska sköta vårt medlemsregister? I uppdraget ingår det att se till att varje person som önskar registrera sig behöver inkomma med

- för- och efternamn;
- personnummer; och
- e-mejladress.

Först behöver vi kontrollera att personen är inskriven på något av de programmen som föreningen är till för. Om vi får ett OK på det behöver vi kontrollera om personen ifråga är kårmedlem, eller tvärt om. Albin tror att de flesta kommer falla på den senare punkten om de faller på något. Det kommer bli en del mejlande till studieadministrationen och IT-sektionens ordförande.

Vi kommer behöva sköta den här processen manuellt om vi inte lyckas bygga något system där vi kan kontrollera de här sakerna automatiskt. Eller i alla fall automatiserar delar av processen. Albin har tankar på hur det kan se ut.


\subsubsection*{Förslag till beslut}

\begin{attsatser}
    \item sekreteraren åläggs att sköta medlemsregistret.
\end{attsatser}








\section{Avslutande av möte}

\subsection{Mötesutvärdering}

\subsection{Nästa möte}

\subsection{Mötets avslutande}

Mötet beräknas avslutas klockan 13:00.

\styrelsesignaturer

\end{document}
