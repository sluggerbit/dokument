\documentclass[protokoll]{dvd}

\KOMAoptions{
    headwidth = 18cm,
    footwidth = 18cm,
}

\begin{document}

\title{Styrelsemöte 4}
\subtitle{2021}
\author{Styrelsen}
\date{2021-10-11}

\textbf{Datum:} \csname @date\endcsname\\
\textbf{Tid:} 12:00\\
\textbf{Plats:} Styrelserummet\\
\textbf{Styrelsemedlemmar:}
\begin{närvarande_förtroendevalda}
    \förtroendevald{Ordförande}{Albin Otterhäll}{Ja}
    \förtroendevald{Kassör}{Morgan Thowsen}{Ja}
    \förtroendevald{Vice ordförande}{Samuel Hammersberg}{Ja}
    \förtroendevald{Sekreterare}{Sebastian Selander}{Ja}
    \förtroendevald{SAMO}{Tekla Siesjö}{Ja}
\end{närvarande_förtroendevalda}

% \textbf{Övriga medlemmar:}






\section{Öppnande av möte}

Mötet öppnades av Albin Otterhäll kl. 12.04






\section{Runda bordet}

Runda bordet innebär att varje person berättar hur de känner sig.
Man kan till exempel berätta att man är stressad på grund av en inlämning, irriterad på sin granne, eller bara väldigt glad därför att man ligger i fas med plugget.








\section{Formalia}

\subsection{Styrelsens beslutbarhet}

\blockquote[7 kap. 5 \S~första stycket i stadgan][]{%
    Styrelsen är endast beslutsmässig då samtliga styrelsemedlemmar har fått kallelsen till styrelsemötet och minst hälften av styrelsemedlemmarna är närvarande.
    Ordförande eller vice ordförande måste vara närvarande när beslut tas.
}

Tisdag den 5 oktober föreslog Albin tid och datum för nästa styrelsemöte.
Den 7 oktober fastställdes datumet i styrelsens discordserver.

\subsubsection*{Beslut}

\begin{attsatser}
    \item Styrelsen har uppnått kraven i 7 kap. 5 § första stycket i stadgan och är därmed beslutbar.
\end{attsatser}






\subsection{Fastställande av mötesschema}

För att styrelsen ska kunna fatta ett styrelsebeslut eller protokollföra en diskussion behöver punkten i mötesschemat där styrelsen ska fatta beslut vara inlagd eller föras in i mötesschemat senast vid den här punkten.

\subsubsection*{Beslut}

\begin{attsatser}
    \item mötesschemat fastställs utan några förändringar.
\end{attsatser}







\subsection{Val av protokolljusterare}

Protokolljusterare har till uppgift att kontrollera att protokollet i slutändan reflekterar de faktiska besluten och diskussionerna som fördes under mötet.
Utöver protokolljusteraren så ska mötesordförande och mötessekreteraren signera protokollet.
Vid styrelsemöten ska det endast vara en justerare.
Mötesordförande och mötessekreteraren kan inte vara justerare.


\subsubsection*{Beslut}
\begin{attsatser}
    \item Samuel Hammersberg väljs till protokolljusterare
\end{attsatser}





\section{Rapporter}

\subsubsection{Låsa in brädspel}

Under senaste två styrelsemöten har denna punkt varit aktuell men ingen har åtagit sig att göra arbetet. 

\subsubsection*{Beslut}
\begin{attsatser}
    \item punkten bordläggs
\end{attsatser}









\newpage

\subsection{Ordförande}

\subsubsection{Programledningsmöte}
Ändrad programplan, grundläggande datorteknik obligatorisk i läsperiod 2, linjär algebra flyttad till läsperiod 3, matematisk analys flyttad till läsperiod 4.

\subsubsection{Institutionsråd}
Industrin jagar folk. De har ett stort överskott pengar, cirka 16 miljoner kronor.

\subsubsection{Diskussionskväll om kommittéer}
28 september hade vi diskussionskväll i Monaden angående kommittéer. Folk uppskattade det och styrelsens förslag mottogs väl.

\subsubsection{GSuite}
Albin har tittat på GSuite och försökt lösa problemen med det.






\newpage

\subsection{Vice ordförande}


\subsubsection{Möte med Monadenkommittén}
Samuel har varit med på Monadenkommitténs möten. De har gått bra och Samuel kan nu ta en mer tillbakalutad roll i kommittéen.






\newpage

\subsection{Kassör}

\subsubsection{Bankkonto hos Swedbank}

\begin{description}[style=multiline, widest=00.00, align=left, leftmargin=2.5cm]
    \item[2021-10-11] Ingen uppdatering i ärendet i nuläget.
\end{description}




\subsection{Sekreterare}

\subsubsection{Klarskrivit protokoll}
Sebastian har klarskrivit protokollen för tidigare styrelsemöte samt divisionsstämma. De är utskrivna och redo att signeras.



\subsection{SAMO}

\subsubsection{Möte med Thor}
Tekla har haft möte med Thor där de söker en person som presenterar DV för gymnasiestudenter.
Möjligtvis är det att de söker en studentambassadör. Tekla ska kolla med Thor och återkomma om det är det de söker.

Thor kommer även förbi med diverse spel.

\newpage


\section{Beslutsärenden}

\subsubsection{Bokning av Monaden}

\subsubsection*{Beslut}
\begin{attsatser}
    \item Monadenkommittén bokar Monaden 16 oktober
    \item festkommittén bokar Monaden 30 oktober
\end{attsatser}

\subsubsection{Möte med Divisionsstämman}

Beslut om vilken logga står som en punkt för kommande stämma. 
För att kunna bestämma logga behövs konkreta förslag att presentera under stämman, vilket styrelsen inte har för tillfället.

\subsubsection*{Beslut}
\begin{attsatser}
    \item nästa möte bokas 17 november 17.17 i Monaden 
\end{attsatser}

\subsubsection{Ansvar för medlemsskap}

Vem ska sköta divisionens medlemsregister? I uppdraget ingår det att se till att varje person som önskar registrera sig behöver inkomma med

- för- och efternamn; \\
- personnummer; \\
- e-mejladress. \\

Först behöver styrelsen kontrollera att personen är inskriven på något av de programmen som föreningen är till för. Om styrelsen får ett OK på det behöver de kontrollera om personen ifråga är kårmedlem, eller tvärt om. Albin tror att de flesta kommer falla på den senare punkten om de faller på något. Det kommer bli en del mejlande till studieadministrationen och IT-sektionens ordförande.

Styrelsen kommer behöva sköta den här processen manuellt om de inte lyckas bygga något system där de kan kontrollera det automatiskt. Eller i alla fall automatiserar delar av processen. Albin har tankar på hur det kan se ut.


\subsubsection*{Beslut}

\begin{attsatser}
    \item SAMO tar på sig ansvaret att hålla koll på medlemsskap
\end{attsatser}




\subsubsection{Datum för paxxning av Monaden}

När det blivande sexmästeriet och det blivande rustmästeriet/PR bad om att paxxa Monaden röstade styrelsen om det direkt. Den här gången fungerade det bra, men Albin tänker att det kan skapa problem i framtiden. Till exempel kan fallet att två eller flera kommittéer planerar att hålla event i Monaden samma dag uppstå, och styrelsen godkänner paxxningen för kommittén som ber om det först även om styrelsen senare inser att de hellre hade gett den till en annan.

Styrelsen kan inte heller endast behandla Monadenpaxxningar på styrelsemöten, då de endast håller dem var tredje vecka.

Därför föreslår Albin att styrelsen skapar en process där de behandlar alla Monadenpaxxningar via per capsulam beslut, och att de sedan protokollför samtliga paxxningar på styrelsemöten.


\subsubsection*{Förslag till beslut}

\begin{attsatser}
    \item vi endast röstar på Monadenpaxxningar på fredagar varje vecka
    \item vi på något sätt listar alla paxxningsförfrågningar som vi fått in 
    \item vi informerar tydligt om att vi behandlar paxxningsförfrågningar en specifik idag, och att alla som vill få sina paxxningar införda den veckan behöver ha dem inskickade senast på torsdag
\end{attsatser}


\subsubsection*{Beslut}
\begin{attsatser}
    \item attsatserna röstas ned 
\end{attsatser}






\section{Diskussioner}\label{sec:discussioner}

\subsubsection{Inkommen punkt från kommitté}

Festkommittén har har påpekat att två sittningar per månad kan komma och ställas som ett krav. De tycker detta är orimligt och önskar att det tas upp.

Om man är medlem i en kommitté som styrelsemedlem bör det vara väldigt tydligt att man är där antingen som styrelsemedlem eller privatperson, och att man inte skiftar mellan de rollerna under ett möte.


\subsubsection{Styrelsemailen}

Styrelsemailen vidarebefordras till styrelsens personliga mailadress just nu och detta gör att den personliga mailen blir rörig. Ett alternativ där styrelsemailen kan separeras från personliga mailadresser behöver undersökas.


\subsubsection{Signaturer på protokoll}

I dagsläget signeras varje sida på protokollen. Detta känns överflödigt och finns det möjlighet att endast signera sista sidan.
\\
Enligt god föreningsed ska man signera varje sida på ett protokoll. Enligt stadgan ska styrelsen samt protokolljusterare signera varje sida på protokollen. 

\subsubsection{Styrelsen undviker ansvarsroller inom kommittéer}

Det har märkts att kommittéerna låter styrelsemedlemmar ta huvudroller under möten. Detta kan medföra att medlemmarna tror det är styrelsens åsikter som höjs istället för personens egna åsikter. 


\subsubsection{Inkommet förslag - vem är i Monaden}

Styrelsen har fått förslag utifrån om att implementera något sätt att se vilka personer är i Monaden. Exempelvis som D eller IT har gjort det genom att kolla vilka datorer som är uppkopplade på routern.

Punkten är diskuterad och styrelsen är emot. Det är heller inte styrelsens ansvarsområde.


\subsubsection{Nästa ordförande}

Samuel Hammersberg erbjuder sig att övergå till ordförande. Sebastian Selander erbjuder sig övergå till vice ordförande om styrelsen hittar någon som kan tänka sig vara sekreterare.

\newpage






\section{Avslutande av möte}

\subsection{Mötesutvärdering}

\subsection{Nästa möte}

\subsection{Mötets avslutande}

Mötet avslutades klockan 13.36

\styrelsesignaturer

\end{document}
