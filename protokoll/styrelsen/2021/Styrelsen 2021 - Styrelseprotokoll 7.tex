\documentclass[protokoll]{dvd}

\KOMAoptions{
    headwidth = 18cm,
    footwidth = 18cm,
}

\begin{document}

\title{Styrelsemöte 7}
\subtitle{2021}
\author{Styrelsen}
\date{2021-12-07}

\textbf{Datum:} \csname @date\endcsname\\
\textbf{Tid:} 12.00\\
\textbf{Plats:} Styrelserummet\\
\textbf{Styrelsemedlemmar:}
\begin{närvarande_förtroendevalda}
    \förtroendevald{Ordförande}{Albin Otterhäll}{Ja}
    \förtroendevald{Kassör}{Morgan Thowsen}{Ja}
    \förtroendevald{Vice ordförande}{Samuel Hammersberg}{Ja}
    \förtroendevald{Sekreterare}{Sebastian Selander}{Ja}{Avvek 12.40, under riktlinjer för logotyp}
    \förtroendevald{SAMO}{Tekla Siesjö}{Ja}
\end{närvarande_förtroendevalda}

% \textbf{Övriga medlemmar:}

\section{Öppnande av möte}

Mötet öppnas kl 12:02 av Albin Otterhäll

\section{Runda bordet}

    Runda bordet innebär att varje person berättar hur de känner sig.
    Man kan till exempel berätta att man är stressad på grund av en inlämning, irriterad på sin granne, eller bara väldigt glad därför att man ligger i fas med plugget.

\section{Formalia}

    \subsection{Styrelsens beslutbarhet}

        \blockquote[7 kap. 5 \S~första stycket i stadgan][]{
            Styrelsen är endast beslutsmässig då samtliga styrelsemedlemmar har fått kallelsen till styrelsemötet och minst hälften av styrelsemedlemmarna är närvarande.
            Ordförande eller vice ordförande måste vara närvarande när beslut tas.
        }

        Under förra möte styrelsemöte bestämdes det preliminärt datum för nästa möte. Eftersom ingen hade invändningar inför mötet fastställdes det implicit. 

        \subsubsection*{Beslut}

        \begin{attsatser}
            \item styrelsen har uppnått kraven i 7 kap. 5 § första stycket i stadgan och är därmed beslutbar.
        \end{attsatser}

    \subsection{Fastställande av mötesschema}

        För att styrelsen ska kunna fatta ett styrelsebeslut eller protokollföra en diskussion behöver punkten i mötesschemat där styrelsen ska fatta beslut vara inlagd eller föras in i mötesschemat senast vid den här punkten.

        \subsubsection*{Beslut}

        \begin{attsatser}
            \item mötesschemat fastställs utan några förändringar.
        \end{attsatser}


    \subsection{Val av protokolljusterare}

        Protokolljusterare har till uppgift att kontrollera att protokollet i slutändan reflekterar de faktiska besluten och diskussionerna som fördes under mötet.
        Utöver protokolljusteraren så ska mötesordförande och mötessekreteraren signera protokollet.
        Vid styrelsemöten ska det endast vara en justerare.
        Mötesordförande och mötessekreteraren kan inte vara justerare.

        \subsubsection*{Beslut}

        \begin{attsatser}
            \item Samuel Hammersberg väljs till protokolljusterare.
        \end{attsatser}


\newpage

\section{Rapporter}

    \subsection{Styrelseövergripande}

        \subsubsection{Handikappanpassning}
        Som vi alla vet hade Sebastian och Tekla en rundvandring med en elev tidigare i läsperioden.
        Under detta stötte vi på otroligt dålig handikappanpassning i Monaden.
        Vi har mailat Roger om detta men det verkar gå alldeles för långsamt.

\newpage

    \subsection{Ordförande}
        Något finns det säkert att säga! :)

    \subsection{Vice ordförande}

        \subsubsection*{Mummanyckel}
        Det var ett tag sedan nu, men vi har i alla fall en mummanyckel igen!
        Fastspänd på grafikkort till och med.


    \subsection{SAMO}

        \subsubsection*{Uppföljning på handikappanpassning}
            Har gjort ett par uppföljningar angående rampen men det är fortfarande inte fixat

\newpage

%\subsection{Kassör}

    \subsection{Sekreterare}

        \subsubsection*{Beslut per capsulam}
            Alla beslut vi tagit per capsulam under första terminen ska in i detta dokument!

\newpage

\section{Beslutsärenden}

    \subsection{Avvikelse}
    Sebastian lämnar mötet. Albin Otterhäll tar över som sekreterare.

    \subsection{Riktlinjer för logga}
        Då vi lovat att vi ska ha riktlinjer redo senast 17 december är det dags att faktiskt färdigställa dem.

           \subsubsection*{Förslag} 

           \begin{attsatser}
               \item Undvik "opassande" saker. Exempel på vad som är opassande är anspelningar på alkohol eller narkotika; sex; eller andra saker som kan strida mot någon relevant "värdegrund".
               \item Det ska finnas en version som inte strider mot upphovsrätten.
               \item Undvik smådetaljer i loggan.
               \item Loggan ska antingen ha formen av en rektangel, kvadrat, cirkel, pentagon, hexagon, eller en triangel.
               \item Det ska tydligt stå "Datavetenskap" och "Göteborgs universitet" eller "GU". "Göteborgs universitet" ska föredras framför "GU", men om det inte finns plats för hela namnet eller designen påverkas negativt så kan "GU" användas. Texten ska vara i en monospace font eller i en sans-serif font. Det senare har med vårat tillstånd för att få skriva "Göteborgs universitet" i loggan. Loggan kan antingen endast bestå av ett sigill med texten, eller bestå av ett sigill utan text och texten utanför (se GU:s logga som exempel: https://medarbetarportalen.gu.se/Kommunikation/visuell-identitet/grundprofil/logotyp/).
               \item Förslagen får inte heller likna GU:s logga.
               \item Förslagen ska presenteras minst två veckor innan nästa stämmomötet. Att uppdatera listan med förslag kontinuerligt under arbetets gång är att föredra.
               \item Förslagen ska vara tydliga och lämna lite plats för tolkning.
           \end{attsatser}

           \subsubsection*{Beslut}
           \begin{attsatser}
               \item ändra den femte attsatsen till 
                   \begin{displayquote}
                        Det ska tydligt stå antingen ((''Datavetenskap'' \lor ''DV'') \land (''Göteborgs universitet'' \lor ''GU'')) i sigillet.
                        Om det står ''DV'' i sigillet måste det skrivas ut ''Datavetenskap'' utanför sigillet i loggan.
                        ''Göteborgs universitet'' ska föredras framför ''GU'', men om det inte finns plats för hela namnet eller designen påverkas negativt så kan ''GU'' användas.
                        Texten ska vara i en monospace font eller i en sans-serif font.
                        Det senare har med vårat tillstånd för att få skriva ''Göteborgs universitet'' i loggan.
                        Loggan kan antingen endast bestå av ett sigill med texten, eller bestå av ett sigill utan text och texten utanför (se GU:s logga som exempel: https://medarbetarportalen.gu.se/Kommunikation/visuell-identitet/grundprofil/logotyp/).
                   \end{displayquote}
                \item bifalla samtliga föreslagna attsatser med ändringsyrkandet från attsats ett.
                \item fastställa att Sjösektionen har den fulaste sektionsloggan på Chalmers, och det är pinsamt att deras hemsida är en sida på Facebook där man måste vara inloggad för att kunna beskåda den.

           \end{attsatser}

           \subsection{Pub anordnat av ReKo}
            ReKo undrar om de hade kunnat få arrangera en pub i Monaden den 15 dec (onsdag), eller runt om kring där. Puben är öppen för alla, men de är OK om vi skulle ev. sätta som krav typ att första timmen så är det endast datavetare som får komma, och sedan öppnas det för andra.
            \subsubsection*{Förslag till beslut}

            \begin{attsatser}
                \item Att ReKo får Monadenpaxxning om de vill från klockan 17:00 den 15 december.
            \end{attsatser}
            \subsubsection*{Beslut}
            \begin{attsatser}
                \item vi beslutar att inte ta ut en disposition.
                \item vi begär att DV-studenter har förtur den första timmen.
            \end{attsatser}


\newpage

\section{Diskussioner}\label{sec:discussioner}

    \subsection{Räknestuga / kodstuga}
         Vi har fått ett mail av Jonathan angående räknestuga / kodstuga, och om detta hade varit något vi kan anordna för ettorna.
         Han föreslår att vi kanske ska ha några kodkvällar i "Nomaden" ;)

    \subsubsection{Discordserver}
        Albin har skapat en discordserver som vi kan använda för kommunikation internt mellan oss och alla andra aktiva (kommittémedlemmar o.s.v.).

        Det är bra om kommittéerna är ute i relativt god tid med bokningar då det kan påverka andra. Exempelvis inte mindre än 24h framförhållning.

\newpage

\section{Avslutande av möte}

    % \subsection{Tills nästa gång}
    %     \subsubsection{Ordförande}
    %     \subsubsection{Vice ordförande}
    %     \subsubsection{SAMO}
    %     \subsubsection{Kassör}
    %     \subsubsection{Sekreterare}

    \subsection{Mötesutvärdering}
    Mötet gick bra, men det blev väldigt mera slött senare under mötet.\\
    Men det var ändå trevligt. :)

    \subsection{Nästa möte}
    Inget mötesdatum bestämt för nästa styrelsemöte

    \subsection{Mötets avslutande}
    Ablin avslutar mötet klockan 13:27.

\styrelsesignaturer

\end{document}
