\documentclass[protokoll]{dvd}

\KOMAoptions{
    headwidth = 18cm,
    footwidth = 18cm,
}

\begin{document}

\title{Styrelsemöte 2}
\author{Styrelsen 2021}
\date{2021-08-26}

\textbf{Datum:} \csname @date\endcsname\\
\textbf{Tid:} 15:00\\
\textbf{Plats:} Styrelserummet\\
\textbf{Styrelsemedlemmar:}
\begin{närvarande}
    \medlem[Ordförande]{Albin Otterhäll}{Ja}
    \medlem[Kassör]{Morgan Thowsen}{Ja}
\end{närvarande}
\textbf{Övriga medlemmar:} Inga övriga medlemmar närvarande på mötet

\section{Öppnande av möte}

Mötet öppnas av Albin Otterhäll klockan 15:17.

\section{Formalia}

\subsection{Styrelsens beslutbarhet}

\blockquote[7 kap. 5 \S~första stycket i stadgan][]{%
    Styrelsen är endast beslutsmässig då samtliga styrelsemedlemmar har fått kallelsen till styrelsemötet och minst hälften av styrelsemedlemmarna är närvarande.
    Ordförande eller vice ordförande måste vara närvarande när beslut tas.
}

Albin och Morgan kom överens privat att hålla mötet den 26 augusti.

\begin{beslut}
    \item Styrelsen har uppnått kraven i 7 kap. 5 § första stycket i stadgan och är därmed beslutbar.
\end{beslut}

\subsection{Fastställande av mötesschema}

För att styrelsen ska kunna fatta ett styrelsebeslut eller protokollföra en diskussion behöver punkten i mötesschemat där styrelsen ska fatta beslut vara inlagd eller föras in i mötesschemat senast vid den här punkten.

\begin{beslut}
    \item Mötesschemat fastställs utan några förändringar.
\end{beslut}

\subsection{Val av mötessekreterare}

Då det endast sitter två personer i styrelsen, varav den ena är ordförande så går det inte ha en tredje person som mötessekreterare.
Ordförande har planerat mötet och skrivit möteshandlingarna, och har därmed mest koll på det planerade innehållet.

\begin{beslut}
    \item Albin Otterhäll väljs till mötessekreterare.
\end{beslut}

\subsection{Val av protokolljusterare}

Protokolljusterare har till uppgift att kontrollera att protokollet i slutändan reflekterar de faktiska besluten och diskussionerna som fördes under mötet.
Utöver protokolljusteraren så ska mötesordförande och mötessekreteraren signera protokollet.
Vid styrelsemöten ska det endast vara en justerare.
Mötesordförande och mötessekreteraren kan inte vara justerare.

\begin{beslut}
    \item Morgan Thowsen väljs till protokolljusterare.
\end{beslut}

\section{Rapporter}

Mottagningen drog igång måndagen den 16 augusti med grillning och aktiviteter under veckan.
Mycket uppskattat bland både phaddrar och recentiorer!

Styrelsen fortsätter jaga nya medlemmar till divisionen.

\subsection{Ordförande och vice ordförande}

I och med att medlemskap i divisionen är ett krav för att phaddra har medlemsantalet ökat från 10 medlemmr till 30 medlemmar på kort tid.

Tisdagen den 1 september kommer Albin gå på MAJ-möte med institutionen.
Det som kommer dryftas är hur studenternas arbetsmiljö är just nu, och hur den har varit under coronaåret.

Vi har även kommit överens med vice prefekt Roger Johannson att vi kommer genomföra en undersökning bland studenter under läsvecka 1 under läsperiod 2 om studiemiljön.

En mötesförfrågan har skickats till programledningen om att hålla ett möte i början av terminen.

\subsection{Övrig}

Ansökan om att få ett bankonto för ideella föreningar hos Swedbank har skickats, och har ankommit hos dem.
Nu väntar vi bara på svar.

\section{Beslutsärenden}

\subsection{Riva upp beslut om mottagningsbudget}

Under styrelsemöte 4 den 29 oktober 2020 beslutande Styrelsen 2020 under § 4.4 att DVRKs ḱassör ansvarade för mottagningsbudgeten som man får från IT-fakulteten.
Men under dagsläget så är mottagningsbudgeten ganska hårt knuten till divisionens budget, på grund av utlägg och så vidare.
Därför kommer ändå mottagningsbudgeten behöva kollas av divisionskassören.

\begin{beslut}
    \item Att beslutet som fattades av Styrelsen 2020 på Styrelsemöte 4 den 29 oktober 2020 under §4.4 hävs.
\end{beslut}

\subsection{Köpa partylampor till Monaden}

Lamporna som är färgade har bränts ut.
De är väldigt trevliga när man har till exempel sittningar.

\begin{beslut}
    \item Morgan tar på sig att inhandla nya partylampor.
\end{beslut}

\subsection{Köpa HDMI-sladd}

Vi behöver ha en HDMI sladd för att kunna ansluta våra Raspberry Pies till skärmvisare.

\begin{beslut}
    \item Morgan tar på sig att inhandla en ny HDMI-sladd.
\end{beslut}

\subsection{Låsa in brädspel i skåp}

Det har kommit till styrelsens kännedom att flera personer har ''lånat hem'' brädspel från divisionens spelhylla, utan att informera om detta.
Det har även tidigare förekommit stölder av brädspel.
I dagsläget är endast TV-spelen inlåsta i föreningsskåpen i Monaden.

\begin{beslut}
    \item Brädspelen ska inventeras, och vi ska bedöma om spelet ska låsas in på individuell basis.
\end{beslut}

\subsection{Extrastämma}

\begin{beslut}
    \item Extrastämma ska hållas onsdagen den 15 september klockan 18:00 i Monaden.
\end{beslut}

\subsection{Ändringar av personuppgiftsprinciper}

Divisionens rutiner och principer för personuppgifter behöver uppdateras för att spegla hur person- och kontaktuppgifter används idag.

Vi sparar medlemarnas e-mejladresser även för att skicka ut kallelser till stämmor, och eventuellt annan information som endast berör divisionsmedlemmar.
Det kan hända att personer som är medlemmar kanske inte får information från någon annan av Divisionens källor, så vi behöver skicka ut medlemsinformation direkt till dem.

Vi använder även formulär för anmälan till arrangemang, och när vi genomför undersökningar bland studenter på programmen som divisionen riktar sig mot.
Undersökningar görs oftast när man vill ta reda på hur studenternas studiearbetsmiljö ser ut.

Det är rimligt att endast spara medlemsinformaiton så länge man är medlem.
Vid förnyelse av medlemskap kan man bara kontakta medlemmarna och fråga om de önskar förnya sitt medlemskap.

Angående sparandet av mejl är det smidigare att personer tar bort mejl som inte behöver sparas när de slutar som förtroendevald i den posten där de har fått mejl.

I framtiden behöver vi utse någon person som ansvarar för att följa och kontrollera att utrensningar av register och mejl sker.

\begin{beslut}
    \item Ändra 3 kap. §1 från

    \blockquote[][]{%
        Vid medlemsregistrering behandlar Datavetenskapsdivisionen personuppgifter.
        Detta görs för att föreningen ska kunna kontrollera identiteter vid sammanträden samt för att kunna kontrollera att personen har kårmedlemskap. 
    }

    till

    \blockquote[][]{%
        Vid medlemsregistrering behandlar Datavetenskapsdivisionen personuppgifter och kontaktuppgifter.
        Detta görs för att föreningen ska kunna kontrollera identiteter vid sammanträden; för att kunna kontrollera att personen har kårmedlemskap; samt skicka ut medlemsinformation.
    }

    \item Lägga till en paragraf i 3 kap. mellan §3 och §4 med lydelsen

    \blockquote[][]{%
        Divisionen behandlar personuppgifter och kontaktuppgifter i samband med ifyllnad av formulär.
        Behandlingen sker för att kunna genomföra arrangemang och undersökningar i enlighet med divisionens syfte, som är definierat i stadgan.
        Den rättsliga grunden är att fullgöra divisionens syfte.
    }

    \item Ändra 3 kap. §5 andra stycket från

    \blockquote[][]{%
        Som sektionsförening vid Göta studentkår (organisationsnummer 857200-4144) kommer medlemsregistret att delas med kårens IT-sektions styrelse för att verifiera att samtliga medlemar i Datavetenskapsdivisionen är medlemmar i Göta studenkår.
    }

    till

    \blockquote[][]{%
        Som sektionsförening vid Göta studentkår (organisationsnummer 857200-4144) kommer medlemsregistret att delas med kåren för att verifiera att samtliga medlemar i Datavetenskapsdivisionen är medlemmar i Göta studenkår.
    }

    \item Ändra 3 kap. §6 första stycket från

    \blockquote[][]{%
        Personuppgifter som ingår i medlemsregistret sparas endast tills medlemsperiodens slut och ytterligare en månad.
        Den extra månaden där personuppgifterna sparas är till för att göra förnyandet av medlemskapet smidigare för båda parter.
    }

    till

    \blockquote[][]{%
        Personuppgifter som ingår i medlemsregistret sparas endast tills medlemsperiodens slut.
    }

    \item Ändra 3 kap. §6 tredje stycket från

    \blockquote[][]{%
        Handlingar av ringa eller tillfällig betydelse gallras i regel direkt eller senast efter sex månader.
    }

    till

    \blockquote[][]{%
        Handlingar av ringa eller tillfällig betydelse gallras i regel direkt eller senast efter tolv månader.
    }

\end{beslut}

\section{Diskussioner}\label{sec:discussioner}

\subsection{Större kök}

Det är väldigt tydligt att vårt kök är för litet när man lagar mat inför ett större arrangemang, till exempel en sittning.
Vi behöver mera arbetsytor, men främst plats för en större spis, och kanske till och med flera i framtiden.
Det är vice prefekt Roger Johansson som är huvudansvarig för Monaden.

\begin{åtagande}
    \item Albin dryftar frågan med Roger Johansson vid lämpligt tillfälle.
\end{åtagande}

\subsection{Skåpen i Monaden}

Flera personer har inkommit med önskemål om att hyra skåpen i Monaden.

\begin{åtagande}
    \item Morgan ska skriva ett utkast till nytt skåpskontrakt, för att sedan beslutas om på nästa styrelsemöte.
\end{åtagande}

\subsection{Pant}

Sedan 2020 finns det inte längre någon pantlåda i Monaden.
Den blev borttagen av städerskorna.
Anledningen var någon som arbetar med arbetsmiljö tyckte att den lyktade för mycket.
Städerskan sade att vi fick samla in pant i Monaden endast om panten tömdes flera gånger i veckan.

\begin{åtagande}
    \item Albin ska prata med DRust om huruvida vi kan komma fram till en smidig lösning.
\end{åtagande}

\section{Avslutande av möte}

Mötet avslutas av Albin Otterhäll klockan 16:42.

\styrelsesignaturer

\end{document}
