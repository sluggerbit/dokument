\documentclass[protokoll]{dvd}

\KOMAoptions{
    headwidth = 18cm,
    footwidth = 18cm,
}

\begin{document}

\title{Styrelsemöte 4}
\subtitle{2021}
\author{Styrelsen}
\date{2021-10-11}

\textbf{Datum:} \csname @date\endcsname\\
\textbf{Tid:} 12:00\\
\textbf{Plats:} Styrelserummet\\
\textbf{Styrelsemedlemmar:}
\begin{närvarande_förtroendevalda}
    \förtroendevald{Ordförande}{Albin Otterhäll}{}
    \förtroendevald{Kassör}{Morgan Thowsen}{}
    \förtroendevald{Vice ordförande}{Samuel Hammersberg}{}
    \förtroendevald{Sekreterare}{Sebastian Selander}{}
    \förtroendevald{SAMO}{Tekla Siesjö}{}
\end{närvarande_förtroendevalda}
% \textbf{Övriga medlemmar:}






\section{Öppnande av möte}

Mötet beräknas öppnas av Albin Otterhäll klockan 12:00.








\section{Runda bordet}

Runda bordet innebär att varje person berättar hur de känner sig.
Man kan till exempel berätta att man är stressad på grund av en inlämning, irriterad på sin granne, eller bara väldigt glad därför att man ligger i fas med plugget.









\section{Formalia}

\subsection{Styrelsens beslutbarhet}

\blockquote[7 kap. 5 \S~första stycket i stadgan][]{%
    Styrelsen är endast beslutsmässig då samtliga styrelsemedlemmar har fått kallelsen till styrelsemötet och minst hälften av styrelsemedlemmarna är närvarande.
    Ordförande eller vice ordförande måste vara närvarande när beslut tas.
}

Tisdag den 5 oktober föreslog Albin tid och datum för nästa styrelsemöte.
Den 7 oktober fastställdes datumet i styrelsens discordserver.

\subsubsection*{Förslag till beslut}

\begin{attsatser}
    \item Styrelsen har uppnått kraven i 7 kap. 5 § första stycket i stadgan och är därmed beslutbar.
\end{attsatser}




\subsection{Fastställande av mötesschema}

För att styrelsen ska kunna fatta ett styrelsebeslut eller protokollföra en diskussion behöver punkten i mötesschemat där styrelsen ska fatta beslut vara inlagd eller föras in i mötesschemat senast vid den här punkten.

\subsubsection*{Förslag till beslut}

\begin{attsatser}
    \item Mötesschemat fastställs utan några förändringar.
\end{attsatser}





\subsection{Val av protokolljusterare}

Protokolljusterare har till uppgift att kontrollera att protokollet i slutändan reflekterar de faktiska besluten och diskussionerna som fördes under mötet.
Utöver protokolljusteraren så ska mötesordförande och mötessekreteraren signera protokollet.
Vid styrelsemöten ska det endast vara en justerare.
Mötesordförande och mötessekreteraren kan inte vara justerare.

\subsubsection*{Förslag till beslut}

\begin{attsatser}
\item Samuel Hammersberg väljs till protokolljusterare
\end{attsatser}



\section{Rapporter}

\subsubsection{Låsa in brädspel}

Under senaste två styrelsemöten har denna punkt varit aktuell men ingen har åtagit sig att göra arbetet. 




\newpage

\subsection{Ordförande}


\subsubsection{Nästa ordförande}


\subsubsection{GSuite}

Albin har tittat på GSuite och försökt lösa problemen med det.



\newpage

\subsection{Vice ordförande}


\subsubsection{Inkommet förslag - vem är i Monaden}

Vi har fått förslag utifrån om att implementera något sätt att se vilka personer är i Monaden. Exempelvis som D eller IT har gjort det genom att kolla vilka datorer som är uppkopplade på routern.



\subsubsection{Styrelsen undviker ansvarsroller inom kommittéer}

Det har märkts att kommittéerna låter styrelsemedlemmar ta huvudroller under möten. Detta kan medföra att medlemmarna tror det är styrelsens åsikter som höjs istället för personens egna åsikter. 




\newpage

\subsection{Kassör}

\subsubsection{Bankkonto hos Swedbank}

\begin{description}[style=multiline, widest=00.00, align=left, leftmargin=2.5cm]
    \item[2021-10-11] Ingen uppdatering i ärendet i nuläget.
\end{description}


\subsubsection{Signaturer på protokoll}

I dagsläget signerar vi varje sida på protokollen. Varför behöver vi göra det, och skulle det vara möjligt att endast signera på sista sidan?




\newpage

\subsection{Sekreterare}

\subsubsection{Inkommen punkt från kommitté}

Festkommittén har har påpekat att två sittningar per månad kan ställas som ett krav. De tycker detta är orimligt och önskar att det tas upp.


\subsubsection{Styrelsemailen}

Styrelsemailen vidarebefordras till våra personliga mailadress just nu och detta gör att den personliga mailen blir rörig. Vi behöver hitta ett alternativ där styrelsemailen kan separeras från våra personliga mailadresser.




% \newpage

% \subsection{SAMO}





\newpage

\section{Beslutsärenden}


\subsubsection{Möte med Divisionsstämman}

\subsubsection{Datum för paxxning av Monaden}

\subsubsection{Ansvar för medlemsskap}



\section{Avslutande av möte}

\subsection{Mötesutvärdering}

\subsection{Nästa möte}

\subsection{Mötets avslutande}

Mötet beräknas avslutas klockan 13:00.

\styrelsesignaturer

\end{document}
