\documentclass[protokoll]{dvd}

\KOMAoptions{
    headwidth = 18cm,
    footwidth = 18cm,
}

\begin{document}

\title{Styrelsemöte 11}
\subtitle{2022}
\author{Styrelsen}
\date{2022-10-31}

\textbf{Datum:} \csname @date\endcsname\\
\textbf{Tid:} 15.15
\textbf{Plats:} Styrelserummet\\
\textbf{Styrelsemedlemmar:}
\begin{närvarande_förtroendevalda}
    \förtroendevald{Ordförande}{Samuel Otterhäll}{Ja}
    \förtroendevald{Kassör}{Morgan Thowsen}{Ja}
    \förtroendevald{Vice ordförande}{Vakant}{}
    \förtroendevald{Sekreterare}{Sebastian Selander}{Ja}
    \förtroendevald{SAMO}{Tekla Siesjö}{Ja}
\end{närvarande_förtroendevalda}

\section{Öppnande av möte}

Mötet öppnades 15.15 av Samuel Hammersberg

\section{Runda bordet}

Runda bordet innebär att varje person berättar hur de känner sig.
Man kan till exempel berätta att man är stressad på grund av en inlämning,
irriterad på sin granne, eller bara väldigt glad därför att man ligger i fas med plugget.

\section{Formalia}

\subsection{Styrelsens beslutbarhet}

\blockquote[7 kap. 5 \S~första stycket i stadgan][]{
    Styrelsen är endast beslutsmässig då samtliga styrelsemedlemmar har
    fått kallelsen till styrelsemötet och minst hälften av styrelsemedlemmarna är närvarande.
    Ordförande eller vice ordförande måste vara närvarande när beslut tas.
}

\subsubsection*{Beslut}

\begin{attsatser}
    \item Styrelsen har uppnått kraven i 7 kap.
          5 § första stycket i stadgan och är därmed beslutbar.
\end{attsatser}

\subsection{Fastställande av mötesschema}

För att styrelsen ska kunna fatta ett styrelsebeslut eller protokollföra en diskussion
behöver punkten i mötesschemat där styrelsen ska fatta beslut
vara inlagd eller föras in i mötesschemat senast vid den här punkten.

\subsubsection*{Beslut}

\begin{attsatser}
    \item mötesschemat fastställs utan några förändringar.
\end{attsatser}

\subsection{Val av protokolljusterare}

Protokolljusterare har till uppgift att kontrollera att protokollet
i slutändan reflekterar de faktiska besluten och diskussionerna som fördes under mötet.
Utöver protokolljusteraren så ska mötesordförande och mötessekreteraren signera protokollet.
Vid styrelsemöten ska det endast vara en justerare.
Mötesordförande och mötessekreteraren kan inte vara justerare.

\subsubsection*{Beslut}
\begin{attsatser}
    \item Tekla Siesjö väljs till protokolljusterare.
\end{attsatser}


\section{Rapport}
    \subsection{Aspning ConCats}
    ConCats är intresserade av en aspning. Planering för detta sker aktivt!


% \subsection{Ordförande}

% \subsection{Kassör}

% \subsection{Sekreterare}
% \subsection{SAMO}

% \section{Beslutspunkter}

\section{Diskussion}

    \subsection{Divisionsstämma}
    24 november har vi bestämt att stämman ska genomföras. Vad behöver vi göra innan dess?

    \begin{itemize}
        \item Dubbelkolla med nya kandidater till olika poster
        \item Ordna fram loggan med olika färger
        \item Nya intresserade styrelsemedlemmar
    \end{itemize}

    \subsection{Wiki}
    Det viktigaste är att det ska vara enkelt. Exempelvis att man använder RST (reStructured Text) och lägger upp på datavetenskapdivisionens github.
    Github gör det enkelt med versionshantering, men även lagom svårt att bara ändra saker hejvilt.


\section{Avslutande av möte}

% \subsection{Mötesutvärdering}

\subsection{Nästa möte}

Divisionsstämman.

\subsection{Mötets avslutande}

Mötet avslutades 16.15

\styrelsesignaturer

\end{document}
