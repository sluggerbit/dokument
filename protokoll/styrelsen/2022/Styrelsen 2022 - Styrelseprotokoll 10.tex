\documentclass[protokoll]{dvd}

\KOMAoptions{
    headwidth = 18cm,
    footwidth = 18cm,
}

\begin{document}

\title{Styrelsemöte 10}
\subtitle{2022}
\author{Styrelsen}
\date{2022-10-10}

\textbf{Datum:} \csname @date\endcsname\\
\textbf{Tid:} 13.15 \\
\textbf{Plats:} Styrelserummet\\
\textbf{Styrelsemedlemmar:}
\begin{närvarande_förtroendevalda}
    \förtroendevald{Ordförande}{Samuel Otterhäll}{Ja}
    \förtroendevald{Kassör}{Morgan Thowsen}{Ja}
    \förtroendevald{Vice ordförande}{Vakant}{}
    \förtroendevald{Sekreterare}{Sebastian Selander}{Ja}
    \förtroendevald{SAMO}{Tekla Siesjö}{Ja}
\end{närvarande_förtroendevalda}

\section{Öppnande av möte}

Mötet öppnades 13.15 av Samuel Hammersberg

\section{Runda bordet}

Runda bordet innebär att varje person berättar hur de känner sig.
Man kan till exempel berätta att man är stressad på grund av en inlämning,
irriterad på sin granne, eller bara väldigt glad därför att man ligger i fas med plugget.

\section{Formalia}

\subsection{Styrelsens beslutbarhet}

\blockquote[7 kap. 5 \S~första stycket i stadgan][]{
    Styrelsen är endast beslutsmässig då samtliga styrelsemedlemmar har
    fått kallelsen till styrelsemötet och minst hälften av styrelsemedlemmarna är närvarande.
    Ordförande eller vice ordförande måste vara närvarande när beslut tas.
}

\subsubsection*{Beslut}

\begin{attsatser}
    \item Styrelsen har uppnått kraven i 7 kap.
          5 § första stycket i stadgan och är därmed beslutbar.
\end{attsatser}

\subsection{Fastställande av mötesschema}

För att styrelsen ska kunna fatta ett styrelsebeslut eller protokollföra en diskussion
behöver punkten i mötesschemat där styrelsen ska fatta beslut
vara inlagd eller föras in i mötesschemat senast vid den här punkten.

\subsubsection*{Beslut}

\begin{attsatser}
    \item mötesschemat fastställs utan några förändringar.
\end{attsatser}

\subsection{Val av protokolljusterare}

Protokolljusterare har till uppgift att kontrollera att protokollet
i slutändan reflekterar de faktiska besluten och diskussionerna som fördes under mötet.
Utöver protokolljusteraren så ska mötesordförande och mötessekreteraren signera protokollet.
Vid styrelsemöten ska det endast vara en justerare.
Mötesordförande och mötessekreteraren kan inte vara justerare.

\subsubsection*{Beslut}
\begin{attsatser}
    \item Morgan Thowsen väljs till protokolljusterare
\end{attsatser}

\newpage

\section{Rapport}

\subsection{Styrelseövergripande}
    \subsubsection{Studienämnd}
    Ordförande har talat med intresserade personer samt Alex Gerdes. Arbetet är därmed på gång.

\newpage

% \subsection{Ordförande}
\subsection{Kassör}
    \subsubsection{DVRK}
    Då vi är tvungna att göra våra egna utlägg har det blivit väldigt
    mycket arbete för DVRK vad gäller utbetalningar.
    De har önskat att vi ordnar ett bankkort för att förenkla processen i framtiden.

% \subsection{Sekreterare}
% \subsection{SAMO}

% \section{Beslutspunkter}

\section{Diskussion}

    \subsection{Hyra/låna Monaden}
    Vi har blivit kontaktade av både SystemSex och ReKo med visat intresse att hyra Monaden.
    Enligt Roger Johansson får vi hyra ut/låna ut om vi vill.
    De vill endast veta vem som ska låna/hyra i sådana fall.

    \subsection{Divisionsstämma}
    Vi kör på en fet stämma i november. Nya styrelsekandidater behöver hittas och föreslås då några av oss försvinner.
    24 november preliminärt datum för stämman.

    \subsection{Kommittébudgetar}
    Häromveckan fick vi ett mail från Göta-Karl att vi har möjlighet till extra pengar för event under årets gång.
    Budgeteringen behöver endast vara en estimering och inte så värst detaljerad. \\
    Vi har även fått beröm av Göta över hur väl vi sköter vår ekonomi!

    \subsection{Bankkort}
    Diverse problem och fördelar har diskuterats, i dagsläget är vi relativt positivia till idéen.

\section{Avslutande av möte}

% \subsection{Mötesutvärdering}

\subsection{Nästa möte}

31 oktober.

\subsection{Mötets avslutande}

Mötet avslutades 14.30

\styrelsesignaturer

\end{document}
