\documentclass[protokoll]{dvd}

\KOMAoptions{
    headwidth = 18cm,
    footwidth = 18cm,
}

\begin{document}

\title{Styrelsemöte 5}
\subtitle{2022}
\author{Styrelsen}
\date{2022-04-21}

\textbf{Datum:} \csname @date\endcsname\\
\textbf{Tid:} 11:00\\
\textbf{Plats:} Styrelserummet och Discord\\
\textbf{Styrelsemedlemmar:}
\begin{närvarande_förtroendevalda}
    \förtroendevald{Ordförande}{Samuel Hammersberg}{Ja}
    \förtroendevald{Kassör}{Morgan Thowsen}{Ja}
    \förtroendevald{Vice ordförande}{\emph{Vakant}}{}
    \förtroendevald{Sekreterare}{Sebastian Selander}{Ja}
    \förtroendevald{SAMO}{Tekla Siesjö}{Ja}
\end{närvarande_förtroendevalda}

% \textbf{Övriga medlemmar:}

\section{Öppnande av möte}

Mötet beräknas öppnas av Samuel Hammersberg kl 11.00

\section{Runda bordet}

Runda bordet innebär att varje person berättar hur de känner sig.
Man kan till exempel berätta att man är stressad på grund av en inlämning, irriterad på sin granne, eller bara väldigt glad därför att man ligger i fas med plugget.

\newpage

\section{Formalia}

\subsection{Styrelsens beslutbarhet}

\blockquote[7 kap. 5 \S~första stycket i stadgan][]{%
    Styrelsen är endast beslutsmässig då samtliga styrelsemedlemmar har fått kallelsen till styrelsemötet och minst hälften av styrelsemedlemmarna är närvarande.
    Ordförande eller vice ordförande måste vara närvarande när beslut tas.
}

\subsubsection{Beslut}

\begin{attsatser}
    \item Styrelsen har uppnått kraven i 7 kap. 5 § första stycket i stadgan och är därmed beslutbar.
\end{attsatser}

\subsection{Fastställande av mötesschema}

För att styrelsen ska kunna fatta ett styrelsebeslut eller protokollföra en diskussion behöver punkten i mötesschemat där styrelsen ska fatta beslut vara inlagd eller föras in i mötesschemat senast vid den här punkten.

\subsubsection{Beslut}

\begin{attsatser}
    \item mötesschemat fastställs utan några förändringar.
\end{attsatser}

\subsection{Val av protokolljusterare}

Protokolljusterare har till uppgift att kontrollera att protokollet i slutändan reflekterar de faktiska besluten och diskussionerna som fördes under mötet.
Utöver protokolljusteraren så ska mötesordförande och mötessekreteraren signera protokollet.
Vid styrelsemöten ska det endast vara en justerare.
Mötesordförande och mötessekreteraren kan inte vara justerare.

\subsubsection{Beslut}
\begin{attsatser}
    \item Tekla Siesjö väljs till protokolljusterare
\end{attsatser}

\newpage

\section{Rapport}

\subsection{Styrelseövergripande}

\subsubsection{Finalsittning}
Finalsittningsmöte med Thor idag (21 april), Anthon är intresserad och går på det.

\subsubsection{Böcker}

Har fått frågan av alumn om vi är intresserade av kursböcker, vilket vi självklart är. \\
Lite fler böcker kommer snart stå i bokhyllan!

%\subsection{Ordförande}

\subsection{Kassör}

\subsubsection{Påskfirande}
Concats fick in 545 kr under påskeventet. Det var ett mycket trevligt event!

\subsubsection{Kvitto \& konto}
Kvitton inlämnade till Göta angående Mega6 kläder och bankavgift \\
3000 kr insatt från Göta, nuvarande balans 5733 kr.

%\subsection{Sekreterare}

%\subsection{SAMO}

\newpage

%\section{Beslutsärenden}

\newpage

\section{Diskussion}

\subsubsection{Stämma}
Dags att börja planera för en stämma, inom kort bör vi begära motioner från medlemmar.

\subsubsection{Möte med DVRK}
Tekla representerar styrelsen under mötet med DVRK.

\subsubsection{Handikappsanpassningsärendet}
Ärendet är bifogat till Alex då vi får dålig respons från Roger och Tord.

\newpage

\section{Avslutande av möte}

%\subsection{Arkiverade punkter}
%\begin{itemize}
%    \item
%    \item
%    \item
%\end{itemize}

\subsection{Mötesutvärdering}

\subsection{Nästa möte}
3 maj 13.15 - 14.15

\subsection{Mötets avslutande}

Mötet avslutades 12.00

\styrelsesignaturer

\end{document}
