\documentclass[protokoll]{dvd}

\KOMAoptions{
    headwidth = 18cm,
    footwidth = 18cm,
}

\begin{document}

\title{Styrelsemöte 1}
\subtitle{2022}
\author{Styrelsen}
\date{2022-01-20}

\textbf{Datum:} \csname @date\endcsname\\
\textbf{Tid:} 12.15\\
\textbf{Plats:} Styrelserummet\\
\textbf{Styrelsemedlemmar:}
\begin{närvarande_förtroendevalda}
    \förtroendevald{Ordförande}{Samuel Hammersberg}{Ja}
    \förtroendevald{Kassör}{Morgan Thowsen}{Ja}
    \förtroendevald{Vice ordförande}{Vakant}{Ja}
    \förtroendevald{Sekreterare}{Sebastian Selander}{Ja}
    \förtroendevald{SAMO}{Tekla Siesjö}{Ja}
\end{närvarande_förtroendevalda}

% \textbf{Övriga medlemmar:}

\section{Öppnande av möte}

Mötet öppnas kl 12.15 av Samuel Hammersberg

\section{Runda bordet}

    Runda bordet innebär att varje person berättar hur de känner sig.
    Man kan till exempel berätta att man är stressad på grund av en inlämning, irriterad på sin granne, eller bara väldigt glad därför att man ligger i fas med plugget.

\section{Formalia}

    \subsection{Styrelsens beslutbarhet}

        \blockquote[7 kap. 5 \S~första stycket i stadgan][]{
            Styrelsen är endast beslutsmässig då samtliga styrelsemedlemmar har fått kallelsen till styrelsemötet och minst hälften av styrelsemedlemmarna är närvarande.
            Ordförande eller vice ordförande måste vara närvarande när beslut tas.
        }

        Under förra möte styrelsemöte bestämdes det preliminärt datum för nästa möte. Eftersom ingen hade invändningar inför mötet fastställdes det implicit. 

        \subsubsection*{Beslut}

        \begin{attsatser}
            \item styrelsen har uppnått kraven i 7 kap. 5 § första stycket i stadgan och är därmed beslutbar.
        \end{attsatser}

    \subsection{Fastställande av mötesschema}

        För att styrelsen ska kunna fatta ett styrelsebeslut eller protokollföra en diskussion behöver punkten i mötesschemat där styrelsen ska fatta beslut vara inlagd eller föras in i mötesschemat senast vid den här punkten.

        \subsubsection*{Beslut}

        \begin{attsatser}
            \item mötesschemat fastställs utan några förändringar.
        \end{attsatser}


    \subsection{Val av protokolljusterare}

        Protokolljusterare har till uppgift att kontrollera att protokollet i slutändan reflekterar de faktiska besluten och diskussionerna som fördes under mötet.
        Utöver protokolljusteraren så ska mötesordförande och mötessekreteraren signera protokollet.
        Vid styrelsemöten ska det endast vara en justerare.
        Mötesordförande och mötessekreteraren kan inte vara justerare.

        \subsubsection*{Beslut}

        \begin{attsatser}
            \item Tekla Siesjö väljs till protokolljusterare
        \end{attsatser}


\newpage

\newpage

\section{Rapport}
   \subsection*{Ordförande} 
   Samuel har pratat med Alex Gerdes. Alex är för äskningen och vad den kan innebära för studenterna.
   De vill dock ha en mer detaljerad plan på vad pengarna ska spenderas på.
   Alex kommer med stor sannolikhet få en egen budget han kan dela ut.

\newpage

\newpage

\section{Beslutsärenden}
    Beslut nedan togs per capsulam under 2021 och behöver endast protokollföras.

    \subsection{Domän}
        Vi har fått information från dvare att monaden.se är ledig, och den kan vara
        trevlig att ha!

        \subsubsection{Beslut}
        \begin{attsatser}
            \item köpa domänen
        \end{attsatser}

    \subsection{Ansvar för GSuite}
        Någon kommer behöva ha ansvar för GSuite.

        \subsubsection{Beslut}
        \begin{attsatser}
            \item Albin Otterhäll fortfarande har ansvar över Gsuite
        \end{attsatser}

    \subsection{Konferenstelefon}
        Skulle vara trevlig att ha en egen konferenstelefon då kommittéer eller styrelsen kan behöva ha distansmöten.

        \subsubsection{Beslut}
        \begin{attsatser}
            \item äska för en konferenstelefon
        \end{attsatser}

    \subsection{GSuite}
        Hur ska GSuite fungera, vi skulle kunna ha subdomäner för styrelsen samt alla kommittéer, eller att alla använder dvet direkt.

        \subsubsection{Beslut}
        \begin{attsatser}
            \item vi använder domänen direkt - inga subdomänder
        \end{attsatser}

    \subsection{Covidrelaterat}
        Vaccinpass m.m. har diskuterats då antal fall ökar igen

        \subsubsection{Beslut}
        \begin{attsatser}
            \item vi följer FHMs rekommendationer och Götas krav.
        \end{attsatser}

    \subsection{Inköp av tejp}
        Monadenkommittén har fått efterfrågan på tejp för att sätta upp affischer

        Monadenkommittén undrar nu om de kan få beviljat att köpa 3 rullar a la 50m för ca 40euro. (frakt på 10euro)

        \subsubsection{Beslut}
        \begin{attsatser}
            \item önskan beviljas
        \end{attsatser}

    \subsection{Inköp av chips}
        Monadenkommittén önskar 200kr för chips till diverse event

        \subsubsection{Beslut}
        \begin{attsatser}
            \item önskan beviljas
        \end{attsatser}

\newpage

\newpage

\section{Avslutande av möte}

    \subsection{Mötesutvärdering}

    \subsection{Nästa möte}

    \subsection{Mötets avslutande}

\styrelsesignaturer

\end{document}
