\documentclass[protokoll]{dvd}

\KOMAoptions{
    headwidth = 18cm,
    footwidth = 18cm,
}

\begin{document}

\title{Styrelsemöte 7}
\subtitle{2022}
\author{Styrelsen}
\date{2022-05-10}

\textbf{Datum:} \csname @date\endcsname\\
\textbf{Tid:} 15:15\\
\textbf{Plats:} Styrelserummet och Discord\\
\textbf{Styrelsemedlemmar:}
\begin{närvarande_förtroendevalda}
    \förtroendevald{Ordförande}{Samuel Otterhäll}{Ja}
    \förtroendevald{Kassör}{Morgan Thowsen}{Ja}
    \förtroendevald{Vice ordförande}{Vakant}{}
    \förtroendevald{Sekreterare}{Sebastian Selander}{Ja}
    \förtroendevald{SAMO}{Tekla Siesjö}{Ja}
\end{närvarande_förtroendevalda}

% \textbf{Övriga medlemmar:}

\section{Öppnande av möte}

Mötet öppnades av Samuel Hammersberg kl 15.15

\section{Runda bordet}

Runda bordet innebär att varje person berättar hur de känner sig.
Man kan till exempel berätta att man är stressad på grund av en inlämning, irriterad på sin granne, eller bara väldigt glad därför att man ligger i fas med plugget.

\section{Formalia}

\subsection{Styrelsens beslutbarhet}

\blockquote[7 kap. 5 \S~första stycket i stadgan][]{%
    Styrelsen är endast beslutsmässig då samtliga styrelsemedlemmar har fått kallelsen till styrelsemötet och minst hälften av styrelsemedlemmarna är närvarande.
    Ordförande eller vice ordförande måste vara närvarande när beslut tas.
}

\subsubsection*{Förslag}

\begin{attsatser}
    \item Styrelsen har uppnått kraven i 7 kap. 5 § första stycket i stadgan och är därmed beslutbar.
\end{attsatser}

\subsection{Fastställande av mötesschema}

För att styrelsen ska kunna fatta ett styrelsebeslut eller protokollföra en diskussion behöver punkten i mötesschemat där styrelsen ska fatta beslut vara inlagd eller föras in i mötesschemat senast vid den här punkten.

\subsubsection*{Förslag}

\begin{attsatser}
    \item mötesschemat fastställs utan några förändringar.
\end{attsatser}

\subsection{Val av protokolljusterare}

Protokolljusterare har till uppgift att kontrollera att protokollet i slutändan reflekterar de faktiska besluten och diskussionerna som fördes under mötet.
Utöver protokolljusteraren så ska mötesordförande och mötessekreteraren signera protokollet.
Vid styrelsemöten ska det endast vara en justerare.
Mötesordförande och mötessekreteraren kan inte vara justerare.

\subsubsection*{Förslag}
\begin{attsatser}
    \item Tekla Siesjö väljs till protokolljusterare
\end{attsatser}

\section{Rapport}

% \subsection{Styrelseövergripande}

\newpage

\subsection{Ordförande}

\subsubsection{Massutskick över mail}
Ordnat så att alla i \@dvet kan skicka massutskicka över mail. Dessa mailadresser finns tillgängliga hos styrelsen.

\subsection{Kassör}
    \subsubsection{Arbete}
        \begin{itemize}
            \item Godkänt äskan isbrickor
            \item Bankgironr beställt och inkommet
            \item Kvitton inlämnat på Göta (raspberry pi) 
            \item Frågan om retroaktiv äskan ang. utlägg under våren skickad till Alex
            \item Bett Alex skicka pengar till DVD avseende Femme++ eventet
            \item Bokföringssystem är live
            \item Fakturasystem är live
            \item Bankgironummer införskaffat
            \item discord.monaden.se är ny permainvite för discorden
            \item medlem.monaden.se är ny länk till medlem.dvet.se
            \item megaman.monaden.se är en redirect till http://megaman.monaden/ (detta förutsätter att man är på Monadens nätverk)
            \item Hört av mig till Linode för att kolla om de kan erbjuda gratis hosting av vår hemsida
            \item Godkänt 177kr för 18st dricksglas till Monaden (Mega6) 
            \item Götapengar slut för året
        \end{itemize}

\subsection{Sekreterare}
    \subsubsection{Stadga}
    Uppdaterad stadga funns nu på drive och inväntar PR-review på github

\subsection{SAMO}
    \subsubsection{Studienämnd}
    Skickat mail till DNS om studienämnds-info och fått svar. Ska försöka boka möte med dem innan stämman, preliminärt 17 maj om det passar DNS.

\section{Beslutspunkter}

\subsection{Dvet}
Nu när vi fått tillgång till monaden.se har vi ett par alternativ vad gäller dvet.se
Tre mest relevanta alternativen är:

\begin{attsatser}
    \item ta bort dvet.se och ersätta helt med monaden.se
    \item ha kvar dvet.se och vidarebefordra till monaden.se
    \item vidarebefordra dvet.se till monaden.se
\end{attsatser}

\subsubsection*{Beslut}
    \begin{attsatser}
        \item ha kvar dvet.se och vidarebefodrar till monaden.se, samt behålla dvet som maildomän
    \end{attsatser}

\section{Diskussion}

\subsection{Verksamhetsrapport}
Vi behöver en verksamhetsrapport att infoga i stämman vilket behöver vara färdigt innan fredag.
Sebastian har skickat en bild i Discord över hur det såg ut senast.

\section{Avslutande av möte}
Mötet avslutades kl. 15.54

% \subsection{Mötesutvärdering}

\subsection{Nästa möte}
Tisdag 24 maj eller onsdag 25 maj

\subsection{Mötets avslutande}

\styrelsesignaturer

\end{document}
