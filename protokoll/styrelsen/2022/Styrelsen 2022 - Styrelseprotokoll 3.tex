\documentclass[protokoll]{dvd}

\KOMAoptions{
    headwidth = 18cm,
    footwidth = 18cm,
}

\begin{document}

\title{Styrelsemöte 3}
\subtitle{2022}
\author{Styrelsen}
\date{2022-03-22}

\textbf{Datum:} \csname @date\endcsname\\
\textbf{Tid:} 11.45\\
\textbf{Plats:} Discord\\
\textbf{Styrelsemedlemmar:}
\begin{närvarande_förtroendevalda}
    \förtroendevald{Ordförande}{Samuel Hammersberg}{Ja}
    \förtroendevald{Kassör}{Morgan Thowsen}{Ja}
    \förtroendevald{Vice ordförande}{Vakant}{}
    \förtroendevald{Sekreterare}{Sebastian Selander}{Ja}
    \förtroendevald{SAMO}{Tekla Siesjö}{Ja}
\end{närvarande_förtroendevalda}

% \textbf{Övriga medlemmar:}

\section{Öppnande av möte}

Mötet öppnas 13.25

\section{Runda bordet}

    Runda bordet innebär att varje person berättar hur de känner sig.
    Man kan till exempel berätta att man är stressad på grund av en inlämning, irriterad på sin granne, eller bara väldigt glad därför att man ligger i fas med plugget.

\section{Formalia}

    \subsection{Styrelsens beslutbarhet}

        \blockquote[7 kap. 5 \S~första stycket i stadgan][]{
            Styrelsen är endast beslutsmässig då samtliga styrelsemedlemmar har fått kallelsen till styrelsemötet och minst hälften av styrelsemedlemmarna är närvarande.
            Ordförande eller vice ordförande måste vara närvarande när beslut tas.
        }

        Under förra möte styrelsemöte bestämdes det preliminärt datum för nästa möte. Eftersom ingen hade invändningar inför mötet fastställdes det implicit. 

        \subsubsection*{Förslag}

        \begin{attsatser}
            \item styrelsen har uppnått kraven i 7 kap. 5 § första stycket i stadgan och är därmed beslutbar.
        \end{attsatser}

        \subsubsection*{Beslut}

        \begin{attsatser}
            \item attsatsen bifalles.
        \end{attsatser}

    \subsection{Fastställande av mötesschema}

        För att styrelsen ska kunna fatta ett styrelsebeslut eller protokollföra en diskussion behöver punkten i mötesschemat där styrelsen ska fatta beslut vara inlagd eller föras in i mötesschemat senast vid den här punkten.

        \subsubsection*{Förslag}

        \begin{attsatser}
            \item mötesschemat fastställs utan några förändringar.
        \end{attsatser}

        \subsection*{Beslut}

        \begin{attsatser}
            \item attsatsen fastställs
        \end{attsatser}


    \subsection{Val av protokolljusterare}

        Protokolljusterare har till uppgift att kontrollera att protokollet i slutändan reflekterar de faktiska besluten och diskussionerna som fördes under mötet.
        Utöver protokolljusteraren så ska mötesordförande och mötessekreteraren signera protokollet.
        Vid styrelsemöten ska det endast vara en justerare.
        Mötesordförande och mötessekreteraren kan inte vara justerare.

        \subsubsection*{Förslag}
            \begin{attsatser}
                \item \emph{Inga förslag innan mötet}
            \end{attsatser}

        \subsubsection*{Beslut}
            \begin{attsatser}
                \item Tekla Siesjö väljs till protokolljusterare
            \end{attsatser}


\newpage

\newpage

\section{Rapport}
   \subsection*{Styrelseövergripande}
    Lamporna har äntligen blivit lagade i Monaden!

    \subsubsection*{Vakmästeriet}
    Man kan kontakta vaktmästeriet för saker som vi tidigare kontaktat akademiska hus för. Kan vara bra att veta! \\
    Kontaktar man Chalmers support skickar de även uppdraget vidare till vaktmästeriet.

   \subsection*{Ordförande} 

       \subsubsection*{Möte med Thor}
       Mötet (mottagning) är uppskjutet på grund av tentor, men det är på gång.

   \subsection*{SAMO} 

       \subsubsection*{Husansvarig}
   % kika på mail

       \subsubsection*{Köket}
       De ska se över hela köket och alla vitvaror. Hoppet är på topp!

   % \subsection*{Kassör} 
   % \subsection*{Sekreterare} 

\newpage

\section{Beslutsärenden}

\section{Diskussion}

    \subsection*{Dokument för kontaktinfo, ansvarsområden}
    Startat ett dokument som en guide för framtida styrelser, en kort guide för vem man ska kontaka gällande olika situationer.

    \subsection*{Girls Code Club}
    Vi tar hand om att skicka ut en annons om ansökan av TAs. 

    \subsection*{Swedbank}
    Papperna ska skickas in till Swedbank ännu en gång, behöver bara hitta ett tillfälle då ansvariga är tillgängliga.

    \subsection*{Finalisera budget}
    Än så länge har vi fått in lite låga siffror, ett uppdaterat kalkylark kommer med åsikter från styrelsen. Intresse över att byta soffa och stolar har uppmärksammats, men vi i styrelsen tycker att vi borde gå via institutionen för det.

    \subsection*{Bokföringssystem}
    Två alternativ, open source, vilket skulle innebära att man lär sig hur det går till. Andra alternativet är att hyra ett system. Båda system skulle innebära en del bokföring för styrelsen, den betalda varianten hade varit enklare. Vi ska kika på ett betalt system, speciellt på önskan av DVarm

    \subsection*{Styrelseutökning}
    Vi behöver (till hösten) börja arbeta på att få in ytterligare personer till styrelsen, speciellt när alla av oss börjar med kanditat/masterprojekt efter årsskiftet.

\newpage

\newpage

\newpage

\section{Avslutande av möte}

    \subsection{Mötesutvärdering}
    Ta med alla icke arkiverade kanaler i discord som mötespunkter.

    \subsection{Nästa möte}
    2022-04-07, 11.00 - 12.00

    \subsection{Mötets avslutande}
    Mötet avslutades kl. 15.00

\styrelsesignaturer

\end{document}
