\documentclass[protokoll]{dvd}

\KOMAoptions{
    headwidth = 18cm,
    footwidth = 18cm,
}

\begin{document}

\title{Styrelsemöte 9}
\subtitle{2022}
\author{Styrelsen}
\date{2022-09-19}

\textbf{Datum:} \csname @date\endcsname\\
\textbf{Tid:} 13:00\\
\textbf{Plats:} Styrelserummet\\
\textbf{Styrelsemedlemmar:}
\begin{närvarande_förtroendevalda}
    \förtroendevald{Ordförande}{Samuel Otterhäll}{Ja}
    \förtroendevald{Kassör}{Morgan Thowsen}{Ja}
    \förtroendevald{Vice ordförande}{Vakant}{}
    \förtroendevald{Sekreterare}{Sebastian Selander}{Ja}
    \förtroendevald{SAMO}{Tekla Siesjö}{Ja}
\end{närvarande_förtroendevalda}

\section{Öppnande av möte}

Mötet öppnades av Samuel Hammersberg kl 13.05

\section{Runda bordet}

Runda bordet innebär att varje person berättar hur de känner sig.
Man kan till exempel berätta att man är stressad på grund av en inlämning,
irriterad på sin granne, eller bara väldigt glad därför att man ligger i fas med plugget.

\section{Formalia}

\subsection{Styrelsens beslutbarhet}

\blockquote[7 kap. 5 \S~första stycket i stadgan][]{%
    Styrelsen är endast beslutsmässig då samtliga styrelsemedlemmar har
    fått kallelsen till styrelsemötet och minst hälften av styrelsemedlemmarna är närvarande.
    Ordförande eller vice ordförande måste vara närvarande när beslut tas.
}

\subsubsection*{Beslut}

\begin{attsatser}
    \item Styrelsen har uppnått kraven i 7 kap.
          5 § första stycket i stadgan och är därmed beslutbar.
\end{attsatser}

\subsection{Fastställande av mötesschema}

För att styrelsen ska kunna fatta ett styrelsebeslut eller protokollföra en diskussion
behöver punkten i mötesschemat där styrelsen ska fatta beslut
vara inlagd eller föras in i mötesschemat senast vid den här punkten.

\subsubsection*{Förslag}

\begin{attsatser}
    \item mötesschemat fastställs utan några förändringar.
\end{attsatser}

\subsection{Val av protokolljusterare}

Protokolljusterare har till uppgift att kontrollera att protokollet
i slutändan reflekterar de faktiska besluten och diskussionerna som fördes under mötet.
Utöver protokolljusteraren så ska mötesordförande och mötessekreteraren signera protokollet.
Vid styrelsemöten ska det endast vara en justerare.
Mötesordförande och mötessekreteraren kan inte vara justerare.

\subsubsection*{Förslag}
\begin{attsatser}
    \item Tekla Siesjö väljs till protokolljusterare.
\end{attsatser}

\section{Rapport}

\subsection{Styrelseövergripande}
    \subsection{Mottagning 2022}
    Mottagningen har gått bra. Det har varit roligt att träffa de nya studenter, samt även de gamla!
    I år var det även mycket fler masterstudenter än tidigare år vilket har varit roligt!

\newpage

% \subsection{Ordförande}

\subsection{Kassör}
    \subsection{Ekonomi}
    Inget märkvärdigt att rapportera direkt. Den mesta informationen finns hos DVRK.
    DVRK samt ConCats har också fått in lite pengar på sina egna evenemang. Det är trevligt!

    \subsection{Linode}
    Detta ärendet släpar lite då det inte har samma prioritering som andra ärenden.

\subsection{Sekreterare}
    \subsection{Skåp}
    Uthyrningen av skåp har påbörjat. Alla har inte skrivit kontrakt än dock. Det går frammåt!

\subsection{SAMO}
    \subsection{Generellt}
    Har fått frågan om att sitta med i Götas styrelsemöten.
    Tackar dock nej till detta erbjudande då vi redan får det mesta av den informationen.
    I början av mottagning presenterade vi även vem och vad en SAMO är.

% \section{Beslutspunkter}

\section{Diskussion}

    \subsection{Studienämnd}
    Vi har potentiellt någon intreserad för att leda denna kommitté. Detta behöver bara röstas på
    under stämman i sådana fall.

    \subsection{Logga}
    Då vi under senaste stämman röstade om logga har arbetet på detta pågått lite smått.
    Vi har fått ett förslag från en grafiker på Göta. Arbetet på detta ska fortsätta.
    Exempelvis 

    \subsection{Handikappanpassning}
    Det är dags att kontakta Roger Johansson (igen..)
    vad gäller handikappsanpassningen i Monaden, mer specifikt köket.
    Detta kan även presenteras på programrådet, då får vi ännu ett sätt att få våra
    tankar hörda.

    \subsection{Divisionsstämma}
    Hösten kräver en divisionsstämma! Lite diskussion om logga återkommer, samt val om ordförande
    till diverse kommittéer. Vi kör divisionsstämma torsdag sjätte oktober (2022-10-06).

\section{Avslutande av möte}

% \subsection{Mötesutvärdering}

\subsection{Nästa möte}

Vi kör nästa styrelsemöte tredje oktober (2022-10-03)

\subsection{Mötets avslutande}

Mötet avslutades kl. 14.10

\styrelsesignaturer

\end{document}
