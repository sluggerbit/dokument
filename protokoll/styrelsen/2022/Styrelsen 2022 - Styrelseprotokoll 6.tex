\documentclass[protokoll]{dvd}

\KOMAoptions{
    headwidth = 18cm,
    footwidth = 18cm,
}

\begin{document}

\title{Styrelsemöte 6}
\subtitle{2022}
\author{Styrelsen}
\date{2022-05-03}

\textbf{Datum:} \csname @date\endcsname\\
\textbf{Tid:} 13:15\\
\textbf{Plats:} Styrelserummet och Discord\\
\textbf{Styrelsemedlemmar:}
\begin{närvarande_förtroendevalda}
    \förtroendevald{Ordförande}{Samuel Otterhäll}{Ja}
    \förtroendevald{Kassör}{Morgan Thowsen}{Ja}
    \förtroendevald{Vice ordförande}{Vakant}{}
    \förtroendevald{Sekreterare}{Sebastian Selander}{Ja}
    \förtroendevald{SAMO}{Tekla Siesjö}{Ja}
\end{närvarande_förtroendevalda}

% \textbf{Övriga medlemmar:}

\section{Öppnande av möte}

Mötet beräknas öppnades av Samuel Hammersberg kl 13.21

\section{Runda bordet}

Runda bordet innebär att varje person berättar hur de känner sig.
Man kan till exempel berätta att man är stressad på grund av en inlämning, irriterad på sin granne, eller bara väldigt glad därför att man ligger i fas med plugget.

\section{Formalia}

\subsection{Styrelsens beslutbarhet}

\blockquote[7 kap. 5 \S~första stycket i stadgan][]{%
    Styrelsen är endast beslutsmässig då samtliga styrelsemedlemmar har fått kallelsen till styrelsemötet och minst hälften av styrelsemedlemmarna är närvarande.
    Ordförande eller vice ordförande måste vara närvarande när beslut tas.
}

\subsubsection*{Förslag}

\begin{attsatser}
    \item Styrelsen har uppnått kraven i 7 kap. 5 § första stycket i stadgan och är därmed beslutbar.
\end{attsatser}

\subsection{Fastställande av mötesschema}

För att styrelsen ska kunna fatta ett styrelsebeslut eller protokollföra en diskussion behöver punkten i mötesschemat där styrelsen ska fatta beslut vara inlagd eller föras in i mötesschemat senast vid den här punkten.

\subsubsection*{Förslag}

\begin{attsatser}
    \item mötesschemat fastställs utan några förändringar.
\end{attsatser}

\subsection{Val av protokolljusterare}

Protokolljusterare har till uppgift att kontrollera att protokollet i slutändan reflekterar de faktiska besluten och diskussionerna som fördes under mötet.
Utöver protokolljusteraren så ska mötesordförande och mötessekreteraren signera protokollet.
Vid styrelsemöten ska det endast vara en justerare.
Mötesordförande och mötessekreteraren kan inte vara justerare.

\subsubsection*{Förslag}
\begin{attsatser}
    \item Morgan Thowsen väljs till protokolljusterare.
\end{attsatser}

\section{Rapport}

\subsection{Styrelseövergripande}

\subsection{Ordförande}

\subsection{Kassör}
    \subsubsection{HiQ-event}
    Verkar som att vi är tvungna att sköta fakturering för eventet på egen hand då Göta har backat ut i sista sekund.

\subsection{Sekreterare}
    \subsubsection{Papercut credits}
    Har köpt papercut credits för 75 kr.

\subsection{SAMO}
    \subsubsection{Bekräftat medlemmar}
    Påminnelse om att bli medlem i divisionen är utskickad i discord. Har även kollat medlemsskap på blivande ordförande för DVRK.

\section{Beslutsärenden}

\section{Diskussion}
    \subsection{Studienämnd}
    Under programteamsmötet nämnde Jonathan en önskan om att vi lyfter fram idéen om en studienämnd. Studienämndens syfte skulle vara att

    \subsection{Programteamsmöte}
    Alex har sagt under mötet att han önskar att vi ska äska för de saker vi vill ha istället för att vi ska få ut en summa pengar som vi sedan kan spendera.
    Under mötet pratade vi även om arbetsmarknadskommittéen, det gick inte riktigt som önskat, men vi ska göra ett andra försök.

    \subsection{Nästa stämma}
    17 maj är det dags för stämma, senast 10 maj ska motioner ha kommit till styrelsen och vi skickar ut protokollet 13 maj!

    \subsection{Mottagning}
    Vi behöver ta en gruppbild på oss som styrelse inför mottagning som sedan kan inkluderas i mottagningshäftet. DVRK önskar även att styrelsen håller i ett evenemang för studenter.


\section{Avslutande av möte}

\subsection{Mötesutvärdering}

\subsection{Nästa möte}

\subsection{Mötets avslutande}

Mötet avslutades 14.29

\styrelsesignaturer

\end{document}
