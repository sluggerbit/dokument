\documentclass[protokoll]{dvd}

\KOMAoptions{
    headwidth = 18cm,
    footwidth = 18cm,
}

\begin{document}

\title{Styrelsemöte 2}
\subtitle{2022}
\author{Styrelsen}
\date{2022-02-09}

\textbf{Datum:} \csname @date\endcsname\\
\textbf{Tid:} 11.45\\
\textbf{Plats:} Discord\\
\textbf{Styrelsemedlemmar:}
\begin{närvarande_förtroendevalda}
    \förtroendevald{Ordförande}{Samuel Hammersberg}{Ja}
    \förtroendevald{Kassör}{Morgan Thowsen}{Ja}
    \förtroendevald{Vice ordförande}{Vakant}{}
    \förtroendevald{Sekreterare}{Sebastian Selander}{Ja}
    \förtroendevald{SAMO}{Tekla Siesjö}{Ja}
\end{närvarande_förtroendevalda}

% \textbf{Övriga medlemmar:}

\section{Öppnande av möte}

Mötet öppnas kl 11:52 av Samuel Hammersberg

\section{Runda bordet}

    Runda bordet innebär att varje person berättar hur de känner sig.
    Man kan till exempel berätta att man är stressad på grund av en inlämning, irriterad på sin granne, eller bara väldigt glad därför att man ligger i fas med plugget.

\section{Formalia}

    \subsection{Styrelsens beslutbarhet}

        \blockquote[7 kap. 5 \S~första stycket i stadgan][]{
            Styrelsen är endast beslutsmässig då samtliga styrelsemedlemmar har fått kallelsen till styrelsemötet och minst hälften av styrelsemedlemmarna är närvarande.
            Ordförande eller vice ordförande måste vara närvarande när beslut tas.
        }

        Under förra möte styrelsemöte bestämdes det preliminärt datum för nästa möte. Eftersom ingen hade invändningar inför mötet fastställdes det implicit. 

        \subsubsection*{Förslag}

        \begin{attsatser}
            \item styrelsen har uppnått kraven i 7 kap. 5 § första stycket i stadgan och är därmed beslutbar.
        \end{attsatser}

        \subsubsection*{Beslut}

        \begin{attsatser}
            \item attsatsen bifalles.
        \end{attsatser}

    \subsection{Fastställande av mötesschema}

        För att styrelsen ska kunna fatta ett styrelsebeslut eller protokollföra en diskussion behöver punkten i mötesschemat där styrelsen ska fatta beslut vara inlagd eller föras in i mötesschemat senast vid den här punkten.

        \subsubsection*{Förslag}

        \begin{attsatser}
            \item mötesschemat fastställs utan några förändringar.
        \end{attsatser}


    \subsection{Val av protokolljusterare}

        Protokolljusterare har till uppgift att kontrollera att protokollet i slutändan reflekterar de faktiska besluten och diskussionerna som fördes under mötet.
        Utöver protokolljusteraren så ska mötesordförande och mötessekreteraren signera protokollet.
        Vid styrelsemöten ska det endast vara en justerare.
        Mötesordförande och mötessekreteraren kan inte vara justerare.

        \subsubsection*{Förslag}
            \begin{attsatser}
                \item Morgan Thowsen väljs till protokolljusterare.
            \end{attsatser}


\newpage

\newpage

\section{Rapport}
   \subsection*{Styrelseövergripande}
        Ett av låsen till de stora skåpen har tyvärr gått sönder. Samuel införskaffade ett nytt på samma dag som tur var!
   % \subsection*{Ordförande} 
   % \subsection*{SAMO} 
   % \subsection*{Kassör} 
   % \subsection*{Sekreterare} 

\newpage

\section{Beslutsärenden}
    \subsection*{Tillgång till skåp}
    Just nu har endast styrelsen och ordförande för ConCats (Miranda) tillgång skåpen i monaden.
    Ska vi utöka detta eller ordna en alternativ lösning?

    \subsection*{Beslut}
        \begin{attsatser}
            \item mummanyckeln förvaras i ett separat skåp som endast styrelsen och ordförande för ConCats har tillgång till.
            \item ändra kod till vänstra skåpet.
        \end{attsatser}

\section{Diskussion}

    \subsection*{Medlemskap i Göta}
    Just nu finns det inget formellt krav på att man måste vara med i Göta för att få vara kommittémedlem.
    Vi kan tyvärr inte enkelt formellt införa ett krav heller, dock ska vi skicka ut en stark rekommendation att kommittéordföranden kräver att sina medlemmar även är medlemmar i Göta.
    Exempelvis blir det svårt att äska för overaller för sju personer om endast tre personer är medlemmar i Göta.

    \subsection*{Möte med kommittéordförande}
    Möte med kommittéordförande är planerat, där ska främst budget diskuteras då vi i princip fått godkännande av Alex att pengar är på väg in.

    \subsection*{Coronarestriktioner}
    Eftersom regeringen slopat alla coronarestriktioner kräver vi inget speciellt av våra medlemmar generellt idag. Kommittéer får själva bestämma hur de vill göra.

\newpage

\newpage

\newpage

\section{Avslutande av möte}

    % \subsection{Mötesutvärdering}

    \subsection{Nästa möte}
    Preliminärt andra mars.

    \subsection{Mötets avslutande}
    Mötet avslutades kl. 13:05

\styrelsesignaturer

\end{document}
